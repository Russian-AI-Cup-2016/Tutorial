% Added by DocsPostProcessor:
\setlength{\parskip}{0.125in}
% \\ Added by DocsPostProcessor.

% Replaced by DocsPostProcessor:
% \documentclass[11pt]{report}
\documentclass[a4paper]{report}
% \\ Replaced by DocsPostProcessor.

\def\bl{\mbox{}\newline\mbox{}\newline{}}

% Added by DocsPostProcessor:
\usepackage{amssymb,latexsym,amsmath,amscd,mathtext,ifthen}
\usepackage[unicode]{hyperref}
\usepackage{listings}
\usepackage{literat}
\usepackage{graphicx}
\usepackage[top = 2.0cm]{geometry}
\usepackage[T1,T2A]{fontenc}
\usepackage[utf8]{inputenc}
\usepackage[english,russian]{babel}

\input glyphtounicode
\pdfgentounicode=1

\setlength{\skip\footins}{0.5cm}
\setlength{\footnotesep}{0.5cm}

\newcommand\abs[1]{\left|#1\right|}
% \\ Added by DocsPostProcessor.

\usepackage{ifthen}
\newcommand{\hide}[2]{
\ifthenelse{\equal{#1}{inherited}}%
{}%
{}%
}
\newcommand{\entityintro}[3]{%
  \hbox to \hsize{%
    \vbox{%
      \hbox to .2in{}%
    }%
    {\bf #1}%
    \dotfill\pageref{#2}%
  }
  \makebox[\hsize]{%
    \parbox{.4in}{}%
    \parbox[l]{5in}{%
      \vspace{1mm}\it%
      #3%
      \vspace{1mm}%
    }%
  }%
}
\newcommand{\isep}[0]{%
\setlength{\itemsep}{-.4ex}
}
\newcommand{\sld}[0]{%
\setlength{\topsep}{0em}
\setlength{\partopsep}{0em}
\setlength{\parskip}{0em}
\setlength{\parsep}{-1em}
}
\newcommand{\headref}[3]{%
\ifthenelse{#1 = 1}{%
\addcontentsline{toc}{section}{\hspace{\qquad}\protect\numberline{}{#3}}%
}{}%
\ifthenelse{#1 = 2}{%
\addcontentsline{toc}{subsection}{\hspace{\qquad}\protect\numerline{}{#3}}%
}{}%
\ifthenelse{#1 = 3}{%
\addcontentsline{toc}{subsubsection}{\hspace{\qquad}\protect\numerline{}{#3}}%
}{}%
\label{#3}%
\makebox[\textwidth][l]{#2 #3}%
}%
\newcommand{\membername}[1]{{\it #1}\linebreak}
\newcommand{\divideents}[1]{\vskip -1em\indent\rule{2in}{.5mm}}
\newcommand{\refdefined}[1]{
\expandafter\ifx\csname r@#1\endcsname\relax
\relax\else
{$($ in \ref{#1}, page \pageref{#1}$)$}
\fi}
\newcommand{\startsection}[4]{
\gdef\classname{#2}
\subsection{\label{#3}{\bf {\sc #1} #2}}{
\rule[1em]{\hsize}{4pt}\vskip -1em
\vskip .1in 
#4
}%
}
\newcommand{\startsubsubsection}[2]{
\subsubsection{\sc #1}{%
\rule[1em]{\hsize}{2pt}%
#2}
}
\usepackage{color}

% Replaced by DocsPostProcessor:
% \date{DoNotModifyDateHere}
\date{Ноябрь --- декабрь, 2016}
% \\ Replaced by DocsPostProcessor.


% Removed by DocsPostProcessor:
% \pagestyle{myheadings}
% \\ Removed by DocsPostProcessor.

\addtocontents{toc}{\protect\def\protect\packagename{}}
\addtocontents{toc}{\protect\def\protect\classname{}}
\markboth{\protect\packagename -- \protect\classname}{\protect\packagename -- \protect\classname}
\oddsidemargin 0in
\evensidemargin 0in
% \topmargin -.8in
\chardef\bslash=`\\
\textheight 9.4in
\textwidth 6.5in

% Replaced by DocsPostProcessor:
% \title{DoNotModifyTitleHere}
\title{
\includegraphics{images/raic-2015-logo-302x140.png}\\
\vspace{0.61in}
\textsc{\Huge CodeWizards 2016}\\
\vspace{0.50in}
\textsc{\LARGE Правила}\\
\textsc{\small Версия 1.2.0}\\
\vspace{2.75in}
\includegraphics{images/mail-ru-and-ssu-logo.png}
}
% \\ Replaced by DocsPostProcessor.

\begin{document}
\maketitle
\sloppy
\raggedright
\tableofcontents

% Added by DocsPostProcessor:
\markboth{qqq}{{\footnotesize CodeWizards 2016}}
\input TutorialRu.tex
\raggedright
\sloppy
% \\ Added by DocsPostProcessor.

\gdef\packagename{}
\gdef\classname{}
\newpage
\def\packagename{model}
\chapter{\bf Package model}{
\vskip -.25in
\hbox to \hsize{\it Package Contents\hfil Page}
\rule{\hsize}{.7mm}
\vskip .13in
\hbox{\bf Classes}
\entityintro{ActionType}{l0}{Возможные действия волшебника.}
\entityintro{Bonus}{l1}{Класс, определяющий бонус --- неподвижный полезный объект.}
\entityintro{BonusType}{l2}{Тип бонуса.}
\entityintro{Building}{l3}{Класс, определяющий строение.}
\entityintro{BuildingType}{l4}{Тип строения.}
\entityintro{CircularUnit}{l5}{Базовый класс для определения круглых объектов.}
\entityintro{Faction}{l6}{Фракция юнита.}
\entityintro{Game}{l7}{Предоставляет доступ к различным игровым константам.}
\entityintro{LaneType}{l8}{Тип дорожки.}
\entityintro{LivingUnit}{l9}{Класс, определяющий живого юнита круглой формы.}
\entityintro{Message}{l10}{Класс определяет сообщение, которое верховный волшебник ({\tt wizard.master}) может отправлять другим членам
 фракции, используя телепатическую связь.}
\entityintro{Minion}{l11}{Класс, определяющий приспешника волшебника одной из фракций.}
\entityintro{MinionType}{l12}{Тип приспешника.}
\entityintro{Move}{l13}{Стратегия игрока может управлять волшебником посредством установки свойств объекта данного класса.}
\entityintro{Player}{l14}{Содержит данные о текущем состоянии игрока.}
\entityintro{Projectile}{l15}{Класс, определяющий снаряд.}
\entityintro{ProjectileType}{l16}{Тип снаряда.}
\entityintro{SkillType}{l17}{Тип умения.}
\entityintro{Status}{l18}{Магический статус, влияющий на живого юнита.}
\entityintro{StatusType}{l19}{Тип магического статуса, влияющего на живого юнита.}
\entityintro{Tree}{l20}{Класс, определяющий дерево.}
\entityintro{Unit}{l21}{Базовый класс для определения объектов (<<юнитов>>) на игровом поле.}
\entityintro{Wizard}{l22}{Класс, определяющий волшебника.}
\entityintro{World}{l23}{Этот класс описывает игровой мир.}

% Removed by DocsPostProcessor:
% \vskip .1in
% \rule{\hsize}{.7mm}
% \\ Removed by DocsPostProcessor.

\vskip .1in
\newpage
\section{Classes}{
\startsection{Class}{ActionType}{l0}{%
{\small Возможные действия волшебника.
 \bl 
 Волшебник не может совершать новые действия, если он ещё не восстановился после своего предыдущего действия (значение
 {\tt wizard.remainingActionCooldownTicks} больше {\tt 0}).
 \bl 
 Волшебник не может использовать действие, если оно ещё не восстановилось после его предыдущего применения (значение
 {\tt remainingCooldownTicksByAction[actionType.ordinal()]} больше {\tt 0}).}
\vskip .1in 
\startsubsubsection{Declaration}{
\fbox{\vbox{
\hbox{\vbox{\small public final 
class 
ActionType}}
\noindent\hbox{\vbox{{\bf extends} Enum}}
}}}
\startsubsubsection{Fields}{
\begin{itemize}
\item{
public static final ActionType NONE\begin{itemize}\item{\vskip -.9ex Ничего не делать.}\end{itemize}
}
\item{
public static final ActionType STAFF\begin{itemize}\item{\vskip -.9ex Ударить посохом.
 \bl 
 Атака поражает все живые объекты в секторе от {\tt -game.staffSector $/$ 2.0} до {\tt game.staffSector $/$ 2.0}.
 Расстояние от центра волшебника до центра цели не должно превышать значение
 {\tt game.staffRange + livingUnit.radius}.}\end{itemize}
}
\item{
public static final ActionType MAGIC\_MISSILE\begin{itemize}\item{\vskip -.9ex Создать магическую ракету.
 \bl 
 Магическая ракета является базовым заклинанием любого волшебника. Наносит урон при прямом попадании.
 \bl 
 При создании магической ракеты её центр совпадает с центром волшебника, направление определяется как
 {\tt wizard.angle + move.castAngle}, а абсолютное значение скорости равно {\tt game.magicMissileSpeed}.
 Столкновения магического снаряда и создавшего его волшебника игнорируются.
 \bl 
 Требует {\tt game.magicMissileManacost} единиц магической энергии.}\end{itemize}
}
\item{
public static final ActionType FROST\_BOLT\begin{itemize}\item{\vskip -.9ex Создать ледяную стрелу.
 \bl 
 Ледяная стрела наносит урон при прямом попадании, а также замораживает цель.
 \bl 
 При создании ледяной стрелы её центр совпадает с центром волшебника, направление определяется как
 {\tt wizard.angle + move.castAngle}, а абсолютное значение скорости равно {\tt game.frostBoltSpeed}.
 Столкновения магического снаряда и создавшего его волшебника игнорируются.
 \bl 
 Требует {\tt game.frostBoltManacost} единиц магической энергии и изучения умения {\tt FROST\_BOLT}.}\end{itemize}
}
\item{
public static final ActionType FIREBALL\begin{itemize}\item{\vskip -.9ex Создать огненный шар.
 \bl 
 Огненный шар взрывается при достижении максимальной дальности полёта или при столкновении с живым объектом.
 Наносит урон всем близлежащим живым объектам, а также поджигает их.
 \bl 
 При создании огненного шара его центр совпадает с центром волшебника, направление определяется как
 {\tt wizard.angle + move.castAngle}, а абсолютное значение скорости равно {\tt game.fireballSpeed}.
 Столкновения магического снаряда и создавшего его волшебника игнорируются.
 \bl 
 Требует {\tt game.fireballManacost} единиц магической энергии и изучения умения {\tt FIREBALL}.}\end{itemize}
}
\item{
public static final ActionType HASTE\begin{itemize}\item{\vskip -.9ex Временно ускорить волшебника с идентификатором {\tt move.statusTargetId} или самого себя, если такой волшебник
 не найден.
 \bl 
 Требует {\tt game.hasteManacost} единиц магической энергии и изучения умения {\tt HASTE}.}\end{itemize}
}
\item{
public static final ActionType SHIELD\begin{itemize}\item{\vskip -.9ex На время создать магический щит вокруг волшебника с идентификатором {\tt move.statusTargetId} или самого себя,
 если такой волшебник не найден.
 \bl 
 Требует {\tt game.shieldManacost} единиц магической энергии и изучения умения {\tt SHIELD}.}\end{itemize}
}
\end{itemize}
}
\hide{inherited}{
\startsubsubsection{Methods inherited from class {\tt Enum}}{
\par{\small 
\refdefined{l24}\vskip -2em
\begin{itemize}
\item{\vskip -1.9ex 
\membername{clone}
{\tt protected final Object {\bf clone}(  )
}%end signature
}%end item
\divideents{compareTo}
\item{\vskip -1.9ex 
\membername{compareTo}
{\tt public final int {\bf compareTo}( {\tt Enum } {\bf arg0} )
}%end signature
}%end item
\divideents{equals}
\item{\vskip -1.9ex 
\membername{equals}
{\tt public final boolean {\bf equals}( {\tt Object } {\bf arg0} )
}%end signature
}%end item
\divideents{finalize}
\item{\vskip -1.9ex 
\membername{finalize}
{\tt protected final void {\bf finalize}(  )
}%end signature
}%end item
\divideents{getDeclaringClass}
\item{\vskip -1.9ex 
\membername{getDeclaringClass}
{\tt public final Class {\bf getDeclaringClass}(  )
}%end signature
}%end item
\divideents{hashCode}
\item{\vskip -1.9ex 
\membername{hashCode}
{\tt public final int {\bf hashCode}(  )
}%end signature
}%end item
\divideents{name}
\item{\vskip -1.9ex 
\membername{name}
{\tt public final String {\bf name}(  )
}%end signature
}%end item
\divideents{ordinal}
\item{\vskip -1.9ex 
\membername{ordinal}
{\tt public final int {\bf ordinal}(  )
}%end signature
}%end item
\divideents{toString}
\item{\vskip -1.9ex 
\membername{toString}
{\tt public String {\bf toString}(  )
}%end signature
}%end item
\divideents{valueOf}
\item{\vskip -1.9ex 
\membername{valueOf}
{\tt public static Enum {\bf valueOf}( {\tt Class } {\bf arg0},
{\tt String } {\bf arg1} )
}%end signature
}%end item
\end{itemize}
}}
}
}
\startsection{Class}{Bonus}{l1}{%
{\small Класс, определяющий бонус --- неподвижный полезный объект. Содержит также все свойства круглого юнита.}
\vskip .1in 
\startsubsubsection{Declaration}{
\fbox{\vbox{
\hbox{\vbox{\small public 
class 
Bonus}}
\noindent\hbox{\vbox{{\bf extends} CircularUnit}}
}}}

% Removed by DocsPostProcessor:
% \startsubsubsection{Constructors}{
% \vskip -2em
% \begin{itemize}
% \item{\vskip -1.9ex 
% \membername{Bonus}
% {\tt public {\bf Bonus}( {\tt long } {\bf id},
% {\tt double } {\bf x},
% {\tt double } {\bf y},
% {\tt double } {\bf speedX},
% {\tt double } {\bf speedY},
% {\tt double } {\bf angle},
% {\tt Faction } {\bf faction},
% {\tt double } {\bf radius},
% {\tt BonusType } {\bf type} )
% \label{l25}\label{l26}}%end signature
% }%end item
% \end{itemize}
% }
% \\ Removed by DocsPostProcessor.

\startsubsubsection{Methods}{
\vskip -2em
\begin{itemize}
\item{\vskip -1.9ex 
\membername{getType}
{\tt public BonusType {\bf getType}(  )
\label{l27}\label{l28}}%end signature
\begin{itemize}
\sld
\item{{\bf Returns} - 
Возвращает тип бонуса. 
}%end item
\end{itemize}
}%end item
\end{itemize}
}
\hide{inherited}{
\startsubsubsection{Methods inherited from class {\tt CircularUnit}}{
\par{\small 
\refdefined{l5}\vskip -2em
\begin{itemize}
\item{\vskip -1.9ex 
\membername{getRadius}
{\tt public double {\bf getRadius}(  )
}%end signature
\begin{itemize}
\sld
\item{{\bf Returns} - 
Возвращает радиус объекта. 
}%end item
\end{itemize}
}%end item
\end{itemize}
}}
\startsubsubsection{Methods inherited from class {\tt Unit}}{
\par{\small 
\refdefined{l21}\vskip -2em
\begin{itemize}
\item{\vskip -1.9ex 
\membername{getAngle}
{\tt public final double {\bf getAngle}(  )
}%end signature
\begin{itemize}
\sld
\item{{\bf Returns} - 
Возвращает угол поворота объекта в радианах. Нулевой угол соответствует направлению оси абсцисс.
 Положительные значения соответствуют повороту по часовой стрелке. 
}%end item
\end{itemize}
}%end item
\divideents{getAngleTo}
\item{\vskip -1.9ex 
\membername{getAngleTo}
{\tt public double {\bf getAngleTo}( {\tt double } {\bf x},
{\tt double } {\bf y} )
}%end signature
\begin{itemize}
\sld
\item{
\sld
{\bf Parameters}
\sld\isep
  \begin{itemize}
\sld\isep
   \item{
\sld
{\tt x} - X-координата точки.}
   \item{
\sld
{\tt y} - Y-координата точки.}
  \end{itemize}
}%end item
\item{{\bf Returns} - 
Возвращает ориентированный угол [{\tt -PI}, {\tt PI}] между направлением
 данного объекта и вектором из центра данного объекта к указанной точке. 
}%end item
\end{itemize}
}%end item
\divideents{getAngleTo}
\item{\vskip -1.9ex 
\membername{getAngleTo}
{\tt public double {\bf getAngleTo}( {\tt Unit } {\bf unit} )
}%end signature
\begin{itemize}
\sld
\item{
\sld
{\bf Parameters}
\sld\isep
  \begin{itemize}
\sld\isep
   \item{
\sld
{\tt unit} - Объект, к центру которого необходимо определить угол.}
  \end{itemize}
}%end item
\item{{\bf Returns} - 
Возвращает ориентированный угол [{\tt -PI}, {\tt PI}] между направлением
 данного объекта и вектором из центра данного объекта к центру указанного объекта. 
}%end item
\end{itemize}
}%end item
\divideents{getDistanceTo}
\item{\vskip -1.9ex 
\membername{getDistanceTo}
{\tt public double {\bf getDistanceTo}( {\tt double } {\bf x},
{\tt double } {\bf y} )
}%end signature
\begin{itemize}
\sld
\item{
\sld
{\bf Parameters}
\sld\isep
  \begin{itemize}
\sld\isep
   \item{
\sld
{\tt x} - X-координата точки.}
   \item{
\sld
{\tt y} - Y-координата точки.}
  \end{itemize}
}%end item
\item{{\bf Returns} - 
Возвращает расстояние до точки от центра данного объекта. 
}%end item
\end{itemize}
}%end item
\divideents{getDistanceTo}
\item{\vskip -1.9ex 
\membername{getDistanceTo}
{\tt public double {\bf getDistanceTo}( {\tt Unit } {\bf unit} )
}%end signature
\begin{itemize}
\sld
\item{
\sld
{\bf Parameters}
\sld\isep
  \begin{itemize}
\sld\isep
   \item{
\sld
{\tt unit} - Объект, до центра которого необходимо определить расстояние.}
  \end{itemize}
}%end item
\item{{\bf Returns} - 
Возвращает расстояние от центра данного объекта до центра указанного объекта. 
}%end item
\end{itemize}
}%end item
\divideents{getFaction}
\item{\vskip -1.9ex 
\membername{getFaction}
{\tt public Faction {\bf getFaction}(  )
}%end signature
\begin{itemize}
\sld
\item{{\bf Returns} - 
Возвращает фракцию, к которой принадлежит данный юнит. 
}%end item
\end{itemize}
}%end item
\divideents{getId}
\item{\vskip -1.9ex 
\membername{getId}
{\tt public long {\bf getId}(  )
}%end signature
\begin{itemize}
\sld
\item{{\bf Returns} - 
Возвращает уникальный идентификатор объекта. 
}%end item
\end{itemize}
}%end item
\divideents{getSpeedX}
\item{\vskip -1.9ex 
\membername{getSpeedX}
{\tt public final double {\bf getSpeedX}(  )
}%end signature
\begin{itemize}
\sld
\item{{\bf Returns} - 
Возвращает X-составляющую скорости объекта. Ось абсцисс направлена слева направо.
 \bl 
 Для юнитов, способных мгновенно менять свою скорость, возвращается значение перемещения за последний тик. 
}%end item
\end{itemize}
}%end item
\divideents{getSpeedY}
\item{\vskip -1.9ex 
\membername{getSpeedY}
{\tt public final double {\bf getSpeedY}(  )
}%end signature
\begin{itemize}
\sld
\item{{\bf Returns} - 
Возвращает Y-составляющую скорости объекта. Ось ординат направлена сверху вниз.
 \bl 
 Для юнитов, способных мгновенно менять свою скорость, возвращается значение перемещения за последний тик. 
}%end item
\end{itemize}
}%end item
\divideents{getX}
\item{\vskip -1.9ex 
\membername{getX}
{\tt public final double {\bf getX}(  )
}%end signature
\begin{itemize}
\sld
\item{{\bf Returns} - 
Возвращает X-координату центра объекта. Ось абсцисс направлена слева направо. 
}%end item
\end{itemize}
}%end item
\divideents{getY}
\item{\vskip -1.9ex 
\membername{getY}
{\tt public final double {\bf getY}(  )
}%end signature
\begin{itemize}
\sld
\item{{\bf Returns} - 
Возвращает Y-координату центра объекта. Ось ординат направлена сверху вниз. 
}%end item
\end{itemize}
}%end item
\end{itemize}
}}
}
}
\startsection{Class}{BonusType}{l2}{%
{\small Тип бонуса.
 \bl 
 В дополнение к основному эффекту каждый подобранный бонус даёт игроку {\tt game.bonusScoreAmount} баллов, а
 волшебник получает такое же количество опыта.}
\vskip .1in 
\startsubsubsection{Declaration}{
\fbox{\vbox{
\hbox{\vbox{\small public final 
class 
BonusType}}
\noindent\hbox{\vbox{{\bf extends} Enum}}
}}}
\startsubsubsection{Fields}{
\begin{itemize}
\item{
public static final BonusType EMPOWER\begin{itemize}\item{\vskip -.9ex На некоторое время значительно увеличивает урон, наносимый волшебником при ударах посохом и попаданиях магических
 снарядов в цель.}\end{itemize}
}
\item{
public static final BonusType HASTE\begin{itemize}\item{\vskip -.9ex Значительно ускоряет перемещение волшебника.
 \bl 
 Аналогично действию одноимённого заклинания, но длительность статуса выше.}\end{itemize}
}
\item{
public static final BonusType SHIELD\begin{itemize}\item{\vskip -.9ex Уменьшает урон, получаемый волшебником от прямых попаданий магических снарядов.
 \bl 
 Аналогично действию одноимённого заклинания, но длительность статуса выше.}\end{itemize}
}
\end{itemize}
}
\hide{inherited}{
\startsubsubsection{Methods inherited from class {\tt Enum}}{
\par{\small 
\refdefined{l24}\vskip -2em
\begin{itemize}
\item{\vskip -1.9ex 
\membername{clone}
{\tt protected final Object {\bf clone}(  )
}%end signature
}%end item
\divideents{compareTo}
\item{\vskip -1.9ex 
\membername{compareTo}
{\tt public final int {\bf compareTo}( {\tt Enum } {\bf arg0} )
}%end signature
}%end item
\divideents{equals}
\item{\vskip -1.9ex 
\membername{equals}
{\tt public final boolean {\bf equals}( {\tt Object } {\bf arg0} )
}%end signature
}%end item
\divideents{finalize}
\item{\vskip -1.9ex 
\membername{finalize}
{\tt protected final void {\bf finalize}(  )
}%end signature
}%end item
\divideents{getDeclaringClass}
\item{\vskip -1.9ex 
\membername{getDeclaringClass}
{\tt public final Class {\bf getDeclaringClass}(  )
}%end signature
}%end item
\divideents{hashCode}
\item{\vskip -1.9ex 
\membername{hashCode}
{\tt public final int {\bf hashCode}(  )
}%end signature
}%end item
\divideents{name}
\item{\vskip -1.9ex 
\membername{name}
{\tt public final String {\bf name}(  )
}%end signature
}%end item
\divideents{ordinal}
\item{\vskip -1.9ex 
\membername{ordinal}
{\tt public final int {\bf ordinal}(  )
}%end signature
}%end item
\divideents{toString}
\item{\vskip -1.9ex 
\membername{toString}
{\tt public String {\bf toString}(  )
}%end signature
}%end item
\divideents{valueOf}
\item{\vskip -1.9ex 
\membername{valueOf}
{\tt public static Enum {\bf valueOf}( {\tt Class } {\bf arg0},
{\tt String } {\bf arg1} )
}%end signature
}%end item
\end{itemize}
}}
}
}
\startsection{Class}{Building}{l3}{%
{\small Класс, определяющий строение. Фракционные строения самостоятельно атакуют противников в определённом радиусе.
 \bl 
 Строения не могут быть заморожены ({\tt FROZEN}).}
\vskip .1in 
\startsubsubsection{Declaration}{
\fbox{\vbox{
\hbox{\vbox{\small public 
class 
Building}}
\noindent\hbox{\vbox{{\bf extends} LivingUnit}}
}}}

% Removed by DocsPostProcessor:
% \startsubsubsection{Constructors}{
% \vskip -2em
% \begin{itemize}
% \item{\vskip -1.9ex 
% \membername{Building}
% {\tt public {\bf Building}( {\tt long } {\bf id},
% {\tt double } {\bf x},
% {\tt double } {\bf y},
% {\tt double } {\bf speedX},
% {\tt double } {\bf speedY},
% {\tt double } {\bf angle},
% {\tt Faction } {\bf faction},
% {\tt double } {\bf radius},
% {\tt int } {\bf life},
% {\tt int } {\bf maxLife},
% {\tt Status[]} {\bf statuses},
% {\tt BuildingType } {\bf type},
% {\tt double } {\bf visionRange},
% {\tt double } {\bf attackRange},
% {\tt int } {\bf damage},
% {\tt int } {\bf cooldownTicks},
% {\tt int } {\bf remainingActionCooldownTicks} )
% \label{l29}\label{l30}}%end signature
% }%end item
% \end{itemize}
% }
% \\ Removed by DocsPostProcessor.

\startsubsubsection{Methods}{
\vskip -2em
\begin{itemize}
\item{\vskip -1.9ex 
\membername{getAttackRange}
{\tt public double {\bf getAttackRange}(  )
\label{l31}\label{l32}}%end signature
\begin{itemize}
\sld
\item{{\bf Returns} - 
Возвращает максимальное расстояние (от центра до центра),
 на котором данное строение может атаковать другие объекты. 
}%end item
\end{itemize}
}%end item
\divideents{getCooldownTicks}
\item{\vskip -1.9ex 
\membername{getCooldownTicks}
{\tt public int {\bf getCooldownTicks}(  )
\label{l33}\label{l34}}%end signature
\begin{itemize}
\sld
\item{{\bf Returns} - 
Возвращает интервал между атаками. 
}%end item
\end{itemize}
}%end item
\divideents{getDamage}
\item{\vskip -1.9ex 
\membername{getDamage}
{\tt public int {\bf getDamage}(  )
\label{l35}\label{l36}}%end signature
\begin{itemize}
\sld
\item{{\bf Returns} - 
Возвращает урон одной атаки. 
}%end item
\end{itemize}
}%end item
\divideents{getRemainingActionCooldownTicks}
\item{\vskip -1.9ex 
\membername{getRemainingActionCooldownTicks}
{\tt public int {\bf getRemainingActionCooldownTicks}(  )
\label{l37}\label{l38}}%end signature
\begin{itemize}
\sld
\item{{\bf Returns} - 
Возвращает количество тиков, оставшееся до следующей атаки. 
}%end item
\end{itemize}
}%end item
\divideents{getType}
\item{\vskip -1.9ex 
\membername{getType}
{\tt public BuildingType {\bf getType}(  )
\label{l39}\label{l40}}%end signature
\begin{itemize}
\sld
\item{{\bf Returns} - 
Возвращает тип строения. 
}%end item
\end{itemize}
}%end item
\divideents{getVisionRange}
\item{\vskip -1.9ex 
\membername{getVisionRange}
{\tt public double {\bf getVisionRange}(  )
\label{l41}\label{l42}}%end signature
\begin{itemize}
\sld
\item{{\bf Returns} - 
Возвращает максимальное расстояние (от центра до центра),
 на котором данное строение обнаруживает другие объекты. 
}%end item
\end{itemize}
}%end item
\end{itemize}
}
\hide{inherited}{
\startsubsubsection{Methods inherited from class {\tt LivingUnit}}{
\par{\small 
\refdefined{l9}\vskip -2em
\begin{itemize}
\item{\vskip -1.9ex 
\membername{getLife}
{\tt public int {\bf getLife}(  )
}%end signature
\begin{itemize}
\sld
\item{{\bf Returns} - 
Возвращает текущее количество жизненной энергии. 
}%end item
\end{itemize}
}%end item
\divideents{getMaxLife}
\item{\vskip -1.9ex 
\membername{getMaxLife}
{\tt public int {\bf getMaxLife}(  )
}%end signature
\begin{itemize}
\sld
\item{{\bf Returns} - 
Возвращает максимальное количество жизненной энергии. 
}%end item
\end{itemize}
}%end item
\divideents{getStatuses}
\item{\vskip -1.9ex 
\membername{getStatuses}
{\tt public Status[] {\bf getStatuses}(  )
}%end signature
\begin{itemize}
\sld
\item{{\bf Returns} - 
Возвращает магические статусы, влияющие на живого юнита. 
}%end item
\end{itemize}
}%end item
\end{itemize}
}}
\startsubsubsection{Methods inherited from class {\tt CircularUnit}}{
\par{\small 
\refdefined{l5}\vskip -2em
\begin{itemize}
\item{\vskip -1.9ex 
\membername{getRadius}
{\tt public double {\bf getRadius}(  )
}%end signature
\begin{itemize}
\sld
\item{{\bf Returns} - 
Возвращает радиус объекта. 
}%end item
\end{itemize}
}%end item
\end{itemize}
}}
\startsubsubsection{Methods inherited from class {\tt Unit}}{
\par{\small 
\refdefined{l21}\vskip -2em
\begin{itemize}
\item{\vskip -1.9ex 
\membername{getAngle}
{\tt public final double {\bf getAngle}(  )
}%end signature
\begin{itemize}
\sld
\item{{\bf Returns} - 
Возвращает угол поворота объекта в радианах. Нулевой угол соответствует направлению оси абсцисс.
 Положительные значения соответствуют повороту по часовой стрелке. 
}%end item
\end{itemize}
}%end item
\divideents{getAngleTo}
\item{\vskip -1.9ex 
\membername{getAngleTo}
{\tt public double {\bf getAngleTo}( {\tt double } {\bf x},
{\tt double } {\bf y} )
}%end signature
\begin{itemize}
\sld
\item{
\sld
{\bf Parameters}
\sld\isep
  \begin{itemize}
\sld\isep
   \item{
\sld
{\tt x} - X-координата точки.}
   \item{
\sld
{\tt y} - Y-координата точки.}
  \end{itemize}
}%end item
\item{{\bf Returns} - 
Возвращает ориентированный угол [{\tt -PI}, {\tt PI}] между направлением
 данного объекта и вектором из центра данного объекта к указанной точке. 
}%end item
\end{itemize}
}%end item
\divideents{getAngleTo}
\item{\vskip -1.9ex 
\membername{getAngleTo}
{\tt public double {\bf getAngleTo}( {\tt Unit } {\bf unit} )
}%end signature
\begin{itemize}
\sld
\item{
\sld
{\bf Parameters}
\sld\isep
  \begin{itemize}
\sld\isep
   \item{
\sld
{\tt unit} - Объект, к центру которого необходимо определить угол.}
  \end{itemize}
}%end item
\item{{\bf Returns} - 
Возвращает ориентированный угол [{\tt -PI}, {\tt PI}] между направлением
 данного объекта и вектором из центра данного объекта к центру указанного объекта. 
}%end item
\end{itemize}
}%end item
\divideents{getDistanceTo}
\item{\vskip -1.9ex 
\membername{getDistanceTo}
{\tt public double {\bf getDistanceTo}( {\tt double } {\bf x},
{\tt double } {\bf y} )
}%end signature
\begin{itemize}
\sld
\item{
\sld
{\bf Parameters}
\sld\isep
  \begin{itemize}
\sld\isep
   \item{
\sld
{\tt x} - X-координата точки.}
   \item{
\sld
{\tt y} - Y-координата точки.}
  \end{itemize}
}%end item
\item{{\bf Returns} - 
Возвращает расстояние до точки от центра данного объекта. 
}%end item
\end{itemize}
}%end item
\divideents{getDistanceTo}
\item{\vskip -1.9ex 
\membername{getDistanceTo}
{\tt public double {\bf getDistanceTo}( {\tt Unit } {\bf unit} )
}%end signature
\begin{itemize}
\sld
\item{
\sld
{\bf Parameters}
\sld\isep
  \begin{itemize}
\sld\isep
   \item{
\sld
{\tt unit} - Объект, до центра которого необходимо определить расстояние.}
  \end{itemize}
}%end item
\item{{\bf Returns} - 
Возвращает расстояние от центра данного объекта до центра указанного объекта. 
}%end item
\end{itemize}
}%end item
\divideents{getFaction}
\item{\vskip -1.9ex 
\membername{getFaction}
{\tt public Faction {\bf getFaction}(  )
}%end signature
\begin{itemize}
\sld
\item{{\bf Returns} - 
Возвращает фракцию, к которой принадлежит данный юнит. 
}%end item
\end{itemize}
}%end item
\divideents{getId}
\item{\vskip -1.9ex 
\membername{getId}
{\tt public long {\bf getId}(  )
}%end signature
\begin{itemize}
\sld
\item{{\bf Returns} - 
Возвращает уникальный идентификатор объекта. 
}%end item
\end{itemize}
}%end item
\divideents{getSpeedX}
\item{\vskip -1.9ex 
\membername{getSpeedX}
{\tt public final double {\bf getSpeedX}(  )
}%end signature
\begin{itemize}
\sld
\item{{\bf Returns} - 
Возвращает X-составляющую скорости объекта. Ось абсцисс направлена слева направо.
 \bl 
 Для юнитов, способных мгновенно менять свою скорость, возвращается значение перемещения за последний тик. 
}%end item
\end{itemize}
}%end item
\divideents{getSpeedY}
\item{\vskip -1.9ex 
\membername{getSpeedY}
{\tt public final double {\bf getSpeedY}(  )
}%end signature
\begin{itemize}
\sld
\item{{\bf Returns} - 
Возвращает Y-составляющую скорости объекта. Ось ординат направлена сверху вниз.
 \bl 
 Для юнитов, способных мгновенно менять свою скорость, возвращается значение перемещения за последний тик. 
}%end item
\end{itemize}
}%end item
\divideents{getX}
\item{\vskip -1.9ex 
\membername{getX}
{\tt public final double {\bf getX}(  )
}%end signature
\begin{itemize}
\sld
\item{{\bf Returns} - 
Возвращает X-координату центра объекта. Ось абсцисс направлена слева направо. 
}%end item
\end{itemize}
}%end item
\divideents{getY}
\item{\vskip -1.9ex 
\membername{getY}
{\tt public final double {\bf getY}(  )
}%end signature
\begin{itemize}
\sld
\item{{\bf Returns} - 
Возвращает Y-координату центра объекта. Ось ординат направлена сверху вниз. 
}%end item
\end{itemize}
}%end item
\end{itemize}
}}
}
}
\startsection{Class}{BuildingType}{l4}{%
{\small Тип строения.}
\vskip .1in 
\startsubsubsection{Declaration}{
\fbox{\vbox{
\hbox{\vbox{\small public final 
class 
BuildingType}}
\noindent\hbox{\vbox{{\bf extends} Enum}}
}}}
\startsubsubsection{Fields}{
\begin{itemize}
\item{
public static final BuildingType GUARDIAN\_TOWER\begin{itemize}\item{\vskip -.9ex Охранная башня.}\end{itemize}
}
\item{
public static final BuildingType FACTION\_BASE\begin{itemize}\item{\vskip -.9ex База фракции.}\end{itemize}
}
\end{itemize}
}
\hide{inherited}{
\startsubsubsection{Methods inherited from class {\tt Enum}}{
\par{\small 
\refdefined{l24}\vskip -2em
\begin{itemize}
\item{\vskip -1.9ex 
\membername{clone}
{\tt protected final Object {\bf clone}(  )
}%end signature
}%end item
\divideents{compareTo}
\item{\vskip -1.9ex 
\membername{compareTo}
{\tt public final int {\bf compareTo}( {\tt Enum } {\bf arg0} )
}%end signature
}%end item
\divideents{equals}
\item{\vskip -1.9ex 
\membername{equals}
{\tt public final boolean {\bf equals}( {\tt Object } {\bf arg0} )
}%end signature
}%end item
\divideents{finalize}
\item{\vskip -1.9ex 
\membername{finalize}
{\tt protected final void {\bf finalize}(  )
}%end signature
}%end item
\divideents{getDeclaringClass}
\item{\vskip -1.9ex 
\membername{getDeclaringClass}
{\tt public final Class {\bf getDeclaringClass}(  )
}%end signature
}%end item
\divideents{hashCode}
\item{\vskip -1.9ex 
\membername{hashCode}
{\tt public final int {\bf hashCode}(  )
}%end signature
}%end item
\divideents{name}
\item{\vskip -1.9ex 
\membername{name}
{\tt public final String {\bf name}(  )
}%end signature
}%end item
\divideents{ordinal}
\item{\vskip -1.9ex 
\membername{ordinal}
{\tt public final int {\bf ordinal}(  )
}%end signature
}%end item
\divideents{toString}
\item{\vskip -1.9ex 
\membername{toString}
{\tt public String {\bf toString}(  )
}%end signature
}%end item
\divideents{valueOf}
\item{\vskip -1.9ex 
\membername{valueOf}
{\tt public static Enum {\bf valueOf}( {\tt Class } {\bf arg0},
{\tt String } {\bf arg1} )
}%end signature
}%end item
\end{itemize}
}}
}
}
\startsection{Class}{CircularUnit}{l5}{%
{\small Базовый класс для определения круглых объектов. Содержит также все свойства юнита.}
\vskip .1in 
\startsubsubsection{Declaration}{
\fbox{\vbox{
\hbox{\vbox{\small public abstract 
class 
CircularUnit}}
\noindent\hbox{\vbox{{\bf extends} Unit}}
}}}

% Removed by DocsPostProcessor:
% \startsubsubsection{Constructors}{
% \vskip -2em
% \begin{itemize}
% \item{\vskip -1.9ex 
% \membername{CircularUnit}
% {\tt protected {\bf CircularUnit}( {\tt long } {\bf id},
% {\tt double } {\bf x},
% {\tt double } {\bf y},
% {\tt double } {\bf speedX},
% {\tt double } {\bf speedY},
% {\tt double } {\bf angle},
% {\tt Faction } {\bf faction},
% {\tt double } {\bf radius} )
% \label{l43}\label{l44}}%end signature
% }%end item
% \end{itemize}
% }
% \\ Removed by DocsPostProcessor.

\startsubsubsection{Methods}{
\vskip -2em
\begin{itemize}
\item{\vskip -1.9ex 
\membername{getRadius}
{\tt public double {\bf getRadius}(  )
\label{l45}\label{l46}}%end signature
\begin{itemize}
\sld
\item{{\bf Returns} - 
Возвращает радиус объекта. 
}%end item
\end{itemize}
}%end item
\end{itemize}
}
\hide{inherited}{
\startsubsubsection{Methods inherited from class {\tt Unit}}{
\par{\small 
\refdefined{l21}\vskip -2em
\begin{itemize}
\item{\vskip -1.9ex 
\membername{getAngle}
{\tt public final double {\bf getAngle}(  )
}%end signature
\begin{itemize}
\sld
\item{{\bf Returns} - 
Возвращает угол поворота объекта в радианах. Нулевой угол соответствует направлению оси абсцисс.
 Положительные значения соответствуют повороту по часовой стрелке. 
}%end item
\end{itemize}
}%end item
\divideents{getAngleTo}
\item{\vskip -1.9ex 
\membername{getAngleTo}
{\tt public double {\bf getAngleTo}( {\tt double } {\bf x},
{\tt double } {\bf y} )
}%end signature
\begin{itemize}
\sld
\item{
\sld
{\bf Parameters}
\sld\isep
  \begin{itemize}
\sld\isep
   \item{
\sld
{\tt x} - X-координата точки.}
   \item{
\sld
{\tt y} - Y-координата точки.}
  \end{itemize}
}%end item
\item{{\bf Returns} - 
Возвращает ориентированный угол [{\tt -PI}, {\tt PI}] между направлением
 данного объекта и вектором из центра данного объекта к указанной точке. 
}%end item
\end{itemize}
}%end item
\divideents{getAngleTo}
\item{\vskip -1.9ex 
\membername{getAngleTo}
{\tt public double {\bf getAngleTo}( {\tt Unit } {\bf unit} )
}%end signature
\begin{itemize}
\sld
\item{
\sld
{\bf Parameters}
\sld\isep
  \begin{itemize}
\sld\isep
   \item{
\sld
{\tt unit} - Объект, к центру которого необходимо определить угол.}
  \end{itemize}
}%end item
\item{{\bf Returns} - 
Возвращает ориентированный угол [{\tt -PI}, {\tt PI}] между направлением
 данного объекта и вектором из центра данного объекта к центру указанного объекта. 
}%end item
\end{itemize}
}%end item
\divideents{getDistanceTo}
\item{\vskip -1.9ex 
\membername{getDistanceTo}
{\tt public double {\bf getDistanceTo}( {\tt double } {\bf x},
{\tt double } {\bf y} )
}%end signature
\begin{itemize}
\sld
\item{
\sld
{\bf Parameters}
\sld\isep
  \begin{itemize}
\sld\isep
   \item{
\sld
{\tt x} - X-координата точки.}
   \item{
\sld
{\tt y} - Y-координата точки.}
  \end{itemize}
}%end item
\item{{\bf Returns} - 
Возвращает расстояние до точки от центра данного объекта. 
}%end item
\end{itemize}
}%end item
\divideents{getDistanceTo}
\item{\vskip -1.9ex 
\membername{getDistanceTo}
{\tt public double {\bf getDistanceTo}( {\tt Unit } {\bf unit} )
}%end signature
\begin{itemize}
\sld
\item{
\sld
{\bf Parameters}
\sld\isep
  \begin{itemize}
\sld\isep
   \item{
\sld
{\tt unit} - Объект, до центра которого необходимо определить расстояние.}
  \end{itemize}
}%end item
\item{{\bf Returns} - 
Возвращает расстояние от центра данного объекта до центра указанного объекта. 
}%end item
\end{itemize}
}%end item
\divideents{getFaction}
\item{\vskip -1.9ex 
\membername{getFaction}
{\tt public Faction {\bf getFaction}(  )
}%end signature
\begin{itemize}
\sld
\item{{\bf Returns} - 
Возвращает фракцию, к которой принадлежит данный юнит. 
}%end item
\end{itemize}
}%end item
\divideents{getId}
\item{\vskip -1.9ex 
\membername{getId}
{\tt public long {\bf getId}(  )
}%end signature
\begin{itemize}
\sld
\item{{\bf Returns} - 
Возвращает уникальный идентификатор объекта. 
}%end item
\end{itemize}
}%end item
\divideents{getSpeedX}
\item{\vskip -1.9ex 
\membername{getSpeedX}
{\tt public final double {\bf getSpeedX}(  )
}%end signature
\begin{itemize}
\sld
\item{{\bf Returns} - 
Возвращает X-составляющую скорости объекта. Ось абсцисс направлена слева направо.
 \bl 
 Для юнитов, способных мгновенно менять свою скорость, возвращается значение перемещения за последний тик. 
}%end item
\end{itemize}
}%end item
\divideents{getSpeedY}
\item{\vskip -1.9ex 
\membername{getSpeedY}
{\tt public final double {\bf getSpeedY}(  )
}%end signature
\begin{itemize}
\sld
\item{{\bf Returns} - 
Возвращает Y-составляющую скорости объекта. Ось ординат направлена сверху вниз.
 \bl 
 Для юнитов, способных мгновенно менять свою скорость, возвращается значение перемещения за последний тик. 
}%end item
\end{itemize}
}%end item
\divideents{getX}
\item{\vskip -1.9ex 
\membername{getX}
{\tt public final double {\bf getX}(  )
}%end signature
\begin{itemize}
\sld
\item{{\bf Returns} - 
Возвращает X-координату центра объекта. Ось абсцисс направлена слева направо. 
}%end item
\end{itemize}
}%end item
\divideents{getY}
\item{\vskip -1.9ex 
\membername{getY}
{\tt public final double {\bf getY}(  )
}%end signature
\begin{itemize}
\sld
\item{{\bf Returns} - 
Возвращает Y-координату центра объекта. Ось ординат направлена сверху вниз. 
}%end item
\end{itemize}
}%end item
\end{itemize}
}}
}
}
\startsection{Class}{Faction}{l6}{%
{\small Фракция юнита.}
\vskip .1in 
\startsubsubsection{Declaration}{
\fbox{\vbox{
\hbox{\vbox{\small public final 
class 
Faction}}
\noindent\hbox{\vbox{{\bf extends} Enum}}
}}}
\startsubsubsection{Fields}{
\begin{itemize}
\item{
public static final Faction ACADEMY\begin{itemize}\item{\vskip -.9ex Волшебники, последователи и охранные сооружения Академии.}\end{itemize}
}
\item{
public static final Faction RENEGADES\begin{itemize}\item{\vskip -.9ex Волшебники, последователи и охранные сооружения Отступников.}\end{itemize}
}
\item{
public static final Faction NEUTRAL\begin{itemize}\item{\vskip -.9ex Нейтральные юниты. Не нападют первыми, но при получении урона будут обороняться.}\end{itemize}
}
\item{
public static final Faction OTHER\begin{itemize}\item{\vskip -.9ex Все остальные юниты в игре.}\end{itemize}
}
\end{itemize}
}
\hide{inherited}{
\startsubsubsection{Methods inherited from class {\tt Enum}}{
\par{\small 
\refdefined{l24}\vskip -2em
\begin{itemize}
\item{\vskip -1.9ex 
\membername{clone}
{\tt protected final Object {\bf clone}(  )
}%end signature
}%end item
\divideents{compareTo}
\item{\vskip -1.9ex 
\membername{compareTo}
{\tt public final int {\bf compareTo}( {\tt Enum } {\bf arg0} )
}%end signature
}%end item
\divideents{equals}
\item{\vskip -1.9ex 
\membername{equals}
{\tt public final boolean {\bf equals}( {\tt Object } {\bf arg0} )
}%end signature
}%end item
\divideents{finalize}
\item{\vskip -1.9ex 
\membername{finalize}
{\tt protected final void {\bf finalize}(  )
}%end signature
}%end item
\divideents{getDeclaringClass}
\item{\vskip -1.9ex 
\membername{getDeclaringClass}
{\tt public final Class {\bf getDeclaringClass}(  )
}%end signature
}%end item
\divideents{hashCode}
\item{\vskip -1.9ex 
\membername{hashCode}
{\tt public final int {\bf hashCode}(  )
}%end signature
}%end item
\divideents{name}
\item{\vskip -1.9ex 
\membername{name}
{\tt public final String {\bf name}(  )
}%end signature
}%end item
\divideents{ordinal}
\item{\vskip -1.9ex 
\membername{ordinal}
{\tt public final int {\bf ordinal}(  )
}%end signature
}%end item
\divideents{toString}
\item{\vskip -1.9ex 
\membername{toString}
{\tt public String {\bf toString}(  )
}%end signature
}%end item
\divideents{valueOf}
\item{\vskip -1.9ex 
\membername{valueOf}
{\tt public static Enum {\bf valueOf}( {\tt Class } {\bf arg0},
{\tt String } {\bf arg1} )
}%end signature
}%end item
\end{itemize}
}}
}
}
\startsection{Class}{Game}{l7}{%
{\small Предоставляет доступ к различным игровым константам.}
\vskip .1in 
\startsubsubsection{Declaration}{
\fbox{\vbox{
\hbox{\vbox{\small public 
class 
Game}}
\noindent\hbox{\vbox{{\bf extends} Object}}
}}}

% Removed by DocsPostProcessor:
% \startsubsubsection{Constructors}{
% \vskip -2em
% \begin{itemize}
% \item{\vskip -1.9ex 
% \membername{Game}
% {\tt public {\bf Game}( {\tt long } {\bf randomSeed},
% {\tt int } {\bf tickCount},
% {\tt double } {\bf mapSize},
% {\tt boolean } {\bf skillsEnabled},
% {\tt boolean } {\bf rawMessagesEnabled},
% {\tt double } {\bf friendlyFireDamageFactor},
% {\tt double } {\bf buildingDamageScoreFactor},
% {\tt double } {\bf buildingEliminationScoreFactor},
% {\tt double } {\bf minionDamageScoreFactor},
% {\tt double } {\bf minionEliminationScoreFactor},
% {\tt double } {\bf wizardDamageScoreFactor},
% {\tt double } {\bf wizardEliminationScoreFactor},
% {\tt double } {\bf teamWorkingScoreFactor},
% {\tt int } {\bf victoryScore},
% {\tt double } {\bf scoreGainRange},
% {\tt int } {\bf rawMessageMaxLength},
% {\tt double } {\bf rawMessageTransmissionSpeed},
% {\tt double } {\bf wizardRadius},
% {\tt double } {\bf wizardCastRange},
% {\tt double } {\bf wizardVisionRange},
% {\tt double } {\bf wizardForwardSpeed},
% {\tt double } {\bf wizardBackwardSpeed},
% {\tt double } {\bf wizardStrafeSpeed},
% {\tt int } {\bf wizardBaseLife},
% {\tt int } {\bf wizardLifeGrowthPerLevel},
% {\tt int } {\bf wizardBaseMana},
% {\tt int } {\bf wizardManaGrowthPerLevel},
% {\tt double } {\bf wizardBaseLifeRegeneration},
% {\tt double } {\bf wizardLifeRegenerationGrowthPerLevel},
% {\tt double } {\bf wizardBaseManaRegeneration},
% {\tt double } {\bf wizardManaRegenerationGrowthPerLevel},
% {\tt double } {\bf wizardMaxTurnAngle},
% {\tt int } {\bf wizardMaxResurrectionDelayTicks},
% {\tt int } {\bf wizardMinResurrectionDelayTicks},
% {\tt int } {\bf wizardActionCooldownTicks},
% {\tt int } {\bf staffCooldownTicks},
% {\tt int } {\bf magicMissileCooldownTicks},
% {\tt int } {\bf frostBoltCooldownTicks},
% {\tt int } {\bf fireballCooldownTicks},
% {\tt int } {\bf hasteCooldownTicks},
% {\tt int } {\bf shieldCooldownTicks},
% {\tt int } {\bf magicMissileManacost},
% {\tt int } {\bf frostBoltManacost},
% {\tt int } {\bf fireballManacost},
% {\tt int } {\bf hasteManacost},
% {\tt int } {\bf shieldManacost},
% {\tt int } {\bf staffDamage},
% {\tt double } {\bf staffSector},
% {\tt double } {\bf staffRange},
% {\tt int[]} {\bf levelUpXpValues},
% {\tt double } {\bf minionRadius},
% {\tt double } {\bf minionVisionRange},
% {\tt double } {\bf minionSpeed},
% {\tt double } {\bf minionMaxTurnAngle},
% {\tt int } {\bf minionLife},
% {\tt int } {\bf factionMinionAppearanceIntervalTicks},
% {\tt int } {\bf orcWoodcutterActionCooldownTicks},
% {\tt int } {\bf orcWoodcutterDamage},
% {\tt double } {\bf orcWoodcutterAttackSector},
% {\tt double } {\bf orcWoodcutterAttackRange},
% {\tt int } {\bf fetishBlowdartActionCooldownTicks},
% {\tt double } {\bf fetishBlowdartAttackRange},
% {\tt double } {\bf fetishBlowdartAttackSector},
% {\tt double } {\bf bonusRadius},
% {\tt int } {\bf bonusAppearanceIntervalTicks},
% {\tt int } {\bf bonusScoreAmount},
% {\tt double } {\bf dartRadius},
% {\tt double } {\bf dartSpeed},
% {\tt int } {\bf dartDirectDamage},
% {\tt double } {\bf magicMissileRadius},
% {\tt double } {\bf magicMissileSpeed},
% {\tt int } {\bf magicMissileDirectDamage},
% {\tt double } {\bf frostBoltRadius},
% {\tt double } {\bf frostBoltSpeed},
% {\tt int } {\bf frostBoltDirectDamage},
% {\tt double } {\bf fireballRadius},
% {\tt double } {\bf fireballSpeed},
% {\tt double } {\bf fireballExplosionMaxDamageRange},
% {\tt double } {\bf fireballExplosionMinDamageRange},
% {\tt int } {\bf fireballExplosionMaxDamage},
% {\tt int } {\bf fireballExplosionMinDamage},
% {\tt double } {\bf guardianTowerRadius},
% {\tt double } {\bf guardianTowerVisionRange},
% {\tt double } {\bf guardianTowerLife},
% {\tt double } {\bf guardianTowerAttackRange},
% {\tt int } {\bf guardianTowerDamage},
% {\tt int } {\bf guardianTowerCooldownTicks},
% {\tt double } {\bf factionBaseRadius},
% {\tt double } {\bf factionBaseVisionRange},
% {\tt double } {\bf factionBaseLife},
% {\tt double } {\bf factionBaseAttackRange},
% {\tt int } {\bf factionBaseDamage},
% {\tt int } {\bf factionBaseCooldownTicks},
% {\tt int } {\bf burningDurationTicks},
% {\tt int } {\bf burningSummaryDamage},
% {\tt int } {\bf empoweredDurationTicks},
% {\tt double } {\bf empoweredDamageFactor},
% {\tt int } {\bf frozenDurationTicks},
% {\tt int } {\bf hastenedDurationTicks},
% {\tt double } {\bf hastenedBonusDurationFactor},
% {\tt double } {\bf hastenedMovementBonusFactor},
% {\tt double } {\bf hastenedRotationBonusFactor},
% {\tt int } {\bf shieldedDurationTicks},
% {\tt double } {\bf shieldedBonusDurationFactor},
% {\tt double } {\bf shieldedDirectDamageAbsorptionFactor},
% {\tt double } {\bf auraSkillRange},
% {\tt double } {\bf rangeBonusPerSkillLevel},
% {\tt int } {\bf magicalDamageBonusPerSkillLevel},
% {\tt int } {\bf staffDamageBonusPerSkillLevel},
% {\tt double } {\bf movementBonusFactorPerSkillLevel},
% {\tt int } {\bf magicalDamageAbsorptionPerSkillLevel} )
% \label{l47}\label{l48}}%end signature
% }%end item
% \end{itemize}
% }
% \\ Removed by DocsPostProcessor.

\startsubsubsection{Methods}{
\vskip -2em
\begin{itemize}
\item{\vskip -1.9ex 
\membername{getAuraSkillRange}
{\tt public double {\bf getAuraSkillRange}(  )
\label{l49}\label{l50}}%end signature
\begin{itemize}
\sld
\item{{\bf Returns} - 
Возвращает дальность действия аур. 
}%end item
\end{itemize}
}%end item
\divideents{getBonusAppearanceIntervalTicks}
\item{\vskip -1.9ex 
\membername{getBonusAppearanceIntervalTicks}
{\tt public int {\bf getBonusAppearanceIntervalTicks}(  )
\label{l51}\label{l52}}%end signature
\begin{itemize}
\sld
\item{{\bf Returns} - 
Возвращает интервал появления бонусов.
 \bl 
 Каждый раз по прошествии указанного интервала симулятор игры создаёт до двух бонусов в точках
 ({\tt mapSize * 0.3}, {\tt mapSize * 0.3}) и ({\tt mapSize * 0.7}, {\tt mapSize * 0.7}). Если любая часть
 области появления бонуса уже занята волшебником или другим бонусом, то создание бонуса будет отложено до
 окончания очередного интервала. 
}%end item
\end{itemize}
}%end item
\divideents{getBonusRadius}
\item{\vskip -1.9ex 
\membername{getBonusRadius}
{\tt public double {\bf getBonusRadius}(  )
\label{l53}\label{l54}}%end signature
\begin{itemize}
\sld
\item{{\bf Returns} - 
Возвращает радиус бонуса. 
}%end item
\end{itemize}
}%end item
\divideents{getBonusScoreAmount}
\item{\vskip -1.9ex 
\membername{getBonusScoreAmount}
{\tt public int {\bf getBonusScoreAmount}(  )
\label{l55}\label{l56}}%end signature
\begin{itemize}
\sld
\item{{\bf Returns} - 
Возвращает количество баллов, начисляемых игроку, волшебник которого подробрал бонус.
 \bl 
 Сам волшебник получает такое же количество опыта. 
}%end item
\end{itemize}
}%end item
\divideents{getBuildingDamageScoreFactor}
\item{\vskip -1.9ex 
\membername{getBuildingDamageScoreFactor}
{\tt public double {\bf getBuildingDamageScoreFactor}(  )
\label{l57}\label{l58}}%end signature
\begin{itemize}
\sld
\item{{\bf Returns} - 
Возвращает коэффициент опыта, получаемого волшебником при нанесении урона строениям противоположной
 фракции. 
}%end item
\end{itemize}
}%end item
\divideents{getBuildingEliminationScoreFactor}
\item{\vskip -1.9ex 
\membername{getBuildingEliminationScoreFactor}
{\tt public double {\bf getBuildingEliminationScoreFactor}(  )
\label{l59}\label{l60}}%end signature
\begin{itemize}
\sld
\item{{\bf Returns} - 
Возвращает коэффициент опыта, получаемого волшебником за разрушение строения противоположной фракции.
 \bl 
 Применяется к максимальному количеству жизненной энергии строения. 
}%end item
\end{itemize}
}%end item
\divideents{getBurningDurationTicks}
\item{\vskip -1.9ex 
\membername{getBurningDurationTicks}
{\tt public int {\bf getBurningDurationTicks}(  )
\label{l61}\label{l62}}%end signature
\begin{itemize}
\sld
\item{{\bf Returns} - 
Возвращает длительность действия статуса {\tt BURNING}. 
}%end item
\end{itemize}
}%end item
\divideents{getBurningSummaryDamage}
\item{\vskip -1.9ex 
\membername{getBurningSummaryDamage}
{\tt public int {\bf getBurningSummaryDamage}(  )
\label{l63}\label{l64}}%end signature
\begin{itemize}
\sld
\item{{\bf Returns} - 
Возвращает суммарный урон, получаемый живым юнитом за время действия статуса {\tt BURNING}. 
}%end item
\end{itemize}
}%end item
\divideents{getDartDirectDamage}
\item{\vskip -1.9ex 
\membername{getDartDirectDamage}
{\tt public int {\bf getDartDirectDamage}(  )
\label{l65}\label{l66}}%end signature
\begin{itemize}
\sld
\item{{\bf Returns} - 
Возвращает урон дротика. 
}%end item
\end{itemize}
}%end item
\divideents{getDartRadius}
\item{\vskip -1.9ex 
\membername{getDartRadius}
{\tt public double {\bf getDartRadius}(  )
\label{l67}\label{l68}}%end signature
\begin{itemize}
\sld
\item{{\bf Returns} - 
Возвращает радиус дротика. 
}%end item
\end{itemize}
}%end item
\divideents{getDartSpeed}
\item{\vskip -1.9ex 
\membername{getDartSpeed}
{\tt public double {\bf getDartSpeed}(  )
\label{l69}\label{l70}}%end signature
\begin{itemize}
\sld
\item{{\bf Returns} - 
Возвращает скорость полёта дротика. 
}%end item
\end{itemize}
}%end item
\divideents{getEmpoweredDamageFactor}
\item{\vskip -1.9ex 
\membername{getEmpoweredDamageFactor}
{\tt public double {\bf getEmpoweredDamageFactor}(  )
\label{l71}\label{l72}}%end signature
\begin{itemize}
\sld
\item{{\bf Returns} - 
Возвращает мультипликатор урона, наносимого живым юнитом под действием статуса {\tt EMPOWERED}.
 \bl 
 Мультипликатор применяется к ударам в ближнем бою, прямым попаданиям снарядов, а также взрыву <<Огненного шара>>,
 но не применяется к урону, получаемому от статусов. 
}%end item
\end{itemize}
}%end item
\divideents{getEmpoweredDurationTicks}
\item{\vskip -1.9ex 
\membername{getEmpoweredDurationTicks}
{\tt public int {\bf getEmpoweredDurationTicks}(  )
\label{l73}\label{l74}}%end signature
\begin{itemize}
\sld
\item{{\bf Returns} - 
Возвращает длительность действия статуса {\tt EMPOWERED}. 
}%end item
\end{itemize}
}%end item
\divideents{getFactionBaseAttackRange}
\item{\vskip -1.9ex 
\membername{getFactionBaseAttackRange}
{\tt public double {\bf getFactionBaseAttackRange}(  )
\label{l75}\label{l76}}%end signature
\begin{itemize}
\sld
\item{{\bf Returns} - 
Возвращает максимальное расстояние (от центра до центра), на котором база фракции может атаковать
 другие объекты. 
}%end item
\end{itemize}
}%end item
\divideents{getFactionBaseCooldownTicks}
\item{\vskip -1.9ex 
\membername{getFactionBaseCooldownTicks}
{\tt public int {\bf getFactionBaseCooldownTicks}(  )
\label{l77}\label{l78}}%end signature
\begin{itemize}
\sld
\item{{\bf Returns} - 
Возвращает минимально возможную задержку между двумя последовательными атаками базы фракции. 
}%end item
\end{itemize}
}%end item
\divideents{getFactionBaseDamage}
\item{\vskip -1.9ex 
\membername{getFactionBaseDamage}
{\tt public int {\bf getFactionBaseDamage}(  )
\label{l79}\label{l80}}%end signature
\begin{itemize}
\sld
\item{{\bf Returns} - 
Возвращает урон одной атаки базы фракции. 
}%end item
\end{itemize}
}%end item
\divideents{getFactionBaseLife}
\item{\vskip -1.9ex 
\membername{getFactionBaseLife}
{\tt public double {\bf getFactionBaseLife}(  )
\label{l81}\label{l82}}%end signature
\begin{itemize}
\sld
\item{{\bf Returns} - 
Возвращает начальное значение жизненной энергии базы фракции. 
}%end item
\end{itemize}
}%end item
\divideents{getFactionBaseRadius}
\item{\vskip -1.9ex 
\membername{getFactionBaseRadius}
{\tt public double {\bf getFactionBaseRadius}(  )
\label{l83}\label{l84}}%end signature
\begin{itemize}
\sld
\item{{\bf Returns} - 
Возвращает радиус базы фракции. 
}%end item
\end{itemize}
}%end item
\divideents{getFactionBaseVisionRange}
\item{\vskip -1.9ex 
\membername{getFactionBaseVisionRange}
{\tt public double {\bf getFactionBaseVisionRange}(  )
\label{l85}\label{l86}}%end signature
\begin{itemize}
\sld
\item{{\bf Returns} - 
Возвращает максимальное расстояние (от центра до центра), на котором база фракции обнаруживает другие
 объекты. 
}%end item
\end{itemize}
}%end item
\divideents{getFactionMinionAppearanceIntervalTicks}
\item{\vskip -1.9ex 
\membername{getFactionMinionAppearanceIntervalTicks}
{\tt public int {\bf getFactionMinionAppearanceIntervalTicks}(  )
\label{l87}\label{l88}}%end signature
\begin{itemize}
\sld
\item{{\bf Returns} - 
Возвращает интервал, с которым появляются миньоны двух противостоящих фракций ({\tt ACADEMY} и
 {\tt RENEGADES}).
 \bl 
 Миньоны каждой из этих фракций появляются тремя группами (по одной на дорожку) недалеко от своей базы. Группа
 состоит и трёх орков и одного фетиша. Сразу после появления миньоны начинают продвижение по своей дорожке в
 сторону базы противоположной фракции, при этом атакуя всех противников на своём пути. 
}%end item
\end{itemize}
}%end item
\divideents{getFetishBlowdartActionCooldownTicks}
\item{\vskip -1.9ex 
\membername{getFetishBlowdartActionCooldownTicks}
{\tt public int {\bf getFetishBlowdartActionCooldownTicks}(  )
\label{l89}\label{l90}}%end signature
\begin{itemize}
\sld
\item{{\bf Returns} - 
Возвращает минимально возможную задержку между двумя последовательными атаками фетиша. 
}%end item
\end{itemize}
}%end item
\divideents{getFetishBlowdartAttackRange}
\item{\vskip -1.9ex 
\membername{getFetishBlowdartAttackRange}
{\tt public double {\bf getFetishBlowdartAttackRange}(  )
\label{l91}\label{l92}}%end signature
\begin{itemize}
\sld
\item{{\bf Returns} - 
Возвращает дальность полёта дротика, выпущенного фетишем. 
}%end item
\end{itemize}
}%end item
\divideents{getFetishBlowdartAttackSector}
\item{\vskip -1.9ex 
\membername{getFetishBlowdartAttackSector}
{\tt public double {\bf getFetishBlowdartAttackSector}(  )
\label{l93}\label{l94}}%end signature
\begin{itemize}
\sld
\item{{\bf Returns} - 
Возвращает сектор метания дротика фетишем.
 \bl 
 Угол полёта дротика относительно направления фетиша ограничен интервалом от
 {\tt -fetishBlowdartAttackSector $/$ 2.0} до {\tt fetishBlowdartAttackSector $/$ 2.0}. 
}%end item
\end{itemize}
}%end item
\divideents{getFireballCooldownTicks}
\item{\vskip -1.9ex 
\membername{getFireballCooldownTicks}
{\tt public int {\bf getFireballCooldownTicks}(  )
\label{l95}\label{l96}}%end signature
\begin{itemize}
\sld
\item{{\bf Returns} - 
Возвращает минимально возможную задержку между двумя последовательными заклинаниями <<Огненный шар>>. 
}%end item
\end{itemize}
}%end item
\divideents{getFireballExplosionMaxDamage}
\item{\vskip -1.9ex 
\membername{getFireballExplosionMaxDamage}
{\tt public int {\bf getFireballExplosionMaxDamage}(  )
\label{l97}\label{l98}}%end signature
\begin{itemize}
\sld
\item{{\bf Returns} - 
Возвращает урон <<Огненного шара>> в эпицентре взрыва.
 \bl 
 Живой юнит получает {\tt fireballExplosionMaxDamage} единиц урона, если расстояние от центра взрыва до
 ближайшей точки этого юнита не превышает {\tt fireballExplosionMaxDamageRange}. По мере увеличения расстояния
 до {\tt fireballExplosionMinDamageRange}, урон <<Огненного шара>> равномерно снижается и достигает
 {\tt fireballExplosionMinDamage}. Если расстояние от центра взрыва до ближайшей точки живого юнита превышает
 {\tt fireballExplosionMinDamageRange}, то урон ему не наносится.
 \bl 
 Если живой юнит получил какой-либо урон от взрыва <<Огненного шара>>, то он загорается ({\tt BURNING}). 
}%end item
\end{itemize}
}%end item
\divideents{getFireballExplosionMaxDamageRange}
\item{\vskip -1.9ex 
\membername{getFireballExplosionMaxDamageRange}
{\tt public double {\bf getFireballExplosionMaxDamageRange}(  )
\label{l99}\label{l100}}%end signature
\begin{itemize}
\sld
\item{{\bf Returns} - 
Возвращает радиус области, в которой живые юниты получают максимальный урон от взрыва <<Огненного шара>>. 
}%end item
\item{{\bf See Also}
  \begin{itemize}
   \item{{\tt Game.getFireballExplosionMaxDamage()} {\small 
\refdefined{l101}}%end \small
}%end item
  \end{itemize}
}%end item
\end{itemize}
}%end item
\divideents{getFireballExplosionMinDamage}
\item{\vskip -1.9ex 
\membername{getFireballExplosionMinDamage}
{\tt public int {\bf getFireballExplosionMinDamage}(  )
\label{l102}\label{l103}}%end signature
\begin{itemize}
\sld
\item{{\bf Returns} - 
Возвращает урон <<Огненного шара>> на периферии взрыва. 
}%end item
\item{{\bf See Also}
  \begin{itemize}
   \item{{\tt Game.getFireballExplosionMaxDamage()} {\small 
\refdefined{l101}}%end \small
}%end item
  \end{itemize}
}%end item
\end{itemize}
}%end item
\divideents{getFireballExplosionMinDamageRange}
\item{\vskip -1.9ex 
\membername{getFireballExplosionMinDamageRange}
{\tt public double {\bf getFireballExplosionMinDamageRange}(  )
\label{l104}\label{l105}}%end signature
\begin{itemize}
\sld
\item{{\bf Returns} - 
Возвращает радиус области, в которой живые юниты получают какой-либо урон от взрыва <<Огненного шара>>. 
}%end item
\item{{\bf See Also}
  \begin{itemize}
   \item{{\tt Game.getFireballExplosionMaxDamage()} {\small 
\refdefined{l101}}%end \small
}%end item
  \end{itemize}
}%end item
\end{itemize}
}%end item
\divideents{getFireballManacost}
\item{\vskip -1.9ex 
\membername{getFireballManacost}
{\tt public int {\bf getFireballManacost}(  )
\label{l106}\label{l107}}%end signature
\begin{itemize}
\sld
\item{{\bf Returns} - 
Возвращает количество магической энергии, требуемой для заклинания <<Огненный шар>>. 
}%end item
\end{itemize}
}%end item
\divideents{getFireballRadius}
\item{\vskip -1.9ex 
\membername{getFireballRadius}
{\tt public double {\bf getFireballRadius}(  )
\label{l108}\label{l109}}%end signature
\begin{itemize}
\sld
\item{{\bf Returns} - 
Возвращает радиус <<Огненного шара>>. 
}%end item
\end{itemize}
}%end item
\divideents{getFireballSpeed}
\item{\vskip -1.9ex 
\membername{getFireballSpeed}
{\tt public double {\bf getFireballSpeed}(  )
\label{l110}\label{l111}}%end signature
\begin{itemize}
\sld
\item{{\bf Returns} - 
Возвращает скорость полёта <<Огненного шара>>. 
}%end item
\end{itemize}
}%end item
\divideents{getFriendlyFireDamageFactor}
\item{\vskip -1.9ex 
\membername{getFriendlyFireDamageFactor}
{\tt public double {\bf getFriendlyFireDamageFactor}(  )
\label{l112}\label{l113}}%end signature
\begin{itemize}
\sld
\item{{\bf Returns} - 
Возвращает коэффициент урона, наносимого волшебниками одной фракции друг другу в результате
 дружественного огня.
 \bl 
 Значение зависит от режима игры, но не может выходить за границы интервала от {\tt 0.0} до {\tt 1.0}.
 \bl 
 Вне зависимости от режима игры, волшебники не могут наносить урон союзным миньонам и структурам. 
}%end item
\end{itemize}
}%end item
\divideents{getFrostBoltCooldownTicks}
\item{\vskip -1.9ex 
\membername{getFrostBoltCooldownTicks}
{\tt public int {\bf getFrostBoltCooldownTicks}(  )
\label{l114}\label{l115}}%end signature
\begin{itemize}
\sld
\item{{\bf Returns} - 
Возвращает минимально возможную задержку между двумя последовательными заклинаниями <<Ледяная стрела>>. 
}%end item
\end{itemize}
}%end item
\divideents{getFrostBoltDirectDamage}
\item{\vskip -1.9ex 
\membername{getFrostBoltDirectDamage}
{\tt public int {\bf getFrostBoltDirectDamage}(  )
\label{l116}\label{l117}}%end signature
\begin{itemize}
\sld
\item{{\bf Returns} - 
Возвращает урон <<Ледяной стрелы>>. 
}%end item
\end{itemize}
}%end item
\divideents{getFrostBoltManacost}
\item{\vskip -1.9ex 
\membername{getFrostBoltManacost}
{\tt public int {\bf getFrostBoltManacost}(  )
\label{l118}\label{l119}}%end signature
\begin{itemize}
\sld
\item{{\bf Returns} - 
Возвращает количество магической энергии, требуемой для заклинания <<Ледяная стрела>>. 
}%end item
\end{itemize}
}%end item
\divideents{getFrostBoltRadius}
\item{\vskip -1.9ex 
\membername{getFrostBoltRadius}
{\tt public double {\bf getFrostBoltRadius}(  )
\label{l120}\label{l121}}%end signature
\begin{itemize}
\sld
\item{{\bf Returns} - 
Возвращает радиус <<Ледяной стрелы>>. 
}%end item
\end{itemize}
}%end item
\divideents{getFrostBoltSpeed}
\item{\vskip -1.9ex 
\membername{getFrostBoltSpeed}
{\tt public double {\bf getFrostBoltSpeed}(  )
\label{l122}\label{l123}}%end signature
\begin{itemize}
\sld
\item{{\bf Returns} - 
Возвращает скорость полёта <<Ледяной стрелы>>. 
}%end item
\end{itemize}
}%end item
\divideents{getFrozenDurationTicks}
\item{\vskip -1.9ex 
\membername{getFrozenDurationTicks}
{\tt public int {\bf getFrozenDurationTicks}(  )
\label{l124}\label{l125}}%end signature
\begin{itemize}
\sld
\item{{\bf Returns} - 
Возвращает длительность действия статуса {\tt FROZEN}. 
}%end item
\end{itemize}
}%end item
\divideents{getGuardianTowerAttackRange}
\item{\vskip -1.9ex 
\membername{getGuardianTowerAttackRange}
{\tt public double {\bf getGuardianTowerAttackRange}(  )
\label{l126}\label{l127}}%end signature
\begin{itemize}
\sld
\item{{\bf Returns} - 
Возвращает максимальное расстояние (от центра до центра), на котором охранная башня может атаковать
 другие объекты. 
}%end item
\end{itemize}
}%end item
\divideents{getGuardianTowerCooldownTicks}
\item{\vskip -1.9ex 
\membername{getGuardianTowerCooldownTicks}
{\tt public int {\bf getGuardianTowerCooldownTicks}(  )
\label{l128}\label{l129}}%end signature
\begin{itemize}
\sld
\item{{\bf Returns} - 
Возвращает минимально возможную задержку между двумя последовательными атаками охранной башни. 
}%end item
\end{itemize}
}%end item
\divideents{getGuardianTowerDamage}
\item{\vskip -1.9ex 
\membername{getGuardianTowerDamage}
{\tt public int {\bf getGuardianTowerDamage}(  )
\label{l130}\label{l131}}%end signature
\begin{itemize}
\sld
\item{{\bf Returns} - 
Возвращает урон одной атаки охранной башни. 
}%end item
\end{itemize}
}%end item
\divideents{getGuardianTowerLife}
\item{\vskip -1.9ex 
\membername{getGuardianTowerLife}
{\tt public double {\bf getGuardianTowerLife}(  )
\label{l132}\label{l133}}%end signature
\begin{itemize}
\sld
\item{{\bf Returns} - 
Возвращает начальное значение жизненной энергии охранной башни. 
}%end item
\end{itemize}
}%end item
\divideents{getGuardianTowerRadius}
\item{\vskip -1.9ex 
\membername{getGuardianTowerRadius}
{\tt public double {\bf getGuardianTowerRadius}(  )
\label{l134}\label{l135}}%end signature
\begin{itemize}
\sld
\item{{\bf Returns} - 
Возвращает радиус охранной башни. 
}%end item
\end{itemize}
}%end item
\divideents{getGuardianTowerVisionRange}
\item{\vskip -1.9ex 
\membername{getGuardianTowerVisionRange}
{\tt public double {\bf getGuardianTowerVisionRange}(  )
\label{l136}\label{l137}}%end signature
\begin{itemize}
\sld
\item{{\bf Returns} - 
Возвращает максимальное расстояние (от центра до центра), на котором охранная башня обнаруживает другие
 объекты. 
}%end item
\end{itemize}
}%end item
\divideents{getHasteCooldownTicks}
\item{\vskip -1.9ex 
\membername{getHasteCooldownTicks}
{\tt public int {\bf getHasteCooldownTicks}(  )
\label{l138}\label{l139}}%end signature
\begin{itemize}
\sld
\item{{\bf Returns} - 
Возвращает минимально возможную задержку между двумя последовательными заклинаниями <<Ускорение>>. 
}%end item
\end{itemize}
}%end item
\divideents{getHasteManacost}
\item{\vskip -1.9ex 
\membername{getHasteManacost}
{\tt public int {\bf getHasteManacost}(  )
\label{l140}\label{l141}}%end signature
\begin{itemize}
\sld
\item{{\bf Returns} - 
Возвращает количество магической энергии, требуемой для заклинания <<Ускорение>>. 
}%end item
\end{itemize}
}%end item
\divideents{getHastenedBonusDurationFactor}
\item{\vskip -1.9ex 
\membername{getHastenedBonusDurationFactor}
{\tt public double {\bf getHastenedBonusDurationFactor}(  )
\label{l142}\label{l143}}%end signature
\begin{itemize}
\sld
\item{{\bf Returns} - 
Возвращает мультилпикатор длительности действия статуса {\tt HASTENED} в случае подбора бонуса. 
}%end item
\end{itemize}
}%end item
\divideents{getHastenedDurationTicks}
\item{\vskip -1.9ex 
\membername{getHastenedDurationTicks}
{\tt public int {\bf getHastenedDurationTicks}(  )
\label{l144}\label{l145}}%end signature
\begin{itemize}
\sld
\item{{\bf Returns} - 
Возвращает длительность действия статуса {\tt HASTENED}. 
}%end item
\end{itemize}
}%end item
\divideents{getHastenedMovementBonusFactor}
\item{\vskip -1.9ex 
\membername{getHastenedMovementBonusFactor}
{\tt public double {\bf getHastenedMovementBonusFactor}(  )
\label{l146}\label{l147}}%end signature
\begin{itemize}
\sld
\item{{\bf Returns} - 
Возвращает относительное увеличение скорости перемещения в результате дествия статуса {\tt HASTENED}.
 \bl 
 Увеличение скорости от действия статуса {\tt HASTENED} и увеличение скорости в результате изучения умений,
 являющихся пререквизитами умения {\tt HASTE}, являются аддитивными. Таким образом, максимальное значение
 скорости волшебника составляет
 {\tt 1.0 + 4.0 * movementBonusFactorPerSkillLevel + hastenedMovementBonusFactor} от базовой. 
}%end item
\end{itemize}
}%end item
\divideents{getHastenedRotationBonusFactor}
\item{\vskip -1.9ex 
\membername{getHastenedRotationBonusFactor}
{\tt public double {\bf getHastenedRotationBonusFactor}(  )
\label{l148}\label{l149}}%end signature
\begin{itemize}
\sld
\item{{\bf Returns} - 
Возвращает относительное увеличение скорости поворота в результате дествия статуса {\tt HASTENED}. 
}%end item
\end{itemize}
}%end item
\divideents{getLevelUpXpValues}
\item{\vskip -1.9ex 
\membername{getLevelUpXpValues}
{\tt public int[] {\bf getLevelUpXpValues}(  )
\label{l150}\label{l151}}%end signature
\begin{itemize}
\sld
\item{{\bf Returns} - 
Возвращает последовательность неотрицательных целых чисел.
 \bl 
 Количество чисел равно количеству уровней, которые волшебник может получить в данном режиме игры. Значение с
 индексом {\tt N} определяет количество опыта, которое необходимо набрать волшебнику уровня {\tt N} для
 получения следующего уровня. Таким образом, количество опыта, необходимое волшебнику начального уровня для
 получения уровня {\tt N}, равно сумме первых {\tt N} элементов. 
}%end item
\end{itemize}
}%end item
\divideents{getMagicalDamageAbsorptionPerSkillLevel}
\item{\vskip -1.9ex 
\membername{getMagicalDamageAbsorptionPerSkillLevel}
{\tt public int {\bf getMagicalDamageAbsorptionPerSkillLevel}(  )
\label{l152}\label{l153}}%end signature
\begin{itemize}
\sld
\item{{\bf Returns} - 
Возвращает абсолютное уменьшение урона, получаемого волшебником в результате прямых попаданий магических
 снарядов, взрыва <<Огненного шара>> и атак строений, за каждое последовательное изучение умения, являющегося
 одним из пререквизитов умения {\tt SHIELD}. 
}%end item
\end{itemize}
}%end item
\divideents{getMagicalDamageBonusPerSkillLevel}
\item{\vskip -1.9ex 
\membername{getMagicalDamageBonusPerSkillLevel}
{\tt public int {\bf getMagicalDamageBonusPerSkillLevel}(  )
\label{l154}\label{l155}}%end signature
\begin{itemize}
\sld
\item{{\bf Returns} - 
Возвращает абсолютное увеличение урона, наносимого волшебником в результате прямых попаданий магических
 снарядов и взрыва <<Огненного шара>>, за каждое последовательное изучение умения, являющегося одним из
 пререквизитов умения {\tt FROST\_BOLT}. 
}%end item
\end{itemize}
}%end item
\divideents{getMagicMissileCooldownTicks}
\item{\vskip -1.9ex 
\membername{getMagicMissileCooldownTicks}
{\tt public int {\bf getMagicMissileCooldownTicks}(  )
\label{l156}\label{l157}}%end signature
\begin{itemize}
\sld
\item{{\bf Returns} - 
Возвращает минимально возможную задержку между двумя последовательными заклинаниями <<Магическая
 ракета>>. 
}%end item
\end{itemize}
}%end item
\divideents{getMagicMissileDirectDamage}
\item{\vskip -1.9ex 
\membername{getMagicMissileDirectDamage}
{\tt public int {\bf getMagicMissileDirectDamage}(  )
\label{l158}\label{l159}}%end signature
\begin{itemize}
\sld
\item{{\bf Returns} - 
Возвращает урон <<Магической ракеты>>. 
}%end item
\end{itemize}
}%end item
\divideents{getMagicMissileManacost}
\item{\vskip -1.9ex 
\membername{getMagicMissileManacost}
{\tt public int {\bf getMagicMissileManacost}(  )
\label{l160}\label{l161}}%end signature
\begin{itemize}
\sld
\item{{\bf Returns} - 
Возвращает количество магической энергии, требуемой для заклинания <<Магическая ракета>>. 
}%end item
\end{itemize}
}%end item
\divideents{getMagicMissileRadius}
\item{\vskip -1.9ex 
\membername{getMagicMissileRadius}
{\tt public double {\bf getMagicMissileRadius}(  )
\label{l162}\label{l163}}%end signature
\begin{itemize}
\sld
\item{{\bf Returns} - 
Возвращает радиус <<Магической ракеты>>. 
}%end item
\end{itemize}
}%end item
\divideents{getMagicMissileSpeed}
\item{\vskip -1.9ex 
\membername{getMagicMissileSpeed}
{\tt public double {\bf getMagicMissileSpeed}(  )
\label{l164}\label{l165}}%end signature
\begin{itemize}
\sld
\item{{\bf Returns} - 
Возвращает скорость полёта <<Магической ракеты>>. 
}%end item
\end{itemize}
}%end item
\divideents{getMapSize}
\item{\vskip -1.9ex 
\membername{getMapSize}
{\tt public double {\bf getMapSize}(  )
\label{l166}\label{l167}}%end signature
\begin{itemize}
\sld
\item{{\bf Returns} - 
Возвращает размер (ширину и высоту) карты. 
}%end item
\end{itemize}
}%end item
\divideents{getMinionDamageScoreFactor}
\item{\vskip -1.9ex 
\membername{getMinionDamageScoreFactor}
{\tt public double {\bf getMinionDamageScoreFactor}(  )
\label{l168}\label{l169}}%end signature
\begin{itemize}
\sld
\item{{\bf Returns} - 
Возвращает коэффициент опыта, получаемого волшебником при нанесении урона миньонам другой фракции. 
}%end item
\end{itemize}
}%end item
\divideents{getMinionEliminationScoreFactor}
\item{\vskip -1.9ex 
\membername{getMinionEliminationScoreFactor}
{\tt public double {\bf getMinionEliminationScoreFactor}(  )
\label{l170}\label{l171}}%end signature
\begin{itemize}
\sld
\item{{\bf Returns} - 
Возвращает коэффициент опыта, получаемого волшебником за уничтожение миньона другой фракции.
 \bl 
 Применяется к максимальному количеству жизненной энергии миньона. 
}%end item
\end{itemize}
}%end item
\divideents{getMinionLife}
\item{\vskip -1.9ex 
\membername{getMinionLife}
{\tt public int {\bf getMinionLife}(  )
\label{l172}\label{l173}}%end signature
\begin{itemize}
\sld
\item{{\bf Returns} - 
Возвращает максимальное значение жизненной энергии миньона. 
}%end item
\end{itemize}
}%end item
\divideents{getMinionMaxTurnAngle}
\item{\vskip -1.9ex 
\membername{getMinionMaxTurnAngle}
{\tt public double {\bf getMinionMaxTurnAngle}(  )
\label{l174}\label{l175}}%end signature
\begin{itemize}
\sld
\item{{\bf Returns} - 
Возвращает ограничение на изменение угла поворота миньона за один тик. 
}%end item
\end{itemize}
}%end item
\divideents{getMinionRadius}
\item{\vskip -1.9ex 
\membername{getMinionRadius}
{\tt public double {\bf getMinionRadius}(  )
\label{l176}\label{l177}}%end signature
\begin{itemize}
\sld
\item{{\bf Returns} - 
Возвращает радиус миньона. 
}%end item
\end{itemize}
}%end item
\divideents{getMinionSpeed}
\item{\vskip -1.9ex 
\membername{getMinionSpeed}
{\tt public double {\bf getMinionSpeed}(  )
\label{l178}\label{l179}}%end signature
\begin{itemize}
\sld
\item{{\bf Returns} - 
Возвращает скорость миньона при движении вперёд.
 \bl 
 Миньонам недоступно использование других видов движения, а также перемещение со скоростью, отличной от указанной. 
}%end item
\end{itemize}
}%end item
\divideents{getMinionVisionRange}
\item{\vskip -1.9ex 
\membername{getMinionVisionRange}
{\tt public double {\bf getMinionVisionRange}(  )
\label{l180}\label{l181}}%end signature
\begin{itemize}
\sld
\item{{\bf Returns} - 
Возвращает максимальное расстояние (от центра до центра), на котором миньон обнаруживает другие
 объекты. 
}%end item
\end{itemize}
}%end item
\divideents{getMovementBonusFactorPerSkillLevel}
\item{\vskip -1.9ex 
\membername{getMovementBonusFactorPerSkillLevel}
{\tt public double {\bf getMovementBonusFactorPerSkillLevel}(  )
\label{l182}\label{l183}}%end signature
\begin{itemize}
\sld
\item{{\bf Returns} - 
Возвращает относительное увеличение скорости перемещения за каждое последовательное изучение умения,
 являющегося одним из пререквизитов умения {\tt HASTE}.
 \bl 
 Увеличение скорости от действия статуса {\tt HASTENED} и увеличение скорости в результате изучения умений,
 являющихся пререквизитами умения {\tt HASTE}, являются аддитивными. Таким образом, максимальное значение
 скорости волшебника составляет
 {\tt 1.0 + 4.0 * movementBonusFactorPerSkillLevel + hastenedMovementBonusFactor} от базовой. 
}%end item
\end{itemize}
}%end item
\divideents{getOrcWoodcutterActionCooldownTicks}
\item{\vskip -1.9ex 
\membername{getOrcWoodcutterActionCooldownTicks}
{\tt public int {\bf getOrcWoodcutterActionCooldownTicks}(  )
\label{l184}\label{l185}}%end signature
\begin{itemize}
\sld
\item{{\bf Returns} - 
Возвращает минимально возможную задержку между двумя последовательными атаками орка-дровосека. 
}%end item
\end{itemize}
}%end item
\divideents{getOrcWoodcutterAttackRange}
\item{\vskip -1.9ex 
\membername{getOrcWoodcutterAttackRange}
{\tt public double {\bf getOrcWoodcutterAttackRange}(  )
\label{l186}\label{l187}}%end signature
\begin{itemize}
\sld
\item{{\bf Returns} - 
Возвращает дальность действия топора орка.
 \bl 
 Атака топором поражает все живые объекты, для каждого из которых верно, что расстояние от его центра до центра
 орка-дровосека не превышает значение {\tt orcWoodcutterAttackRange + livingUnit.radius}. 
}%end item
\end{itemize}
}%end item
\divideents{getOrcWoodcutterAttackSector}
\item{\vskip -1.9ex 
\membername{getOrcWoodcutterAttackSector}
{\tt public double {\bf getOrcWoodcutterAttackSector}(  )
\label{l188}\label{l189}}%end signature
\begin{itemize}
\sld
\item{{\bf Returns} - 
Возвращает сектор действия топора орка.
 \bl 
 Атака топором поражает все живые объекты в секторе от {\tt -orcWoodcutterAttackSector $/$ 2.0} до
 {\tt orcWoodcutterAttackSector $/$ 2.0}. 
}%end item
\end{itemize}
}%end item
\divideents{getOrcWoodcutterDamage}
\item{\vskip -1.9ex 
\membername{getOrcWoodcutterDamage}
{\tt public int {\bf getOrcWoodcutterDamage}(  )
\label{l190}\label{l191}}%end signature
\begin{itemize}
\sld
\item{{\bf Returns} - 
Возвращает урон одной атаки орка-дровосека. 
}%end item
\end{itemize}
}%end item
\divideents{getRandomSeed}
\item{\vskip -1.9ex 
\membername{getRandomSeed}
{\tt public long {\bf getRandomSeed}(  )
\label{l192}\label{l193}}%end signature
\begin{itemize}
\sld
\item{{\bf Returns} - 
Возвращает некоторое число, которое ваша стратегия может использовать для инициализации генератора
 случайных чисел. Данное значение имеет рекомендательный характер, однако позволит более точно воспроизводить
 прошедшие игры. 
}%end item
\end{itemize}
}%end item
\divideents{getRangeBonusPerSkillLevel}
\item{\vskip -1.9ex 
\membername{getRangeBonusPerSkillLevel}
{\tt public double {\bf getRangeBonusPerSkillLevel}(  )
\label{l194}\label{l195}}%end signature
\begin{itemize}
\sld
\item{{\bf Returns} - 
Возвращает абсолютное увеличение дальности заклинаний волшебника за каждое последовательное изучение
 умения, являющегося одним из пререквизитов умения {\tt ADVANCED\_MAGIC\_MISSILE}. 
}%end item
\end{itemize}
}%end item
\divideents{getRawMessageMaxLength}
\item{\vskip -1.9ex 
\membername{getRawMessageMaxLength}
{\tt public int {\bf getRawMessageMaxLength}(  )
\label{l196}\label{l197}}%end signature
\begin{itemize}
\sld
\item{{\bf Returns} - 
Возвращает максимально возможную длину низкоуровневого сообщения.
 \bl 
 Сообщения, длина которых превышает указанное значение, будут проигнорированы. 
}%end item
\end{itemize}
}%end item
\divideents{getRawMessageTransmissionSpeed}
\item{\vskip -1.9ex 
\membername{getRawMessageTransmissionSpeed}
{\tt public double {\bf getRawMessageTransmissionSpeed}(  )
\label{l198}\label{l199}}%end signature
\begin{itemize}
\sld
\item{{\bf Returns} - 
Возвращает скорость отправки сообщения.
 \bl 
 Если текстовая часть сообщения пуста, то адресат получит его уже в следующий игровой тик. В противном случае,
 момент получения сообщения будет отложен на {\tt ceil(message.rawMessage.length $/$ rawMessageTransmissionSpeed)}
 игровых тиков. 
}%end item
\end{itemize}
}%end item
\divideents{getScoreGainRange}
\item{\vskip -1.9ex 
\membername{getScoreGainRange}
{\tt public double {\bf getScoreGainRange}(  )
\label{l200}\label{l201}}%end signature
\begin{itemize}
\sld
\item{{\bf Returns} - 
Возвращает максимальное расстояние, на котором волшебник получает опыт при уничтожении союзником юнита
 другой фракции.
 \bl 
 При уничтожении противника опыт равномерно распределяется между всеми волшебниками, находящимися на расстоянии от
 цели, на превышающем {\tt scoreGainRange}, а также юнитом, нанёсшим урон, если это тоже волшебник.
 \bl 
 При нанесении противнику урона, не приводящему к уничтожению юнита, данный параметр не применяется, а опыт
 полностью достаётся атакующему волшебнику. В случае атаки миньона или строения опыт не достаётся никому.
 \bl 
 Учитывается расстояние между центрами юнитов. 
}%end item
\end{itemize}
}%end item
\divideents{getShieldCooldownTicks}
\item{\vskip -1.9ex 
\membername{getShieldCooldownTicks}
{\tt public int {\bf getShieldCooldownTicks}(  )
\label{l202}\label{l203}}%end signature
\begin{itemize}
\sld
\item{{\bf Returns} - 
Возвращает минимально возможную задержку между двумя последовательными заклинаниями <<Щит>>. 
}%end item
\end{itemize}
}%end item
\divideents{getShieldedBonusDurationFactor}
\item{\vskip -1.9ex 
\membername{getShieldedBonusDurationFactor}
{\tt public double {\bf getShieldedBonusDurationFactor}(  )
\label{l204}\label{l205}}%end signature
\begin{itemize}
\sld
\item{{\bf Returns} - 
Возвращает мультилпикатор длительности действия статуса {\tt SHIELDED} в случае подбора бонуса. 
}%end item
\end{itemize}
}%end item
\divideents{getShieldedDirectDamageAbsorptionFactor}
\item{\vskip -1.9ex 
\membername{getShieldedDirectDamageAbsorptionFactor}
{\tt public double {\bf getShieldedDirectDamageAbsorptionFactor}(  )
\label{l206}\label{l207}}%end signature
\begin{itemize}
\sld
\item{{\bf Returns} - 
Возвращает часть урона, поглощаемую щитом.
 \bl 
 Снижение урона применяется к ударам в ближнем бою, прямым попаданиям снарядов, а также взрыву <<Огненного шара>>,
 но не применяется к урону, получаемому от статусов. 
}%end item
\end{itemize}
}%end item
\divideents{getShieldedDurationTicks}
\item{\vskip -1.9ex 
\membername{getShieldedDurationTicks}
{\tt public int {\bf getShieldedDurationTicks}(  )
\label{l208}\label{l209}}%end signature
\begin{itemize}
\sld
\item{{\bf Returns} - 
Возвращает длительность действия статуса {\tt SHIELDED}. 
}%end item
\end{itemize}
}%end item
\divideents{getShieldManacost}
\item{\vskip -1.9ex 
\membername{getShieldManacost}
{\tt public int {\bf getShieldManacost}(  )
\label{l210}\label{l211}}%end signature
\begin{itemize}
\sld
\item{{\bf Returns} - 
Возвращает количество магической энергии, требуемой для заклинания <<Щит>>. 
}%end item
\end{itemize}
}%end item
\divideents{getStaffCooldownTicks}
\item{\vskip -1.9ex 
\membername{getStaffCooldownTicks}
{\tt public int {\bf getStaffCooldownTicks}(  )
\label{l212}\label{l213}}%end signature
\begin{itemize}
\sld
\item{{\bf Returns} - 
Возвращает минимально возможную задержку между двумя последовательными ударами посохом. 
}%end item
\end{itemize}
}%end item
\divideents{getStaffDamage}
\item{\vskip -1.9ex 
\membername{getStaffDamage}
{\tt public int {\bf getStaffDamage}(  )
\label{l214}\label{l215}}%end signature
\begin{itemize}
\sld
\item{{\bf Returns} - 
Возвращает базовый урон удара посохом.
 \bl 
 Эффективный урон может быть выше в результате действия некоторых аур и$/$или изучения волшебником некоторых
 умений. 
}%end item
\end{itemize}
}%end item
\divideents{getStaffDamageBonusPerSkillLevel}
\item{\vskip -1.9ex 
\membername{getStaffDamageBonusPerSkillLevel}
{\tt public int {\bf getStaffDamageBonusPerSkillLevel}(  )
\label{l216}\label{l217}}%end signature
\begin{itemize}
\sld
\item{{\bf Returns} - 
Возвращает абсолютное увеличение урона, наносимого волшебником в ближнем бою, за каждое последовательное
 изучение умения, являющегося одним из пререквизитов умения {\tt FIREBALL}. 
}%end item
\end{itemize}
}%end item
\divideents{getStaffRange}
\item{\vskip -1.9ex 
\membername{getStaffRange}
{\tt public double {\bf getStaffRange}(  )
\label{l218}\label{l219}}%end signature
\begin{itemize}
\sld
\item{{\bf Returns} - 
Возвращает дальность действия посоха волшебника.
 \bl 
 Атака посохом поражает все живые объекты, для каждого из которых верно, что расстояние от его центра до центра
 волшебника не превышает значение {\tt staffRange + livingUnit.radius}. 
}%end item
\end{itemize}
}%end item
\divideents{getStaffSector}
\item{\vskip -1.9ex 
\membername{getStaffSector}
{\tt public double {\bf getStaffSector}(  )
\label{l220}\label{l221}}%end signature
\begin{itemize}
\sld
\item{{\bf Returns} - 
Возвращает сектор действия посоха волшебника.
 \bl 
 Атака посохом поражает все живые объекты в секторе от {\tt -staffSector $/$ 2.0} до {\tt staffSector $/$ 2.0}.
 Этим же интервалом ограничены относительный угол снаряда, а также зона применения магического статуса. 
}%end item
\end{itemize}
}%end item
\divideents{getTeamWorkingScoreFactor}
\item{\vskip -1.9ex 
\membername{getTeamWorkingScoreFactor}
{\tt public double {\bf getTeamWorkingScoreFactor}(  )
\label{l222}\label{l223}}%end signature
\begin{itemize}
\sld
\item{{\bf Returns} - 
Возвращает мультипликатор опыта, применяемый в случае уничтожения юнита противника при участии двух или
 более волшебников.
 \bl 
 После применения мультипликатора количество опыта округляется вниз до ближайшего целого значения. 
}%end item
\end{itemize}
}%end item
\divideents{getTickCount}
\item{\vskip -1.9ex 
\membername{getTickCount}
{\tt public int {\bf getTickCount}(  )
\label{l224}\label{l225}}%end signature
\begin{itemize}
\sld
\item{{\bf Returns} - 
Возвращает базовую длительность игры в тиках. Реальная длительность может отличаться от этого значения в
 меньшую сторону. Эквивалентно {\tt world.tickCount}. 
}%end item
\end{itemize}
}%end item
\divideents{getVictoryScore}
\item{\vskip -1.9ex 
\membername{getVictoryScore}
{\tt public int {\bf getVictoryScore}(  )
\label{l226}\label{l227}}%end signature
\begin{itemize}
\sld
\item{{\bf Returns} - 
Возвращает количество баллов, получаемых всеми игроками фракции в случае победы --- разрушения базы
 противоположной фракции. 
}%end item
\end{itemize}
}%end item
\divideents{getWizardActionCooldownTicks}
\item{\vskip -1.9ex 
\membername{getWizardActionCooldownTicks}
{\tt public int {\bf getWizardActionCooldownTicks}(  )
\label{l228}\label{l229}}%end signature
\begin{itemize}
\sld
\item{{\bf Returns} - 
Возвращает минимально возможную задержку между любыми двумя последовательными действиями волшебника. 
}%end item
\end{itemize}
}%end item
\divideents{getWizardBackwardSpeed}
\item{\vskip -1.9ex 
\membername{getWizardBackwardSpeed}
{\tt public double {\bf getWizardBackwardSpeed}(  )
\label{l230}\label{l231}}%end signature
\begin{itemize}
\sld
\item{{\bf Returns} - 
Возвращает базовое ограничение скорости волшебника при движении назад.
 \bl 
 Эффективное ограничение может быть выше в результате действия некоторых аур и$/$или изучения волшебником некоторых
 умений, а также в результате действия статуса {\tt HASTENED}. 
}%end item
\end{itemize}
}%end item
\divideents{getWizardBaseLife}
\item{\vskip -1.9ex 
\membername{getWizardBaseLife}
{\tt public int {\bf getWizardBaseLife}(  )
\label{l232}\label{l233}}%end signature
\begin{itemize}
\sld
\item{{\bf Returns} - 
Возвращает максимальное значение жизненной энергии волшебника на уровне {\tt 0}. 
}%end item
\end{itemize}
}%end item
\divideents{getWizardBaseLifeRegeneration}
\item{\vskip -1.9ex 
\membername{getWizardBaseLifeRegeneration}
{\tt public double {\bf getWizardBaseLifeRegeneration}(  )
\label{l234}\label{l235}}%end signature
\begin{itemize}
\sld
\item{{\bf Returns} - 
Возвращает количество жизненной энергии, которое волшебник уровня {\tt 0} восстанавливает за один тик. 
}%end item
\end{itemize}
}%end item
\divideents{getWizardBaseMana}
\item{\vskip -1.9ex 
\membername{getWizardBaseMana}
{\tt public int {\bf getWizardBaseMana}(  )
\label{l236}\label{l237}}%end signature
\begin{itemize}
\sld
\item{{\bf Returns} - 
Возвращает максимальное значение магической энергии волшебника на уровне {\tt 0}. 
}%end item
\end{itemize}
}%end item
\divideents{getWizardBaseManaRegeneration}
\item{\vskip -1.9ex 
\membername{getWizardBaseManaRegeneration}
{\tt public double {\bf getWizardBaseManaRegeneration}(  )
\label{l238}\label{l239}}%end signature
\begin{itemize}
\sld
\item{{\bf Returns} - 
Возвращает количество магической энергии, которое волшебник уровня {\tt 0} восстанавливает за один тик. 
}%end item
\end{itemize}
}%end item
\divideents{getWizardCastRange}
\item{\vskip -1.9ex 
\membername{getWizardCastRange}
{\tt public double {\bf getWizardCastRange}(  )
\label{l240}\label{l241}}%end signature
\begin{itemize}
\sld
\item{{\bf Returns} - 
Возвращает базовую дальность заклинаний волшебника.
 \bl 
 Эффективная дальность ({\tt wizard.castRange}) может быть выше в результате действия некоторых аур и$/$или
 изучения волшебником некоторых умений. 
}%end item
\end{itemize}
}%end item
\divideents{getWizardDamageScoreFactor}
\item{\vskip -1.9ex 
\membername{getWizardDamageScoreFactor}
{\tt public double {\bf getWizardDamageScoreFactor}(  )
\label{l242}\label{l243}}%end signature
\begin{itemize}
\sld
\item{{\bf Returns} - 
Возвращает коэффициент опыта, получаемого волшебником при нанесении урона волшебникам противоположной
 фракции. 
}%end item
\end{itemize}
}%end item
\divideents{getWizardEliminationScoreFactor}
\item{\vskip -1.9ex 
\membername{getWizardEliminationScoreFactor}
{\tt public double {\bf getWizardEliminationScoreFactor}(  )
\label{l244}\label{l245}}%end signature
\begin{itemize}
\sld
\item{{\bf Returns} - 
Возвращает коэффициент опыта, получаемого волшебником за разрушение телесной оболочки волшебника
 противоположной фракции.
 \bl 
 Применяется к максимальному количеству жизненной энергии волшебника. 
}%end item
\end{itemize}
}%end item
\divideents{getWizardForwardSpeed}
\item{\vskip -1.9ex 
\membername{getWizardForwardSpeed}
{\tt public double {\bf getWizardForwardSpeed}(  )
\label{l246}\label{l247}}%end signature
\begin{itemize}
\sld
\item{{\bf Returns} - 
Возвращает базовое ограничение скорости волшебника при движении вперёд.
 \bl 
 Эффективное ограничение может быть выше в результате действия некоторых аур и$/$или изучения волшебником некоторых
 умений, а также в результате действия статуса {\tt HASTENED}. 
}%end item
\end{itemize}
}%end item
\divideents{getWizardLifeGrowthPerLevel}
\item{\vskip -1.9ex 
\membername{getWizardLifeGrowthPerLevel}
{\tt public int {\bf getWizardLifeGrowthPerLevel}(  )
\label{l248}\label{l249}}%end signature
\begin{itemize}
\sld
\item{{\bf Returns} - 
Возвращает прирост жизненной энергии волшебника за уровень. 
}%end item
\end{itemize}
}%end item
\divideents{getWizardLifeRegenerationGrowthPerLevel}
\item{\vskip -1.9ex 
\membername{getWizardLifeRegenerationGrowthPerLevel}
{\tt public double {\bf getWizardLifeRegenerationGrowthPerLevel}(  )
\label{l250}\label{l251}}%end signature
\begin{itemize}
\sld
\item{{\bf Returns} - 
Возвращает прирост скорости регенерации жизненной энергии волшебника за один уровень. 
}%end item
\end{itemize}
}%end item
\divideents{getWizardManaGrowthPerLevel}
\item{\vskip -1.9ex 
\membername{getWizardManaGrowthPerLevel}
{\tt public int {\bf getWizardManaGrowthPerLevel}(  )
\label{l252}\label{l253}}%end signature
\begin{itemize}
\sld
\item{{\bf Returns} - 
Возвращает прирост магической энергии волшебника за уровень. 
}%end item
\end{itemize}
}%end item
\divideents{getWizardManaRegenerationGrowthPerLevel}
\item{\vskip -1.9ex 
\membername{getWizardManaRegenerationGrowthPerLevel}
{\tt public double {\bf getWizardManaRegenerationGrowthPerLevel}(  )
\label{l254}\label{l255}}%end signature
\begin{itemize}
\sld
\item{{\bf Returns} - 
Возвращает прирост скорости регенерации магической энергии волшебника за один уровень. 
}%end item
\end{itemize}
}%end item
\divideents{getWizardMaxResurrectionDelayTicks}
\item{\vskip -1.9ex 
\membername{getWizardMaxResurrectionDelayTicks}
{\tt public int {\bf getWizardMaxResurrectionDelayTicks}(  )
\label{l256}\label{l257}}%end signature
\begin{itemize}
\sld
\item{{\bf Returns} - 
Возвращает максимально возможную задержку возрождения волшебника после смерти его телесной оболочки.
 \bl 
 Если волшебник погибает сразу после своего возрождения, то он будет автоматически воскрешён на своей начальной
 позиции (или недалеко от неё, если это невозможно) через {\tt wizardMaxResurrectionDelayTicks} тиков. Каждый
 игровой тик жизни волшебника уменьшшает эту задержку на единицу. Задержка возрождения не может стать меньше, чем
 {\tt wizardMinResurrectionDelayTicks}. 
}%end item
\end{itemize}
}%end item
\divideents{getWizardMaxTurnAngle}
\item{\vskip -1.9ex 
\membername{getWizardMaxTurnAngle}
{\tt public double {\bf getWizardMaxTurnAngle}(  )
\label{l258}\label{l259}}%end signature
\begin{itemize}
\sld
\item{{\bf Returns} - 
Возвращает базовое ограничение на изменение угла поворота волшебника за один тик.
 \bl 
 Эффективное ограничение может быть выше в {\tt 1.0 + hastenedRotationBonusFactor} раз в результате действия
 статуса {\tt HASTENED}. 
}%end item
\end{itemize}
}%end item
\divideents{getWizardMinResurrectionDelayTicks}
\item{\vskip -1.9ex 
\membername{getWizardMinResurrectionDelayTicks}
{\tt public int {\bf getWizardMinResurrectionDelayTicks}(  )
\label{l260}\label{l261}}%end signature
\begin{itemize}
\sld
\item{{\bf Returns} - 
Возвращает минимально возможную задержку возрождения волшебника после смерти его телесной оболочки.
 \bl 
 Если волшебник погибает сразу после своего возрождения, то он будет автоматически воскрешён на своей начальной
 позиции (или недалеко от неё, если это невозможно) через {\tt wizardMaxResurrectionDelayTicks} тиков. Каждый
 игровой тик жизни волшебника уменьшшает эту задержку на единицу. Задержка возрождения не может стать меньше, чем
 {\tt wizardMinResurrectionDelayTicks}. 
}%end item
\end{itemize}
}%end item
\divideents{getWizardRadius}
\item{\vskip -1.9ex 
\membername{getWizardRadius}
{\tt public double {\bf getWizardRadius}(  )
\label{l262}\label{l263}}%end signature
\begin{itemize}
\sld
\item{{\bf Returns} - 
Возвращает радиус волшебника. 
}%end item
\end{itemize}
}%end item
\divideents{getWizardStrafeSpeed}
\item{\vskip -1.9ex 
\membername{getWizardStrafeSpeed}
{\tt public double {\bf getWizardStrafeSpeed}(  )
\label{l264}\label{l265}}%end signature
\begin{itemize}
\sld
\item{{\bf Returns} - 
Возвращает базовое ограничение скорости волшебника при движении боком.
 \bl 
 Эффективное ограничение может быть выше в результате действия некоторых аур и$/$или изучения волшебником некоторых
 умений, а также в результате действия статуса {\tt HASTENED}. 
}%end item
\end{itemize}
}%end item
\divideents{getWizardVisionRange}
\item{\vskip -1.9ex 
\membername{getWizardVisionRange}
{\tt public double {\bf getWizardVisionRange}(  )
\label{l266}\label{l267}}%end signature
\begin{itemize}
\sld
\item{{\bf Returns} - 
Возвращает максимальное расстояние (от центра до центра), на котором волшебник обнаруживает другие
 объекты. 
}%end item
\end{itemize}
}%end item
\divideents{isRawMessagesEnabled}
\item{\vskip -1.9ex 
\membername{isRawMessagesEnabled}
{\tt public boolean {\bf isRawMessagesEnabled}(  )
\label{l268}\label{l269}}%end signature
\begin{itemize}
\sld
\item{{\bf Returns} - 
Возвращает {\tt true}, если и только если верховные волшебники в данной игре могут передавать
 низкоуровневые сообщения другим волшебникам своей фракции. 
}%end item
\end{itemize}
}%end item
\divideents{isSkillsEnabled}
\item{\vskip -1.9ex 
\membername{isSkillsEnabled}
{\tt public boolean {\bf isSkillsEnabled}(  )
\label{l270}\label{l271}}%end signature
\begin{itemize}
\sld
\item{{\bf Returns} - 
Возвращает {\tt true}, если и только если в данной игре волшебники могут повышать свой уровень
 (накапливая опыт) и изучать новые умения. 
}%end item
\end{itemize}
}%end item
\end{itemize}
}
\hide{inherited}{
}
}
\startsection{Class}{LaneType}{l8}{%
{\small Тип дорожки.}
\vskip .1in 
\startsubsubsection{Declaration}{
\fbox{\vbox{
\hbox{\vbox{\small public final 
class 
LaneType}}
\noindent\hbox{\vbox{{\bf extends} Enum}}
}}}
\startsubsubsection{Fields}{
\begin{itemize}
\item{
public static final LaneType TOP\begin{itemize}\item{\vskip -.9ex Верхняя. Проходит через левый нижний, левый верхний и правый верхний углы карты.}\end{itemize}
}
\item{
public static final LaneType MIDDLE\begin{itemize}\item{\vskip -.9ex Центральная. Напрямую соединяет левый нижний и правый верхний углы карты.}\end{itemize}
}
\item{
public static final LaneType BOTTOM\begin{itemize}\item{\vskip -.9ex Нижняя. Проходит через левый нижний, правый нижний и правый верхний углы карты.}\end{itemize}
}
\end{itemize}
}
\hide{inherited}{
\startsubsubsection{Methods inherited from class {\tt Enum}}{
\par{\small 
\refdefined{l24}\vskip -2em
\begin{itemize}
\item{\vskip -1.9ex 
\membername{clone}
{\tt protected final Object {\bf clone}(  )
}%end signature
}%end item
\divideents{compareTo}
\item{\vskip -1.9ex 
\membername{compareTo}
{\tt public final int {\bf compareTo}( {\tt Enum } {\bf arg0} )
}%end signature
}%end item
\divideents{equals}
\item{\vskip -1.9ex 
\membername{equals}
{\tt public final boolean {\bf equals}( {\tt Object } {\bf arg0} )
}%end signature
}%end item
\divideents{finalize}
\item{\vskip -1.9ex 
\membername{finalize}
{\tt protected final void {\bf finalize}(  )
}%end signature
}%end item
\divideents{getDeclaringClass}
\item{\vskip -1.9ex 
\membername{getDeclaringClass}
{\tt public final Class {\bf getDeclaringClass}(  )
}%end signature
}%end item
\divideents{hashCode}
\item{\vskip -1.9ex 
\membername{hashCode}
{\tt public final int {\bf hashCode}(  )
}%end signature
}%end item
\divideents{name}
\item{\vskip -1.9ex 
\membername{name}
{\tt public final String {\bf name}(  )
}%end signature
}%end item
\divideents{ordinal}
\item{\vskip -1.9ex 
\membername{ordinal}
{\tt public final int {\bf ordinal}(  )
}%end signature
}%end item
\divideents{toString}
\item{\vskip -1.9ex 
\membername{toString}
{\tt public String {\bf toString}(  )
}%end signature
}%end item
\divideents{valueOf}
\item{\vskip -1.9ex 
\membername{valueOf}
{\tt public static Enum {\bf valueOf}( {\tt Class } {\bf arg0},
{\tt String } {\bf arg1} )
}%end signature
}%end item
\end{itemize}
}}
}
}
\startsection{Class}{LivingUnit}{l9}{%
{\small Класс, определяющий живого юнита круглой формы.}
\vskip .1in 
\startsubsubsection{Declaration}{
\fbox{\vbox{
\hbox{\vbox{\small public abstract 
class 
LivingUnit}}
\noindent\hbox{\vbox{{\bf extends} CircularUnit}}
}}}

% Removed by DocsPostProcessor:
% \startsubsubsection{Constructors}{
% \vskip -2em
% \begin{itemize}
% \item{\vskip -1.9ex 
% \membername{LivingUnit}
% {\tt protected {\bf LivingUnit}( {\tt long } {\bf id},
% {\tt double } {\bf x},
% {\tt double } {\bf y},
% {\tt double } {\bf speedX},
% {\tt double } {\bf speedY},
% {\tt double } {\bf angle},
% {\tt Faction } {\bf faction},
% {\tt double } {\bf radius},
% {\tt int } {\bf life},
% {\tt int } {\bf maxLife},
% {\tt Status[]} {\bf statuses} )
% \label{l272}\label{l273}}%end signature
% }%end item
% \end{itemize}
% }
% \\ Removed by DocsPostProcessor.

\startsubsubsection{Methods}{
\vskip -2em
\begin{itemize}
\item{\vskip -1.9ex 
\membername{getLife}
{\tt public int {\bf getLife}(  )
\label{l274}\label{l275}}%end signature
\begin{itemize}
\sld
\item{{\bf Returns} - 
Возвращает текущее количество жизненной энергии. 
}%end item
\end{itemize}
}%end item
\divideents{getMaxLife}
\item{\vskip -1.9ex 
\membername{getMaxLife}
{\tt public int {\bf getMaxLife}(  )
\label{l276}\label{l277}}%end signature
\begin{itemize}
\sld
\item{{\bf Returns} - 
Возвращает максимальное количество жизненной энергии. 
}%end item
\end{itemize}
}%end item
\divideents{getStatuses}
\item{\vskip -1.9ex 
\membername{getStatuses}
{\tt public Status[] {\bf getStatuses}(  )
\label{l278}\label{l279}}%end signature
\begin{itemize}
\sld
\item{{\bf Returns} - 
Возвращает магические статусы, влияющие на живого юнита. 
}%end item
\end{itemize}
}%end item
\end{itemize}
}
\hide{inherited}{
\startsubsubsection{Methods inherited from class {\tt CircularUnit}}{
\par{\small 
\refdefined{l5}\vskip -2em
\begin{itemize}
\item{\vskip -1.9ex 
\membername{getRadius}
{\tt public double {\bf getRadius}(  )
}%end signature
\begin{itemize}
\sld
\item{{\bf Returns} - 
Возвращает радиус объекта. 
}%end item
\end{itemize}
}%end item
\end{itemize}
}}
\startsubsubsection{Methods inherited from class {\tt Unit}}{
\par{\small 
\refdefined{l21}\vskip -2em
\begin{itemize}
\item{\vskip -1.9ex 
\membername{getAngle}
{\tt public final double {\bf getAngle}(  )
}%end signature
\begin{itemize}
\sld
\item{{\bf Returns} - 
Возвращает угол поворота объекта в радианах. Нулевой угол соответствует направлению оси абсцисс.
 Положительные значения соответствуют повороту по часовой стрелке. 
}%end item
\end{itemize}
}%end item
\divideents{getAngleTo}
\item{\vskip -1.9ex 
\membername{getAngleTo}
{\tt public double {\bf getAngleTo}( {\tt double } {\bf x},
{\tt double } {\bf y} )
}%end signature
\begin{itemize}
\sld
\item{
\sld
{\bf Parameters}
\sld\isep
  \begin{itemize}
\sld\isep
   \item{
\sld
{\tt x} - X-координата точки.}
   \item{
\sld
{\tt y} - Y-координата точки.}
  \end{itemize}
}%end item
\item{{\bf Returns} - 
Возвращает ориентированный угол [{\tt -PI}, {\tt PI}] между направлением
 данного объекта и вектором из центра данного объекта к указанной точке. 
}%end item
\end{itemize}
}%end item
\divideents{getAngleTo}
\item{\vskip -1.9ex 
\membername{getAngleTo}
{\tt public double {\bf getAngleTo}( {\tt Unit } {\bf unit} )
}%end signature
\begin{itemize}
\sld
\item{
\sld
{\bf Parameters}
\sld\isep
  \begin{itemize}
\sld\isep
   \item{
\sld
{\tt unit} - Объект, к центру которого необходимо определить угол.}
  \end{itemize}
}%end item
\item{{\bf Returns} - 
Возвращает ориентированный угол [{\tt -PI}, {\tt PI}] между направлением
 данного объекта и вектором из центра данного объекта к центру указанного объекта. 
}%end item
\end{itemize}
}%end item
\divideents{getDistanceTo}
\item{\vskip -1.9ex 
\membername{getDistanceTo}
{\tt public double {\bf getDistanceTo}( {\tt double } {\bf x},
{\tt double } {\bf y} )
}%end signature
\begin{itemize}
\sld
\item{
\sld
{\bf Parameters}
\sld\isep
  \begin{itemize}
\sld\isep
   \item{
\sld
{\tt x} - X-координата точки.}
   \item{
\sld
{\tt y} - Y-координата точки.}
  \end{itemize}
}%end item
\item{{\bf Returns} - 
Возвращает расстояние до точки от центра данного объекта. 
}%end item
\end{itemize}
}%end item
\divideents{getDistanceTo}
\item{\vskip -1.9ex 
\membername{getDistanceTo}
{\tt public double {\bf getDistanceTo}( {\tt Unit } {\bf unit} )
}%end signature
\begin{itemize}
\sld
\item{
\sld
{\bf Parameters}
\sld\isep
  \begin{itemize}
\sld\isep
   \item{
\sld
{\tt unit} - Объект, до центра которого необходимо определить расстояние.}
  \end{itemize}
}%end item
\item{{\bf Returns} - 
Возвращает расстояние от центра данного объекта до центра указанного объекта. 
}%end item
\end{itemize}
}%end item
\divideents{getFaction}
\item{\vskip -1.9ex 
\membername{getFaction}
{\tt public Faction {\bf getFaction}(  )
}%end signature
\begin{itemize}
\sld
\item{{\bf Returns} - 
Возвращает фракцию, к которой принадлежит данный юнит. 
}%end item
\end{itemize}
}%end item
\divideents{getId}
\item{\vskip -1.9ex 
\membername{getId}
{\tt public long {\bf getId}(  )
}%end signature
\begin{itemize}
\sld
\item{{\bf Returns} - 
Возвращает уникальный идентификатор объекта. 
}%end item
\end{itemize}
}%end item
\divideents{getSpeedX}
\item{\vskip -1.9ex 
\membername{getSpeedX}
{\tt public final double {\bf getSpeedX}(  )
}%end signature
\begin{itemize}
\sld
\item{{\bf Returns} - 
Возвращает X-составляющую скорости объекта. Ось абсцисс направлена слева направо.
 \bl 
 Для юнитов, способных мгновенно менять свою скорость, возвращается значение перемещения за последний тик. 
}%end item
\end{itemize}
}%end item
\divideents{getSpeedY}
\item{\vskip -1.9ex 
\membername{getSpeedY}
{\tt public final double {\bf getSpeedY}(  )
}%end signature
\begin{itemize}
\sld
\item{{\bf Returns} - 
Возвращает Y-составляющую скорости объекта. Ось ординат направлена сверху вниз.
 \bl 
 Для юнитов, способных мгновенно менять свою скорость, возвращается значение перемещения за последний тик. 
}%end item
\end{itemize}
}%end item
\divideents{getX}
\item{\vskip -1.9ex 
\membername{getX}
{\tt public final double {\bf getX}(  )
}%end signature
\begin{itemize}
\sld
\item{{\bf Returns} - 
Возвращает X-координату центра объекта. Ось абсцисс направлена слева направо. 
}%end item
\end{itemize}
}%end item
\divideents{getY}
\item{\vskip -1.9ex 
\membername{getY}
{\tt public final double {\bf getY}(  )
}%end signature
\begin{itemize}
\sld
\item{{\bf Returns} - 
Возвращает Y-координату центра объекта. Ось ординат направлена сверху вниз. 
}%end item
\end{itemize}
}%end item
\end{itemize}
}}
}
}
\startsection{Class}{Message}{l10}{%
{\small Класс определяет сообщение, которое верховный волшебник ({\tt wizard.master}) может отправлять другим членам
 фракции, используя телепатическую связь.
 \bl 
 Сообщение отправляется персонально каждому волшебнику. Другие волшебники не могут его перехватить.
 \bl 
 Адресат получает сообщение в следующий игровой тик или позднее, в зависимости от размера сообщения.
 \bl 
 Волшебник волен проигнорировать как любую отдельную часть сообщения, так и всё сообщение целиком, однако это может
 привести к поражению дружественной фракции.}
\vskip .1in 
\startsubsubsection{Declaration}{
\fbox{\vbox{
\hbox{\vbox{\small public 
class 
Message}}
\noindent\hbox{\vbox{{\bf extends} Object}}
}}}

% Removed by DocsPostProcessor:
% \startsubsubsection{Constructors}{
% \vskip -2em
% \begin{itemize}
% \item{\vskip -1.9ex 
% \membername{Message}
% {\tt public {\bf Message}( {\tt LaneType } {\bf lane},
% {\tt SkillType } {\bf skillToLearn},
% {\tt byte[]} {\bf rawMessage} )
% \label{l280}\label{l281}}%end signature
% }%end item
% \end{itemize}
% }
% \\ Removed by DocsPostProcessor.

\startsubsubsection{Methods}{
\vskip -2em
\begin{itemize}
\item{\vskip -1.9ex 
\membername{getLane}
{\tt public LaneType {\bf getLane}(  )
\label{l282}\label{l283}}%end signature
\begin{itemize}
\sld
\item{{\bf Returns} - 
Возвращает указание контролировать определённую дорожку. 
}%end item
\end{itemize}
}%end item
\divideents{getRawMessage}
\item{\vskip -1.9ex 
\membername{getRawMessage}
{\tt public byte[] {\bf getRawMessage}(  )
\label{l284}\label{l285}}%end signature
\begin{itemize}
\sld
\item{{\bf Returns} - 
Возвращает текстовое сообщение на забытом древнем языке.
 \bl 
 Максимальная длина сообщения составляет {\tt game.rawMessageMaxLength}. При этом, скорость отправки сообщения
 зависит от его длины. Если текстовая часть сообщения пуста, то адресат получит его уже в следующий игровой тик.
 В противном случае, момент получения сообщения будет отложен на
 {\tt ceil(rawMessage.length $/$ game.rawMessageTransmissionSpeed)} игровых тиков.
 \bl 
 Значение поля может быть доступно не во всех режимах игры. 
}%end item
\end{itemize}
}%end item
\divideents{getSkillToLearn}
\item{\vskip -1.9ex 
\membername{getSkillToLearn}
{\tt public SkillType {\bf getSkillToLearn}(  )
\label{l286}\label{l287}}%end signature
\begin{itemize}
\sld
\item{{\bf Returns} - 
Возвращает указание изучить какое-либо умение.
 \bl 
 Умение может требовать предварительного изучения других умений или быть недоступно для изучения в данный момент в
 связи с низким уровнем волшебника. Волшебнику рекомендуется запомнить указание и двигаться в направлении его
 реализации. При этом, более позднее указание должно считаться более приоритетным.
 \bl 
 Значение поля может быть доступно не во всех режимах игры. 
}%end item
\end{itemize}
}%end item
\end{itemize}
}
\hide{inherited}{
}
}
\startsection{Class}{Minion}{l11}{%
{\small Класс, определяющий приспешника волшебника одной из фракций. Содержит также все свойства живого юнита.
 \bl 
 Миньоны, оставшиеся по той или иной причине без хозяина, часто объединяются в небольшие группы и селятся в лесах.
 Они крайне настороженно относятся ко всем другим волшебникам и их миньонам.}
\vskip .1in 
\startsubsubsection{Declaration}{
\fbox{\vbox{
\hbox{\vbox{\small public 
class 
Minion}}
\noindent\hbox{\vbox{{\bf extends} LivingUnit}}
}}}

% Removed by DocsPostProcessor:
% \startsubsubsection{Constructors}{
% \vskip -2em
% \begin{itemize}
% \item{\vskip -1.9ex 
% \membername{Minion}
% {\tt public {\bf Minion}( {\tt long } {\bf id},
% {\tt double } {\bf x},
% {\tt double } {\bf y},
% {\tt double } {\bf speedX},
% {\tt double } {\bf speedY},
% {\tt double } {\bf angle},
% {\tt Faction } {\bf faction},
% {\tt double } {\bf radius},
% {\tt int } {\bf life},
% {\tt int } {\bf maxLife},
% {\tt Status[]} {\bf statuses},
% {\tt MinionType } {\bf type},
% {\tt double } {\bf visionRange},
% {\tt int } {\bf damage},
% {\tt int } {\bf cooldownTicks},
% {\tt int } {\bf remainingActionCooldownTicks} )
% \label{l288}\label{l289}}%end signature
% }%end item
% \end{itemize}
% }
% \\ Removed by DocsPostProcessor.

\startsubsubsection{Methods}{
\vskip -2em
\begin{itemize}
\item{\vskip -1.9ex 
\membername{getCooldownTicks}
{\tt public int {\bf getCooldownTicks}(  )
\label{l290}\label{l291}}%end signature
\begin{itemize}
\sld
\item{{\bf Returns} - 
Возвращает интервал между атаками. 
}%end item
\end{itemize}
}%end item
\divideents{getDamage}
\item{\vskip -1.9ex 
\membername{getDamage}
{\tt public int {\bf getDamage}(  )
\label{l292}\label{l293}}%end signature
\begin{itemize}
\sld
\item{{\bf Returns} - 
Возвращает урон одной атаки. 
}%end item
\end{itemize}
}%end item
\divideents{getRemainingActionCooldownTicks}
\item{\vskip -1.9ex 
\membername{getRemainingActionCooldownTicks}
{\tt public int {\bf getRemainingActionCooldownTicks}(  )
\label{l294}\label{l295}}%end signature
\begin{itemize}
\sld
\item{{\bf Returns} - 
Возвращает количество тиков, оставшееся до следующей атаки. 
}%end item
\end{itemize}
}%end item
\divideents{getType}
\item{\vskip -1.9ex 
\membername{getType}
{\tt public MinionType {\bf getType}(  )
\label{l296}\label{l297}}%end signature
\begin{itemize}
\sld
\item{{\bf Returns} - 
Возвращает тип миньона. 
}%end item
\end{itemize}
}%end item
\divideents{getVisionRange}
\item{\vskip -1.9ex 
\membername{getVisionRange}
{\tt public double {\bf getVisionRange}(  )
\label{l298}\label{l299}}%end signature
\begin{itemize}
\sld
\item{{\bf Returns} - 
Возвращает максимальное расстояние (от центра до центра),
 на котором данный миньон обнаруживает другие объекты. 
}%end item
\end{itemize}
}%end item
\end{itemize}
}
\hide{inherited}{
\startsubsubsection{Methods inherited from class {\tt LivingUnit}}{
\par{\small 
\refdefined{l9}\vskip -2em
\begin{itemize}
\item{\vskip -1.9ex 
\membername{getLife}
{\tt public int {\bf getLife}(  )
}%end signature
\begin{itemize}
\sld
\item{{\bf Returns} - 
Возвращает текущее количество жизненной энергии. 
}%end item
\end{itemize}
}%end item
\divideents{getMaxLife}
\item{\vskip -1.9ex 
\membername{getMaxLife}
{\tt public int {\bf getMaxLife}(  )
}%end signature
\begin{itemize}
\sld
\item{{\bf Returns} - 
Возвращает максимальное количество жизненной энергии. 
}%end item
\end{itemize}
}%end item
\divideents{getStatuses}
\item{\vskip -1.9ex 
\membername{getStatuses}
{\tt public Status[] {\bf getStatuses}(  )
}%end signature
\begin{itemize}
\sld
\item{{\bf Returns} - 
Возвращает магические статусы, влияющие на живого юнита. 
}%end item
\end{itemize}
}%end item
\end{itemize}
}}
\startsubsubsection{Methods inherited from class {\tt CircularUnit}}{
\par{\small 
\refdefined{l5}\vskip -2em
\begin{itemize}
\item{\vskip -1.9ex 
\membername{getRadius}
{\tt public double {\bf getRadius}(  )
}%end signature
\begin{itemize}
\sld
\item{{\bf Returns} - 
Возвращает радиус объекта. 
}%end item
\end{itemize}
}%end item
\end{itemize}
}}
\startsubsubsection{Methods inherited from class {\tt Unit}}{
\par{\small 
\refdefined{l21}\vskip -2em
\begin{itemize}
\item{\vskip -1.9ex 
\membername{getAngle}
{\tt public final double {\bf getAngle}(  )
}%end signature
\begin{itemize}
\sld
\item{{\bf Returns} - 
Возвращает угол поворота объекта в радианах. Нулевой угол соответствует направлению оси абсцисс.
 Положительные значения соответствуют повороту по часовой стрелке. 
}%end item
\end{itemize}
}%end item
\divideents{getAngleTo}
\item{\vskip -1.9ex 
\membername{getAngleTo}
{\tt public double {\bf getAngleTo}( {\tt double } {\bf x},
{\tt double } {\bf y} )
}%end signature
\begin{itemize}
\sld
\item{
\sld
{\bf Parameters}
\sld\isep
  \begin{itemize}
\sld\isep
   \item{
\sld
{\tt x} - X-координата точки.}
   \item{
\sld
{\tt y} - Y-координата точки.}
  \end{itemize}
}%end item
\item{{\bf Returns} - 
Возвращает ориентированный угол [{\tt -PI}, {\tt PI}] между направлением
 данного объекта и вектором из центра данного объекта к указанной точке. 
}%end item
\end{itemize}
}%end item
\divideents{getAngleTo}
\item{\vskip -1.9ex 
\membername{getAngleTo}
{\tt public double {\bf getAngleTo}( {\tt Unit } {\bf unit} )
}%end signature
\begin{itemize}
\sld
\item{
\sld
{\bf Parameters}
\sld\isep
  \begin{itemize}
\sld\isep
   \item{
\sld
{\tt unit} - Объект, к центру которого необходимо определить угол.}
  \end{itemize}
}%end item
\item{{\bf Returns} - 
Возвращает ориентированный угол [{\tt -PI}, {\tt PI}] между направлением
 данного объекта и вектором из центра данного объекта к центру указанного объекта. 
}%end item
\end{itemize}
}%end item
\divideents{getDistanceTo}
\item{\vskip -1.9ex 
\membername{getDistanceTo}
{\tt public double {\bf getDistanceTo}( {\tt double } {\bf x},
{\tt double } {\bf y} )
}%end signature
\begin{itemize}
\sld
\item{
\sld
{\bf Parameters}
\sld\isep
  \begin{itemize}
\sld\isep
   \item{
\sld
{\tt x} - X-координата точки.}
   \item{
\sld
{\tt y} - Y-координата точки.}
  \end{itemize}
}%end item
\item{{\bf Returns} - 
Возвращает расстояние до точки от центра данного объекта. 
}%end item
\end{itemize}
}%end item
\divideents{getDistanceTo}
\item{\vskip -1.9ex 
\membername{getDistanceTo}
{\tt public double {\bf getDistanceTo}( {\tt Unit } {\bf unit} )
}%end signature
\begin{itemize}
\sld
\item{
\sld
{\bf Parameters}
\sld\isep
  \begin{itemize}
\sld\isep
   \item{
\sld
{\tt unit} - Объект, до центра которого необходимо определить расстояние.}
  \end{itemize}
}%end item
\item{{\bf Returns} - 
Возвращает расстояние от центра данного объекта до центра указанного объекта. 
}%end item
\end{itemize}
}%end item
\divideents{getFaction}
\item{\vskip -1.9ex 
\membername{getFaction}
{\tt public Faction {\bf getFaction}(  )
}%end signature
\begin{itemize}
\sld
\item{{\bf Returns} - 
Возвращает фракцию, к которой принадлежит данный юнит. 
}%end item
\end{itemize}
}%end item
\divideents{getId}
\item{\vskip -1.9ex 
\membername{getId}
{\tt public long {\bf getId}(  )
}%end signature
\begin{itemize}
\sld
\item{{\bf Returns} - 
Возвращает уникальный идентификатор объекта. 
}%end item
\end{itemize}
}%end item
\divideents{getSpeedX}
\item{\vskip -1.9ex 
\membername{getSpeedX}
{\tt public final double {\bf getSpeedX}(  )
}%end signature
\begin{itemize}
\sld
\item{{\bf Returns} - 
Возвращает X-составляющую скорости объекта. Ось абсцисс направлена слева направо.
 \bl 
 Для юнитов, способных мгновенно менять свою скорость, возвращается значение перемещения за последний тик. 
}%end item
\end{itemize}
}%end item
\divideents{getSpeedY}
\item{\vskip -1.9ex 
\membername{getSpeedY}
{\tt public final double {\bf getSpeedY}(  )
}%end signature
\begin{itemize}
\sld
\item{{\bf Returns} - 
Возвращает Y-составляющую скорости объекта. Ось ординат направлена сверху вниз.
 \bl 
 Для юнитов, способных мгновенно менять свою скорость, возвращается значение перемещения за последний тик. 
}%end item
\end{itemize}
}%end item
\divideents{getX}
\item{\vskip -1.9ex 
\membername{getX}
{\tt public final double {\bf getX}(  )
}%end signature
\begin{itemize}
\sld
\item{{\bf Returns} - 
Возвращает X-координату центра объекта. Ось абсцисс направлена слева направо. 
}%end item
\end{itemize}
}%end item
\divideents{getY}
\item{\vskip -1.9ex 
\membername{getY}
{\tt public final double {\bf getY}(  )
}%end signature
\begin{itemize}
\sld
\item{{\bf Returns} - 
Возвращает Y-координату центра объекта. Ось ординат направлена сверху вниз. 
}%end item
\end{itemize}
}%end item
\end{itemize}
}}
}
}
\startsection{Class}{MinionType}{l12}{%
{\small Тип приспешника.}
\vskip .1in 
\startsubsubsection{Declaration}{
\fbox{\vbox{
\hbox{\vbox{\small public final 
class 
MinionType}}
\noindent\hbox{\vbox{{\bf extends} Enum}}
}}}
\startsubsubsection{Fields}{
\begin{itemize}
\item{
public static final MinionType ORC\_WOODCUTTER\begin{itemize}\item{\vskip -.9ex Боец ближнего боя и, по совместительству, мастер на все руки. Помогает волшебнику в хозяйстве.
 \bl 
 Не так силён, как воин орков, но для потерявшего бдительность противника может быть весьма опасен.}\end{itemize}
}
\item{
public static final MinionType FETISH\_BLOWDART\begin{itemize}\item{\vskip -.9ex Магическое создание, поражающее противников хозяина острыми дротиками. В мирное время занимается охотой.}\end{itemize}
}
\end{itemize}
}
\hide{inherited}{
\startsubsubsection{Methods inherited from class {\tt Enum}}{
\par{\small 
\refdefined{l24}\vskip -2em
\begin{itemize}
\item{\vskip -1.9ex 
\membername{clone}
{\tt protected final Object {\bf clone}(  )
}%end signature
}%end item
\divideents{compareTo}
\item{\vskip -1.9ex 
\membername{compareTo}
{\tt public final int {\bf compareTo}( {\tt Enum } {\bf arg0} )
}%end signature
}%end item
\divideents{equals}
\item{\vskip -1.9ex 
\membername{equals}
{\tt public final boolean {\bf equals}( {\tt Object } {\bf arg0} )
}%end signature
}%end item
\divideents{finalize}
\item{\vskip -1.9ex 
\membername{finalize}
{\tt protected final void {\bf finalize}(  )
}%end signature
}%end item
\divideents{getDeclaringClass}
\item{\vskip -1.9ex 
\membername{getDeclaringClass}
{\tt public final Class {\bf getDeclaringClass}(  )
}%end signature
}%end item
\divideents{hashCode}
\item{\vskip -1.9ex 
\membername{hashCode}
{\tt public final int {\bf hashCode}(  )
}%end signature
}%end item
\divideents{name}
\item{\vskip -1.9ex 
\membername{name}
{\tt public final String {\bf name}(  )
}%end signature
}%end item
\divideents{ordinal}
\item{\vskip -1.9ex 
\membername{ordinal}
{\tt public final int {\bf ordinal}(  )
}%end signature
}%end item
\divideents{toString}
\item{\vskip -1.9ex 
\membername{toString}
{\tt public String {\bf toString}(  )
}%end signature
}%end item
\divideents{valueOf}
\item{\vskip -1.9ex 
\membername{valueOf}
{\tt public static Enum {\bf valueOf}( {\tt Class } {\bf arg0},
{\tt String } {\bf arg1} )
}%end signature
}%end item
\end{itemize}
}}
}
}
\startsection{Class}{Move}{l13}{%
{\small Стратегия игрока может управлять волшебником посредством установки свойств объекта данного класса.}
\vskip .1in 
\startsubsubsection{Declaration}{
\fbox{\vbox{
\hbox{\vbox{\small public 
class 
Move}}
\noindent\hbox{\vbox{{\bf extends} Object}}
}}}

% Removed by DocsPostProcessor:
% \startsubsubsection{Constructors}{
% \vskip -2em
% \begin{itemize}
% \item{\vskip -1.9ex 
% \membername{Move}
% {\tt public {\bf Move}(  )
% \label{l300}\label{l301}}%end signature
% }%end item
% \end{itemize}
% }
% \\ Removed by DocsPostProcessor.

\startsubsubsection{Methods}{
\vskip -2em
\begin{itemize}
\item{\vskip -1.9ex 
\membername{getAction}
{\tt public ActionType {\bf getAction}(  )
\label{l302}\label{l303}}%end signature
\begin{itemize}
\sld
\item{{\bf Returns} - 
Возвращает текущее действие волшебника. 
}%end item
\end{itemize}
}%end item
\divideents{getCastAngle}
\item{\vskip -1.9ex 
\membername{getCastAngle}
{\tt public double {\bf getCastAngle}(  )
\label{l304}\label{l305}}%end signature
\begin{itemize}
\sld
\item{{\bf Returns} - 
Возвращает текущий угол полёта магического снаряда. 
}%end item
\end{itemize}
}%end item
\divideents{getMaxCastDistance}
\item{\vskip -1.9ex 
\membername{getMaxCastDistance}
{\tt public double {\bf getMaxCastDistance}(  )
\label{l306}\label{l307}}%end signature
\begin{itemize}
\sld
\item{{\bf Returns} - 
Возвращает текущую установку для дальней границы боевого применения магического снаряда. 
}%end item
\end{itemize}
}%end item
\divideents{getMessages}
\item{\vskip -1.9ex 
\membername{getMessages}
{\tt public Message[] {\bf getMessages}(  )
\label{l308}\label{l309}}%end signature
\begin{itemize}
\sld
\item{{\bf Returns} - 
Возвращает текущие сообщения для волшебников дружественной фракции. 
}%end item
\end{itemize}
}%end item
\divideents{getMinCastDistance}
\item{\vskip -1.9ex 
\membername{getMinCastDistance}
{\tt public double {\bf getMinCastDistance}(  )
\label{l310}\label{l311}}%end signature
\begin{itemize}
\sld
\item{{\bf Returns} - 
Возвращает текущую установку для ближней границы боевого применения магического снаряда. 
}%end item
\end{itemize}
}%end item
\divideents{getSkillToLearn}
\item{\vskip -1.9ex 
\membername{getSkillToLearn}
{\tt public SkillType {\bf getSkillToLearn}(  )
\label{l312}\label{l313}}%end signature
\begin{itemize}
\sld
\item{{\bf Returns} - 
Возвращает выбранное для изучения умение. 
}%end item
\end{itemize}
}%end item
\divideents{getSpeed}
\item{\vskip -1.9ex 
\membername{getSpeed}
{\tt public double {\bf getSpeed}(  )
\label{l314}\label{l315}}%end signature
\begin{itemize}
\sld
\item{{\bf Returns} - 
Возвращает текущую установку скорости перемещения. 
}%end item
\end{itemize}
}%end item
\divideents{getStatusTargetId}
\item{\vskip -1.9ex 
\membername{getStatusTargetId}
{\tt public long {\bf getStatusTargetId}(  )
\label{l316}\label{l317}}%end signature
\begin{itemize}
\sld
\item{{\bf Returns} - 
Возвращает идентификатор текущей цели для применения магического статуса. 
}%end item
\end{itemize}
}%end item
\divideents{getStrafeSpeed}
\item{\vskip -1.9ex 
\membername{getStrafeSpeed}
{\tt public double {\bf getStrafeSpeed}(  )
\label{l318}\label{l319}}%end signature
\begin{itemize}
\sld
\item{{\bf Returns} - 
Возвращает текущую установку скорости перемещения боком. 
}%end item
\end{itemize}
}%end item
\divideents{getTurn}
\item{\vskip -1.9ex 
\membername{getTurn}
{\tt public double {\bf getTurn}(  )
\label{l320}\label{l321}}%end signature
\begin{itemize}
\sld
\item{{\bf Returns} - 
Возвращает текущий угол поворота волшебника. 
}%end item
\end{itemize}
}%end item
\divideents{setAction}
\item{\vskip -1.9ex 
\membername{setAction}
{\tt public void {\bf setAction}( {\tt ActionType } {\bf action} )
\label{l322}\label{l323}}%end signature
\begin{itemize}
\sld
\item{
\sld
{\bf Usage}
  \begin{itemize}\isep
   \item{
Устанавливает действие волшебника.
 \bl 
 Действие может быть проигнорировано игровым симулятором, если у волшебника недостаточно магической энергии для
 его совершения и$/$или волшебник ещё не успел восстановиться после предыдущего действия.
}%end item
  \end{itemize}
}
\end{itemize}
}%end item
\divideents{setCastAngle}
\item{\vskip -1.9ex 
\membername{setCastAngle}
{\tt public void {\bf setCastAngle}( {\tt double } {\bf castAngle} )
\label{l324}\label{l325}}%end signature
\begin{itemize}
\sld
\item{
\sld
{\bf Usage}
  \begin{itemize}\isep
   \item{
Устанавливает угол полёта магического снаряда.
 \bl 
 Угол полёта задаётся в радианах относительно текущего направления волшебника и ограничен интервалом от
 {\tt -game.staffSector $/$ 2.0} до {\tt game.staffSector $/$ 2.0}.
 \bl 
 Значения, выходящие за интервал, будут приведены к ближайшей его границе.
 Положительные значения соответствуют повороту по часовой стрелке.
 \bl 
 Параметр будет проигнорирован игровым симулятором, если действие волшебника не связано с созданием магического
 снаряда.
}%end item
  \end{itemize}
}
\end{itemize}
}%end item
\divideents{setMaxCastDistance}
\item{\vskip -1.9ex 
\membername{setMaxCastDistance}
{\tt public void {\bf setMaxCastDistance}( {\tt double } {\bf maxCastDistance} )
\label{l326}\label{l327}}%end signature
\begin{itemize}
\sld
\item{
\sld
{\bf Usage}
  \begin{itemize}\isep
   \item{
Устанавливает дальнюю границу боевого применения магического снаряда.
 \bl 
 Если расстояние от центра снаряда до точки его появления больше, чем значение данного параметра, то снаряд
 убирается из игрового мира. При этом, снаряд типа {\tt FIREBALL} детонирует.
 \bl 
 Значение параметра по умолчанию заведомо выше максимальной дальности полёта любого типа снарядов в игре.
 \bl 
 Параметр будет проигнорирован игровым симулятором, если действие волшебника не связано с созданием магического
 снаряда.
}%end item
  \end{itemize}
}
\end{itemize}
}%end item
\divideents{setMessages}
\item{\vskip -1.9ex 
\membername{setMessages}
{\tt public void {\bf setMessages}( {\tt Message[]} {\bf messages} )
\label{l328}\label{l329}}%end signature
\begin{itemize}
\sld
\item{
\sld
{\bf Usage}
  \begin{itemize}\isep
   \item{
Устанавливает сообщения для волшебников дружественной фракции.
 \bl 
 Доступно для использования только верховному волшебнику ({\tt wizard.master}). Если используется, количество
 сообщений должно быть строго равно количеству волшебников дружественной фракции (живых или ожидающих возрождения)
 за исключением самого верховного волшебника. Нарушение данных условий может привести к игнорированию параметра
 игровым симулятором или даже к обрыву соединения со стратегией участника.
 \bl 
 Сообщения адресуются в порядке возрастания идентификаторов волшебников. Отдельные сообщения могут быть пустыми
 (равны {\tt null}), если это поддерживается языком программирования, который использует стратегия. В противном
 случае все элементы должны быть корректными сообщениями.
 \bl 
 Игровой симулятор вправе проигнорировать сообщение конкретному волшебнику, если для него в системе уже
 зарегистрировано и ещё им не получено другое сообщение. Если в тик получения сообщения волшебник мёртв, то
 данное сообщение будет удалено из игрового мира и волшебник никогда его не получит.
 \bl 
 Отправка сообщений доступна не во всех режимах игры.
}%end item
  \end{itemize}
}
\end{itemize}
}%end item
\divideents{setMinCastDistance}
\item{\vskip -1.9ex 
\membername{setMinCastDistance}
{\tt public void {\bf setMinCastDistance}( {\tt double } {\bf minCastDistance} )
\label{l330}\label{l331}}%end signature
\begin{itemize}
\sld
\item{
\sld
{\bf Usage}
  \begin{itemize}\isep
   \item{
Устанавливает ближнюю границу боевого применения магического снаряда.
 \bl 
 Если расстояние от центра снаряда до точки его появления меньше, чем значение данного параметра, то боевые
 свойства снаряда игнорируются. Снаряд беспрепятственно проходит сквозь все другие игровые объекты, за исключением
 деревьев.
 \bl 
 Значение параметра по умолчанию равно {\tt 0.0}. Столкновения снаряда и юнита, который его создал,
 игнорируются.
 \bl 
 Параметр будет проигнорирован игровым симулятором, если действие волшебника не связано с созданием магического
 снаряда.
}%end item
  \end{itemize}
}
\end{itemize}
}%end item
\divideents{setSkillToLearn}
\item{\vskip -1.9ex 
\membername{setSkillToLearn}
{\tt public void {\bf setSkillToLearn}( {\tt SkillType } {\bf skillToLearn} )
\label{l332}\label{l333}}%end signature
\begin{itemize}
\sld
\item{
\sld
{\bf Usage}
  \begin{itemize}\isep
   \item{
Задаёт установку изучить указанное умение до начала следующего игрового тика.
 \bl 
 Установка будет проигнорирована игровым симулятором, если текущий уровень волшебника меньше либо равен количеству
 уже изученных умений. Некоторые умения также могут требовать предварительного изучения других умений.
 \bl 
 Изучение умений доступно не во всех режимах игры.
}%end item
  \end{itemize}
}
\end{itemize}
}%end item
\divideents{setSpeed}
\item{\vskip -1.9ex 
\membername{setSpeed}
{\tt public void {\bf setSpeed}( {\tt double } {\bf speed} )
\label{l334}\label{l335}}%end signature
\begin{itemize}
\sld
\item{
\sld
{\bf Usage}
  \begin{itemize}\isep
   \item{
Задаёт установку скорости перемещения на один тик.
 \bl 
 Установка скорости перемещения по умолчанию лежит в интервале от {\tt -game.wizardBackwardSpeed} до
 {\tt game.wizardForwardSpeed}, однако границы интервала могут быть расширены в зависимости от изученных
 волшебником умений, от действия некоторых аур, а также в случае действия статуса {\tt HASTENED}.
 \bl 
 Значения, выходящие за интервал, будут приведены к ближайшей его границе.
 Положительные значения соответствуют движению вперёд.
 \bl 
 Если {\tt hypot(speed $/$ maxSpeed, strafeSpeed $/$ maxStrafeSpeed)} больше {\tt 1.0}, то обе установки скорости
 перемещения ({\tt speed} и {\tt strafeSpeed}) будут поделены игровым симулятором на это значение.
}%end item
  \end{itemize}
}
\end{itemize}
}%end item
\divideents{setStatusTargetId}
\item{\vskip -1.9ex 
\membername{setStatusTargetId}
{\tt public void {\bf setStatusTargetId}( {\tt long } {\bf statusTargetId} )
\label{l336}\label{l337}}%end signature
\begin{itemize}
\sld
\item{
\sld
{\bf Usage}
  \begin{itemize}\isep
   \item{
Устанавливает идентификатор цели для применения магического статуса.
 \bl 
 Допустимыми целями являются только волшебники дружественной фракции. Если волшебник с указанным идентификатором
 не найден, то статус применяется непосредственно к волшебнику, совершающему действие. Относительный угол до цели
 должен лежать в интервале от {\tt -game.staffSector $/$ 2.0} до {\tt game.staffSector $/$ 2.0}, а максимальная
 дистанция ограничена дальностью полёта магического снаряда этого волшебника. Её базовое значение равно
 {\tt game.wizardCastRange}, однако оно может быть увеличено после изучения некоторых умений.
 \bl 
 Значение параметра по умолчанию равно {\tt -1}.
 \bl 
 Параметр будет проигнорирован игровым симулятором, если действие волшебника не связано с применением магического
 статуса.
}%end item
  \end{itemize}
}
\end{itemize}
}%end item
\divideents{setStrafeSpeed}
\item{\vskip -1.9ex 
\membername{setStrafeSpeed}
{\tt public void {\bf setStrafeSpeed}( {\tt double } {\bf strafeSpeed} )
\label{l338}\label{l339}}%end signature
\begin{itemize}
\sld
\item{
\sld
{\bf Usage}
  \begin{itemize}\isep
   \item{
Задаёт установку скорости перемещения боком на один тик.
 \bl 
 Установка скорости перемещения по умолчанию лежит в интервале от {\tt -game.wizardStrafeSpeed} до
 {\tt game.wizardStrafeSpeed}, однако границы интервала могут быть расширены в зависимости от изученных
 волшебником умений, от действия некоторых аур, а также в случае действия статуса {\tt HASTENED}.
 \bl 
 Значения, выходящие за интервал, будут приведены к ближайшей его границе.
 Положительные значения соответствуют движению направо.
 \bl 
 Если {\tt hypot(speed $/$ maxSpeed, strafeSpeed $/$ maxStrafeSpeed)} больше {\tt 1.0}, то обе установки скорости
 перемещения ({\tt speed} и {\tt strafeSpeed}) будут поделены игровым симулятором на это значение.
}%end item
  \end{itemize}
}
\end{itemize}
}%end item
\divideents{setTurn}
\item{\vskip -1.9ex 
\membername{setTurn}
{\tt public void {\bf setTurn}( {\tt double } {\bf turn} )
\label{l340}\label{l341}}%end signature
\begin{itemize}
\sld
\item{
\sld
{\bf Usage}
  \begin{itemize}\isep
   \item{
Устанавливает угол поворота волшебника.
 
% Removed by DocsPostProcessor:
% \textless p$/$\textgreater
% \\ Removed by DocsPostProcessor.
 
 Угол поворота задаётся в радианах относительно текущего направления волшебника и обычно ограничен интервалом от
 {\tt -game.wizardMaxTurnAngle} до {\tt game.wizardMaxTurnAngle}. Если на волшебника действует магический
 статус {\tt HASTENED}, то нижнюю и правую границу интервала необходимо умножить на
 {\tt 1.0 + game.hastenedRotationBonusFactor}.
 \bl 
 Значения, выходящие за интервал, будут приведены к ближайшей его границе.
 Положительные значения соответствуют повороту по часовой стрелке.
}%end item
  \end{itemize}
}
\end{itemize}
}%end item
\end{itemize}
}
\hide{inherited}{
}
}
\startsection{Class}{Player}{l14}{%
{\small Содержит данные о текущем состоянии игрока.}
\vskip .1in 
\startsubsubsection{Declaration}{
\fbox{\vbox{
\hbox{\vbox{\small public 
class 
Player}}
\noindent\hbox{\vbox{{\bf extends} Object}}
}}}

% Removed by DocsPostProcessor:
% \startsubsubsection{Constructors}{
% \vskip -2em
% \begin{itemize}
% \item{\vskip -1.9ex 
% \membername{Player}
% {\tt public {\bf Player}( {\tt long } {\bf id},
% {\tt boolean } {\bf me},
% {\tt String } {\bf name},
% {\tt boolean } {\bf strategyCrashed},
% {\tt int } {\bf score},
% {\tt Faction } {\bf faction} )
% \label{l342}\label{l343}}%end signature
% }%end item
% \end{itemize}
% }
% \\ Removed by DocsPostProcessor.

\startsubsubsection{Methods}{
\vskip -2em
\begin{itemize}
\item{\vskip -1.9ex 
\membername{getFaction}
{\tt public Faction {\bf getFaction}(  )
\label{l344}\label{l345}}%end signature
\begin{itemize}
\sld
\item{{\bf Returns} - 
Возвращает фракцию, к которой принадлежит данный игрок. 
}%end item
\end{itemize}
}%end item
\divideents{getId}
\item{\vskip -1.9ex 
\membername{getId}
{\tt public long {\bf getId}(  )
\label{l346}\label{l347}}%end signature
\begin{itemize}
\sld
\item{{\bf Returns} - 
Возвращает уникальный идентификатор игрока. 
}%end item
\end{itemize}
}%end item
\divideents{getName}
\item{\vskip -1.9ex 
\membername{getName}
{\tt public String {\bf getName}(  )
\label{l348}\label{l349}}%end signature
\begin{itemize}
\sld
\item{{\bf Returns} - 
Возвращает имя игрока. 
}%end item
\end{itemize}
}%end item
\divideents{getScore}
\item{\vskip -1.9ex 
\membername{getScore}
{\tt public int {\bf getScore}(  )
\label{l350}\label{l351}}%end signature
\begin{itemize}
\sld
\item{{\bf Returns} - 
Возвращает количество баллов, набранное игроком. 
}%end item
\end{itemize}
}%end item
\divideents{isMe}
\item{\vskip -1.9ex 
\membername{isMe}
{\tt public boolean {\bf isMe}(  )
\label{l352}\label{l353}}%end signature
\begin{itemize}
\sld
\item{{\bf Returns} - 
Возвращает {\tt true} в том и только в том случае, если этот игрок ваш. 
}%end item
\end{itemize}
}%end item
\divideents{isStrategyCrashed}
\item{\vskip -1.9ex 
\membername{isStrategyCrashed}
{\tt public boolean {\bf isStrategyCrashed}(  )
\label{l354}\label{l355}}%end signature
\begin{itemize}
\sld
\item{{\bf Returns} - 
Возвращает специальный флаг --- показатель того, что стратегия игрока <<упала>>.
 Более подробную информацию можно найти в документации к игре. 
}%end item
\end{itemize}
}%end item
\end{itemize}
}
\hide{inherited}{
}
}
\startsection{Class}{Projectile}{l15}{%
{\small Класс, определяющий снаряд. Содержит также все свойства круглого юнита.}
\vskip .1in 
\startsubsubsection{Declaration}{
\fbox{\vbox{
\hbox{\vbox{\small public 
class 
Projectile}}
\noindent\hbox{\vbox{{\bf extends} CircularUnit}}
}}}

% Removed by DocsPostProcessor:
% \startsubsubsection{Constructors}{
% \vskip -2em
% \begin{itemize}
% \item{\vskip -1.9ex 
% \membername{Projectile}
% {\tt public {\bf Projectile}( {\tt long } {\bf id},
% {\tt double } {\bf x},
% {\tt double } {\bf y},
% {\tt double } {\bf speedX},
% {\tt double } {\bf speedY},
% {\tt double } {\bf angle},
% {\tt Faction } {\bf faction},
% {\tt double } {\bf radius},
% {\tt ProjectileType } {\bf type},
% {\tt long } {\bf ownerUnitId},
% {\tt long } {\bf ownerPlayerId} )
% \label{l356}\label{l357}}%end signature
% }%end item
% \end{itemize}
% }
% \\ Removed by DocsPostProcessor.

\startsubsubsection{Methods}{
\vskip -2em
\begin{itemize}
\item{\vskip -1.9ex 
\membername{getOwnerPlayerId}
{\tt public long {\bf getOwnerPlayerId}(  )
\label{l358}\label{l359}}%end signature
\begin{itemize}
\sld
\item{{\bf Returns} - 
Возвращает идентификатор игрока, юнит которого создал данный снаряд или {\tt -1}. 
}%end item
\end{itemize}
}%end item
\divideents{getOwnerUnitId}
\item{\vskip -1.9ex 
\membername{getOwnerUnitId}
{\tt public long {\bf getOwnerUnitId}(  )
\label{l360}\label{l361}}%end signature
\begin{itemize}
\sld
\item{{\bf Returns} - 
Возвращает идентификатор юнита, создавшего данный снаряд. 
}%end item
\end{itemize}
}%end item
\divideents{getType}
\item{\vskip -1.9ex 
\membername{getType}
{\tt public ProjectileType {\bf getType}(  )
\label{l362}\label{l363}}%end signature
\begin{itemize}
\sld
\item{{\bf Returns} - 
Возвращает тип снаряда. 
}%end item
\end{itemize}
}%end item
\end{itemize}
}
\hide{inherited}{
\startsubsubsection{Methods inherited from class {\tt CircularUnit}}{
\par{\small 
\refdefined{l5}\vskip -2em
\begin{itemize}
\item{\vskip -1.9ex 
\membername{getRadius}
{\tt public double {\bf getRadius}(  )
}%end signature
\begin{itemize}
\sld
\item{{\bf Returns} - 
Возвращает радиус объекта. 
}%end item
\end{itemize}
}%end item
\end{itemize}
}}
\startsubsubsection{Methods inherited from class {\tt Unit}}{
\par{\small 
\refdefined{l21}\vskip -2em
\begin{itemize}
\item{\vskip -1.9ex 
\membername{getAngle}
{\tt public final double {\bf getAngle}(  )
}%end signature
\begin{itemize}
\sld
\item{{\bf Returns} - 
Возвращает угол поворота объекта в радианах. Нулевой угол соответствует направлению оси абсцисс.
 Положительные значения соответствуют повороту по часовой стрелке. 
}%end item
\end{itemize}
}%end item
\divideents{getAngleTo}
\item{\vskip -1.9ex 
\membername{getAngleTo}
{\tt public double {\bf getAngleTo}( {\tt double } {\bf x},
{\tt double } {\bf y} )
}%end signature
\begin{itemize}
\sld
\item{
\sld
{\bf Parameters}
\sld\isep
  \begin{itemize}
\sld\isep
   \item{
\sld
{\tt x} - X-координата точки.}
   \item{
\sld
{\tt y} - Y-координата точки.}
  \end{itemize}
}%end item
\item{{\bf Returns} - 
Возвращает ориентированный угол [{\tt -PI}, {\tt PI}] между направлением
 данного объекта и вектором из центра данного объекта к указанной точке. 
}%end item
\end{itemize}
}%end item
\divideents{getAngleTo}
\item{\vskip -1.9ex 
\membername{getAngleTo}
{\tt public double {\bf getAngleTo}( {\tt Unit } {\bf unit} )
}%end signature
\begin{itemize}
\sld
\item{
\sld
{\bf Parameters}
\sld\isep
  \begin{itemize}
\sld\isep
   \item{
\sld
{\tt unit} - Объект, к центру которого необходимо определить угол.}
  \end{itemize}
}%end item
\item{{\bf Returns} - 
Возвращает ориентированный угол [{\tt -PI}, {\tt PI}] между направлением
 данного объекта и вектором из центра данного объекта к центру указанного объекта. 
}%end item
\end{itemize}
}%end item
\divideents{getDistanceTo}
\item{\vskip -1.9ex 
\membername{getDistanceTo}
{\tt public double {\bf getDistanceTo}( {\tt double } {\bf x},
{\tt double } {\bf y} )
}%end signature
\begin{itemize}
\sld
\item{
\sld
{\bf Parameters}
\sld\isep
  \begin{itemize}
\sld\isep
   \item{
\sld
{\tt x} - X-координата точки.}
   \item{
\sld
{\tt y} - Y-координата точки.}
  \end{itemize}
}%end item
\item{{\bf Returns} - 
Возвращает расстояние до точки от центра данного объекта. 
}%end item
\end{itemize}
}%end item
\divideents{getDistanceTo}
\item{\vskip -1.9ex 
\membername{getDistanceTo}
{\tt public double {\bf getDistanceTo}( {\tt Unit } {\bf unit} )
}%end signature
\begin{itemize}
\sld
\item{
\sld
{\bf Parameters}
\sld\isep
  \begin{itemize}
\sld\isep
   \item{
\sld
{\tt unit} - Объект, до центра которого необходимо определить расстояние.}
  \end{itemize}
}%end item
\item{{\bf Returns} - 
Возвращает расстояние от центра данного объекта до центра указанного объекта. 
}%end item
\end{itemize}
}%end item
\divideents{getFaction}
\item{\vskip -1.9ex 
\membername{getFaction}
{\tt public Faction {\bf getFaction}(  )
}%end signature
\begin{itemize}
\sld
\item{{\bf Returns} - 
Возвращает фракцию, к которой принадлежит данный юнит. 
}%end item
\end{itemize}
}%end item
\divideents{getId}
\item{\vskip -1.9ex 
\membername{getId}
{\tt public long {\bf getId}(  )
}%end signature
\begin{itemize}
\sld
\item{{\bf Returns} - 
Возвращает уникальный идентификатор объекта. 
}%end item
\end{itemize}
}%end item
\divideents{getSpeedX}
\item{\vskip -1.9ex 
\membername{getSpeedX}
{\tt public final double {\bf getSpeedX}(  )
}%end signature
\begin{itemize}
\sld
\item{{\bf Returns} - 
Возвращает X-составляющую скорости объекта. Ось абсцисс направлена слева направо.
 \bl 
 Для юнитов, способных мгновенно менять свою скорость, возвращается значение перемещения за последний тик. 
}%end item
\end{itemize}
}%end item
\divideents{getSpeedY}
\item{\vskip -1.9ex 
\membername{getSpeedY}
{\tt public final double {\bf getSpeedY}(  )
}%end signature
\begin{itemize}
\sld
\item{{\bf Returns} - 
Возвращает Y-составляющую скорости объекта. Ось ординат направлена сверху вниз.
 \bl 
 Для юнитов, способных мгновенно менять свою скорость, возвращается значение перемещения за последний тик. 
}%end item
\end{itemize}
}%end item
\divideents{getX}
\item{\vskip -1.9ex 
\membername{getX}
{\tt public final double {\bf getX}(  )
}%end signature
\begin{itemize}
\sld
\item{{\bf Returns} - 
Возвращает X-координату центра объекта. Ось абсцисс направлена слева направо. 
}%end item
\end{itemize}
}%end item
\divideents{getY}
\item{\vskip -1.9ex 
\membername{getY}
{\tt public final double {\bf getY}(  )
}%end signature
\begin{itemize}
\sld
\item{{\bf Returns} - 
Возвращает Y-координату центра объекта. Ось ординат направлена сверху вниз. 
}%end item
\end{itemize}
}%end item
\end{itemize}
}}
}
}
\startsection{Class}{ProjectileType}{l16}{%
{\small Тип снаряда.}
\vskip .1in 
\startsubsubsection{Declaration}{
\fbox{\vbox{
\hbox{\vbox{\small public final 
class 
ProjectileType}}
\noindent\hbox{\vbox{{\bf extends} Enum}}
}}}
\startsubsubsection{Fields}{
\begin{itemize}
\item{
public static final ProjectileType MAGIC\_MISSILE\begin{itemize}\item{\vskip -.9ex Магическая ракета. Небольшой сгусток чистой энергии, который наносит урон при прямом попадании.}\end{itemize}
}
\item{
public static final ProjectileType FROST\_BOLT\begin{itemize}\item{\vskip -.9ex Ледяная стрела. Наносит урон при прямом попадании, а также замораживает цель на {\tt game.frozenDurationTicks}
 тиков. Замороженная цель не может перемещаться и совершать какие-либо действия.}\end{itemize}
}
\item{
public static final ProjectileType FIREBALL\begin{itemize}\item{\vskip -.9ex Огненный шар. Взрывается при достижении максимальной дальности полёта или при столкновении с живым объектом.
 Наносит урон всем живым объектам, если расстояние от центра шара до центра объекта не превышает
 {\tt game.fireballExplosionMinDamageRange + livingUnit.radius}, а также поджигает их ({\tt BURNING}).
 Мгновенный урон уменьшается по мере удаления от эпицентра взрыва.}\end{itemize}
}
\item{
public static final ProjectileType DART\begin{itemize}\item{\vskip -.9ex Дротик. Заострённый предмет, летящий на большой скорости. Наносит урон при прямом попадании.}\end{itemize}
}
\end{itemize}
}
\hide{inherited}{
\startsubsubsection{Methods inherited from class {\tt Enum}}{
\par{\small 
\refdefined{l24}\vskip -2em
\begin{itemize}
\item{\vskip -1.9ex 
\membername{clone}
{\tt protected final Object {\bf clone}(  )
}%end signature
}%end item
\divideents{compareTo}
\item{\vskip -1.9ex 
\membername{compareTo}
{\tt public final int {\bf compareTo}( {\tt Enum } {\bf arg0} )
}%end signature
}%end item
\divideents{equals}
\item{\vskip -1.9ex 
\membername{equals}
{\tt public final boolean {\bf equals}( {\tt Object } {\bf arg0} )
}%end signature
}%end item
\divideents{finalize}
\item{\vskip -1.9ex 
\membername{finalize}
{\tt protected final void {\bf finalize}(  )
}%end signature
}%end item
\divideents{getDeclaringClass}
\item{\vskip -1.9ex 
\membername{getDeclaringClass}
{\tt public final Class {\bf getDeclaringClass}(  )
}%end signature
}%end item
\divideents{hashCode}
\item{\vskip -1.9ex 
\membername{hashCode}
{\tt public final int {\bf hashCode}(  )
}%end signature
}%end item
\divideents{name}
\item{\vskip -1.9ex 
\membername{name}
{\tt public final String {\bf name}(  )
}%end signature
}%end item
\divideents{ordinal}
\item{\vskip -1.9ex 
\membername{ordinal}
{\tt public final int {\bf ordinal}(  )
}%end signature
}%end item
\divideents{toString}
\item{\vskip -1.9ex 
\membername{toString}
{\tt public String {\bf toString}(  )
}%end signature
}%end item
\divideents{valueOf}
\item{\vskip -1.9ex 
\membername{valueOf}
{\tt public static Enum {\bf valueOf}( {\tt Class } {\bf arg0},
{\tt String } {\bf arg1} )
}%end signature
}%end item
\end{itemize}
}}
}
}
\startsection{Class}{SkillType}{l17}{%
{\small Тип умения. Изучение умений может быть доступно не во всех режимах игры (смотрите документацию к
 {\tt game.skillsEnabled}).
 \bl 
 Умения делятся на три категории: активные, пассивные и ауры.
 \begin{itemize}
 \item{\vskip -.8ex 
 Активные умения наделяют волшебника способностью использовать определённое действие, недоступное ранее.
 \textless $/$li\textgreater 
 }
\item{\vskip -.8ex 
 Пассивные умения действуют постоянно, улучшая одну из характеристик волшебника на некоторое значение. При наличии
 нескольких пассивных умений, влияющих на одну характеристику, учитывается только то, которое даёт максимальный
 эффект.
 \textless $/$li\textgreater 
 }
\item{\vskip -.8ex 
 Ауры действуют постоянно, улучшая на некоторое значение одну из характеристик самого волшебника, а также всех союзных
 волшебников на расстоянии, не превышающем {\tt game.auraSkillRange}. При наличии нескольких аур, влияющих на одну
 характеристику, учитывается только та, которая даёт максимальный эффект.
 \textless $/$li\textgreater 
 }
\end{itemize}
}
\vskip .1in 
\startsubsubsection{Declaration}{
\fbox{\vbox{
\hbox{\vbox{\small public final 
class 
SkillType}}
\noindent\hbox{\vbox{{\bf extends} Enum}}
}}}
\startsubsubsection{Fields}{
\begin{itemize}
\item{
public static final SkillType RANGE\_BONUS\_PASSIVE\_1\begin{itemize}\item{\vskip -.9ex Пассивное умение. Увеличивает максимально возможную дальность полёта магического снаряда, а также дальность
 применения магических статусов на {\tt game.rangeBonusPerSkillLevel}.}\end{itemize}
}
\item{
public static final SkillType RANGE\_BONUS\_AURA\_1\begin{itemize}\item{\vskip -.9ex Аура. Увеличивает максимально возможную дальность полёта магического снаряда, а также дальность
 применения магических статусов на {\tt game.rangeBonusPerSkillLevel}.
 \bl 
 Требуется предварительно изучить умение {\tt RANGE\_BONUS\_PASSIVE\_1}.}\end{itemize}
}
\item{
public static final SkillType RANGE\_BONUS\_PASSIVE\_2\begin{itemize}\item{\vskip -.9ex Пассивное умение. Увеличивает максимально возможную дальность полёта магического снаряда, а также дальность
 применения магических статусов на {\tt 2.0 * game.rangeBonusPerSkillLevel}.
 \bl 
 Требуется предварительно изучить умение {\tt RANGE\_BONUS\_AURA\_1}.}\end{itemize}
}
\item{
public static final SkillType RANGE\_BONUS\_AURA\_2\begin{itemize}\item{\vskip -.9ex Аура. Увеличивает максимально возможную дальность полёта магического снаряда, а также дальность
 применения магических статусов на {\tt 2.0 * game.rangeBonusPerSkillLevel}.
 \bl 
 Требуется предварительно изучить умение {\tt RANGE\_BONUS\_PASSIVE\_2}.}\end{itemize}
}
\item{
public static final SkillType ADVANCED\_MAGIC\_MISSILE\begin{itemize}\item{\vskip -.9ex Пассивное умение. Убирает задержку на примение действия {\tt MAGIC\_MISSILE}.
 Общая задержка на действия волшебника {\tt game.wizardActionCooldownTicks} всё ещё применяется.
 \bl 
 Требуется предварительно изучить умение {\tt RANGE\_BONUS\_AURA\_2}.}\end{itemize}
}
\item{
public static final SkillType MAGICAL\_DAMAGE\_BONUS\_PASSIVE\_1\begin{itemize}\item{\vskip -.9ex Пассивное умение. Увеличивает урон, наносимый при прямом попадании магического снаряда, на
 {\tt game.magicalDamageBonusPerSkillLevel}.}\end{itemize}
}
\item{
public static final SkillType MAGICAL\_DAMAGE\_BONUS\_AURA\_1\begin{itemize}\item{\vskip -.9ex Аура. Увеличивает урон, наносимый при прямом попадании магического снаряда, на
 {\tt game.magicalDamageBonusPerSkillLevel}.
 \bl 
 Требуется предварительно изучить умение {\tt MAGICAL\_DAMAGE\_BONUS\_PASSIVE\_1}.}\end{itemize}
}
\item{
public static final SkillType MAGICAL\_DAMAGE\_BONUS\_PASSIVE\_2\begin{itemize}\item{\vskip -.9ex Пассивное умение. Увеличивает урон, наносимый при прямом попадании магического снаряда, на
 {\tt 2.0 * game.magicalDamageBonusPerSkillLevel}.
 \bl 
 Требуется предварительно изучить умение {\tt MAGICAL\_DAMAGE\_BONUS\_AURA\_1}.}\end{itemize}
}
\item{
public static final SkillType MAGICAL\_DAMAGE\_BONUS\_AURA\_2\begin{itemize}\item{\vskip -.9ex Аура. Увеличивает урон, наносимый при прямом попадании магического снаряда, на
 {\tt 2.0 * game.magicalDamageBonusPerSkillLevel}.
 \bl 
 Требуется предварительно изучить умение {\tt MAGICAL\_DAMAGE\_BONUS\_PASSIVE\_2}.}\end{itemize}
}
\item{
public static final SkillType FROST\_BOLT\begin{itemize}\item{\vskip -.9ex Активное умение. Позволяет волшебнику использовать действие {\tt FROST\_BOLT}.
 \bl 
 Требуется предварительно изучить умение {\tt MAGICAL\_DAMAGE\_BONUS\_AURA\_2}.}\end{itemize}
}
\item{
public static final SkillType STAFF\_DAMAGE\_BONUS\_PASSIVE\_1\begin{itemize}\item{\vskip -.9ex Пассивное умение. Увеличивает урон, наносимый при ударе посохом, на {\tt game.staffDamageBonusPerSkillLevel}.}\end{itemize}
}
\item{
public static final SkillType STAFF\_DAMAGE\_BONUS\_AURA\_1\begin{itemize}\item{\vskip -.9ex Аура. Увеличивает урон, наносимый при ударе посохом, на {\tt game.staffDamageBonusPerSkillLevel}.
 \bl 
 Требуется предварительно изучить умение {\tt STAFF\_DAMAGE\_BONUS\_PASSIVE\_1}.}\end{itemize}
}
\item{
public static final SkillType STAFF\_DAMAGE\_BONUS\_PASSIVE\_2\begin{itemize}\item{\vskip -.9ex Пассивное умение. Увеличивает урон, наносимый при ударе посохом, на
 {\tt 2.0 * game.staffDamageBonusPerSkillLevel}.
 \bl 
 Требуется предварительно изучить умение {\tt STAFF\_DAMAGE\_BONUS\_AURA\_1}.}\end{itemize}
}
\item{
public static final SkillType STAFF\_DAMAGE\_BONUS\_AURA\_2\begin{itemize}\item{\vskip -.9ex Аура. Увеличивает урон, наносимый при ударе посохом, на {\tt 2.0 * game.staffDamageBonusPerSkillLevel}.
 \bl 
 Требуется предварительно изучить умение {\tt STAFF\_DAMAGE\_BONUS\_PASSIVE\_2}.}\end{itemize}
}
\item{
public static final SkillType FIREBALL\begin{itemize}\item{\vskip -.9ex Активное умение. Позволяет волшебнику использовать действие {\tt FIREBALL}.
 \bl 
 Требуется предварительно изучить умение {\tt STAFF\_DAMAGE\_BONUS\_AURA\_2}.}\end{itemize}
}
\item{
public static final SkillType MOVEMENT\_BONUS\_FACTOR\_PASSIVE\_1\begin{itemize}\item{\vskip -.9ex Пассивное умение. Увеличивает скорость перемещения в {\tt 1.0 + game.movementBonusFactorPerSkillLevel} раз.
 \bl 
 Увеличение скорости от изучения пассивных умений и увеличение скорости в результате действия аур аддитивны.
 Таким образом, умения {\tt MOVEMENT\_BONUS\_FACTOR\_PASSIVE\_2} и {\tt MOVEMENT\_BONUS\_FACTOR\_AURA\_2} суммарно
 увеличат скорость перемещения в {\tt 1.0 + 4.0 * game.movementBonusFactorPerSkillLevel} раз.}\end{itemize}
}
\item{
public static final SkillType MOVEMENT\_BONUS\_FACTOR\_AURA\_1\begin{itemize}\item{\vskip -.9ex Аура. Увеличивает скорость перемещения в {\tt 1.0 + game.movementBonusFactorPerSkillLevel} раз.
 \bl 
 Требуется предварительно изучить умение {\tt MOVEMENT\_BONUS\_FACTOR\_PASSIVE\_1}.}\end{itemize}
}
\item{
public static final SkillType MOVEMENT\_BONUS\_FACTOR\_PASSIVE\_2\begin{itemize}\item{\vskip -.9ex Пассивное умение. Увеличивает скорость перемещения в {\tt 1.0 + 2.0 * game.movementBonusFactorPerSkillLevel}
 раз.
 \bl 
 Требуется предварительно изучить умение {\tt MOVEMENT\_BONUS\_FACTOR\_AURA\_1}.}\end{itemize}
}
\item{
public static final SkillType MOVEMENT\_BONUS\_FACTOR\_AURA\_2\begin{itemize}\item{\vskip -.9ex Аура. Увеличивает скорость перемещения в {\tt 1.0 + 2.0 * game.movementBonusFactorPerSkillLevel} раз.
 \bl 
 Требуется предварительно изучить умение {\tt MOVEMENT\_BONUS\_FACTOR\_PASSIVE\_2}.}\end{itemize}
}
\item{
public static final SkillType HASTE\begin{itemize}\item{\vskip -.9ex Активное умение. Позволяет волшебнику использовать действие {\tt HASTE}.
 \bl 
 Требуется предварительно изучить умение {\tt MOVEMENT\_BONUS\_FACTOR\_AURA\_2}.}\end{itemize}
}
\item{
public static final SkillType MAGICAL\_DAMAGE\_ABSORPTION\_PASSIVE\_1\begin{itemize}\item{\vskip -.9ex Пассивное умение. Уменьшает урон, получаемый при прямом попадании магического снаряда, на
 {\tt game.magicalDamageAbsorptionPerSkillLevel}.}\end{itemize}
}
\item{
public static final SkillType MAGICAL\_DAMAGE\_ABSORPTION\_AURA\_1\begin{itemize}\item{\vskip -.9ex Аура. Уменьшает урон, получаемый при прямом попадании магического снаряда, на
 {\tt game.magicalDamageAbsorptionPerSkillLevel}.
 \bl 
 Требуется предварительно изучить умение {\tt MAGICAL\_DAMAGE\_ABSORPTION\_PASSIVE\_1}.}\end{itemize}
}
\item{
public static final SkillType MAGICAL\_DAMAGE\_ABSORPTION\_PASSIVE\_2\begin{itemize}\item{\vskip -.9ex Пассивное умение. Уменьшает урон, получаемый при прямом попадании магического снаряда, на
 {\tt 2.0 * game.magicalDamageAbsorptionPerSkillLevel}.
 \bl 
 Требуется предварительно изучить умение {\tt MAGICAL\_DAMAGE\_ABSORPTION\_AURA\_1}.}\end{itemize}
}
\item{
public static final SkillType MAGICAL\_DAMAGE\_ABSORPTION\_AURA\_2\begin{itemize}\item{\vskip -.9ex Аура. Уменьшает урон, получаемый при прямом попадании магического снаряда, на
 {\tt 2.0 * game.magicalDamageAbsorptionPerSkillLevel}.
 \bl 
 Требуется предварительно изучить умение {\tt MAGICAL\_DAMAGE\_ABSORPTION\_PASSIVE\_2}.}\end{itemize}
}
\item{
public static final SkillType SHIELD\begin{itemize}\item{\vskip -.9ex Активное умение. Позволяет волшебнику использовать действие {\tt SHIELD}.
 \bl 
 Требуется предварительно изучить умение {\tt MAGICAL\_DAMAGE\_ABSORPTION\_AURA\_2}.}\end{itemize}
}
\end{itemize}
}
\hide{inherited}{
\startsubsubsection{Methods inherited from class {\tt Enum}}{
\par{\small 
\refdefined{l24}\vskip -2em
\begin{itemize}
\item{\vskip -1.9ex 
\membername{clone}
{\tt protected final Object {\bf clone}(  )
}%end signature
}%end item
\divideents{compareTo}
\item{\vskip -1.9ex 
\membername{compareTo}
{\tt public final int {\bf compareTo}( {\tt Enum } {\bf arg0} )
}%end signature
}%end item
\divideents{equals}
\item{\vskip -1.9ex 
\membername{equals}
{\tt public final boolean {\bf equals}( {\tt Object } {\bf arg0} )
}%end signature
}%end item
\divideents{finalize}
\item{\vskip -1.9ex 
\membername{finalize}
{\tt protected final void {\bf finalize}(  )
}%end signature
}%end item
\divideents{getDeclaringClass}
\item{\vskip -1.9ex 
\membername{getDeclaringClass}
{\tt public final Class {\bf getDeclaringClass}(  )
}%end signature
}%end item
\divideents{hashCode}
\item{\vskip -1.9ex 
\membername{hashCode}
{\tt public final int {\bf hashCode}(  )
}%end signature
}%end item
\divideents{name}
\item{\vskip -1.9ex 
\membername{name}
{\tt public final String {\bf name}(  )
}%end signature
}%end item
\divideents{ordinal}
\item{\vskip -1.9ex 
\membername{ordinal}
{\tt public final int {\bf ordinal}(  )
}%end signature
}%end item
\divideents{toString}
\item{\vskip -1.9ex 
\membername{toString}
{\tt public String {\bf toString}(  )
}%end signature
}%end item
\divideents{valueOf}
\item{\vskip -1.9ex 
\membername{valueOf}
{\tt public static Enum {\bf valueOf}( {\tt Class } {\bf arg0},
{\tt String } {\bf arg1} )
}%end signature
}%end item
\end{itemize}
}}
}
}
\startsection{Class}{Status}{l18}{%
{\small Магический статус, влияющий на живого юнита.}
\vskip .1in 
\startsubsubsection{Declaration}{
\fbox{\vbox{
\hbox{\vbox{\small public 
class 
Status}}
\noindent\hbox{\vbox{{\bf extends} Object}}
}}}

% Removed by DocsPostProcessor:
% \startsubsubsection{Constructors}{
% \vskip -2em
% \begin{itemize}
% \item{\vskip -1.9ex 
% \membername{Status}
% {\tt public {\bf Status}( {\tt long } {\bf id},
% {\tt StatusType } {\bf type},
% {\tt long } {\bf wizardId},
% {\tt long } {\bf playerId},
% {\tt int } {\bf remainingDurationTicks} )
% \label{l364}\label{l365}}%end signature
% }%end item
% \end{itemize}
% }
% \\ Removed by DocsPostProcessor.

\startsubsubsection{Methods}{
\vskip -2em
\begin{itemize}
\item{\vskip -1.9ex 
\membername{getId}
{\tt public long {\bf getId}(  )
\label{l366}\label{l367}}%end signature
\begin{itemize}
\sld
\item{{\bf Returns} - 
Возвращает уникальный идентификатор статуса. 
}%end item
\end{itemize}
}%end item
\divideents{getPlayerId}
\item{\vskip -1.9ex 
\membername{getPlayerId}
{\tt public long {\bf getPlayerId}(  )
\label{l368}\label{l369}}%end signature
\begin{itemize}
\sld
\item{{\bf Returns} - 
Возвращает идентификатор игрока, волшебник которого наложил данный статус, или {\tt -1}. 
}%end item
\end{itemize}
}%end item
\divideents{getRemainingDurationTicks}
\item{\vskip -1.9ex 
\membername{getRemainingDurationTicks}
{\tt public int {\bf getRemainingDurationTicks}(  )
\label{l370}\label{l371}}%end signature
\begin{itemize}
\sld
\item{{\bf Returns} - 
Возвращает оставшуюся длительность действия статуса. 
}%end item
\end{itemize}
}%end item
\divideents{getType}
\item{\vskip -1.9ex 
\membername{getType}
{\tt public StatusType {\bf getType}(  )
\label{l372}\label{l373}}%end signature
\begin{itemize}
\sld
\item{{\bf Returns} - 
Возвращает тип магического статуса. 
}%end item
\end{itemize}
}%end item
\divideents{getWizardId}
\item{\vskip -1.9ex 
\membername{getWizardId}
{\tt public long {\bf getWizardId}(  )
\label{l374}\label{l375}}%end signature
\begin{itemize}
\sld
\item{{\bf Returns} - 
Возвращает идентификатор волшебника, наложившего данный статус, или {\tt -1}. 
}%end item
\end{itemize}
}%end item
\end{itemize}
}
\hide{inherited}{
}
}
\startsection{Class}{StatusType}{l19}{%
{\small Тип магического статуса, влияющего на живого юнита.}
\vskip .1in 
\startsubsubsection{Declaration}{
\fbox{\vbox{
\hbox{\vbox{\small public final 
class 
StatusType}}
\noindent\hbox{\vbox{{\bf extends} Enum}}
}}}
\startsubsubsection{Fields}{
\begin{itemize}
\item{
public static final StatusType BURNING\begin{itemize}\item{\vskip -.9ex Юнит горит. Каждый тик ему наносится некоторый урон.}\end{itemize}
}
\item{
public static final StatusType EMPOWERED\begin{itemize}\item{\vskip -.9ex Юнит наносит больше урона, чем обычно. Не применимо к урону, растянутому во времени.}\end{itemize}
}
\item{
public static final StatusType FROZEN\begin{itemize}\item{\vskip -.9ex Юнит заморожен. Он не может перемещаться и выполнять какие-либо действия.}\end{itemize}
}
\item{
public static final StatusType HASTENED\begin{itemize}\item{\vskip -.9ex Скорость поворота и перемещения юнита увеличена.}\end{itemize}
}
\item{
public static final StatusType SHIELDED\begin{itemize}\item{\vskip -.9ex Юнит получает меньше урона, чем обычно. Не применимо к урону, растянутому во времени.}\end{itemize}
}
\end{itemize}
}
\hide{inherited}{
\startsubsubsection{Methods inherited from class {\tt Enum}}{
\par{\small 
\refdefined{l24}\vskip -2em
\begin{itemize}
\item{\vskip -1.9ex 
\membername{clone}
{\tt protected final Object {\bf clone}(  )
}%end signature
}%end item
\divideents{compareTo}
\item{\vskip -1.9ex 
\membername{compareTo}
{\tt public final int {\bf compareTo}( {\tt Enum } {\bf arg0} )
}%end signature
}%end item
\divideents{equals}
\item{\vskip -1.9ex 
\membername{equals}
{\tt public final boolean {\bf equals}( {\tt Object } {\bf arg0} )
}%end signature
}%end item
\divideents{finalize}
\item{\vskip -1.9ex 
\membername{finalize}
{\tt protected final void {\bf finalize}(  )
}%end signature
}%end item
\divideents{getDeclaringClass}
\item{\vskip -1.9ex 
\membername{getDeclaringClass}
{\tt public final Class {\bf getDeclaringClass}(  )
}%end signature
}%end item
\divideents{hashCode}
\item{\vskip -1.9ex 
\membername{hashCode}
{\tt public final int {\bf hashCode}(  )
}%end signature
}%end item
\divideents{name}
\item{\vskip -1.9ex 
\membername{name}
{\tt public final String {\bf name}(  )
}%end signature
}%end item
\divideents{ordinal}
\item{\vskip -1.9ex 
\membername{ordinal}
{\tt public final int {\bf ordinal}(  )
}%end signature
}%end item
\divideents{toString}
\item{\vskip -1.9ex 
\membername{toString}
{\tt public String {\bf toString}(  )
}%end signature
}%end item
\divideents{valueOf}
\item{\vskip -1.9ex 
\membername{valueOf}
{\tt public static Enum {\bf valueOf}( {\tt Class } {\bf arg0},
{\tt String } {\bf arg1} )
}%end signature
}%end item
\end{itemize}
}}
}
}
\startsection{Class}{Tree}{l20}{%
{\small Класс, определяющий дерево. Содержит также все свойства живого юнита.}
\vskip .1in 
\startsubsubsection{Declaration}{
\fbox{\vbox{
\hbox{\vbox{\small public 
class 
Tree}}
\noindent\hbox{\vbox{{\bf extends} LivingUnit}}
}}}

% Removed by DocsPostProcessor:
% \startsubsubsection{Constructors}{
% \vskip -2em
% \begin{itemize}
% \item{\vskip -1.9ex 
% \membername{Tree}
% {\tt public {\bf Tree}( {\tt long } {\bf id},
% {\tt double } {\bf x},
% {\tt double } {\bf y},
% {\tt double } {\bf speedX},
% {\tt double } {\bf speedY},
% {\tt double } {\bf angle},
% {\tt Faction } {\bf faction},
% {\tt double } {\bf radius},
% {\tt int } {\bf life},
% {\tt int } {\bf maxLife},
% {\tt Status[]} {\bf statuses} )
% \label{l376}\label{l377}}%end signature
% }%end item
% \end{itemize}
% }
% \\ Removed by DocsPostProcessor.

\hide{inherited}{
\startsubsubsection{Methods inherited from class {\tt LivingUnit}}{
\par{\small 
\refdefined{l9}\vskip -2em
\begin{itemize}
\item{\vskip -1.9ex 
\membername{getLife}
{\tt public int {\bf getLife}(  )
}%end signature
\begin{itemize}
\sld
\item{{\bf Returns} - 
Возвращает текущее количество жизненной энергии. 
}%end item
\end{itemize}
}%end item
\divideents{getMaxLife}
\item{\vskip -1.9ex 
\membername{getMaxLife}
{\tt public int {\bf getMaxLife}(  )
}%end signature
\begin{itemize}
\sld
\item{{\bf Returns} - 
Возвращает максимальное количество жизненной энергии. 
}%end item
\end{itemize}
}%end item
\divideents{getStatuses}
\item{\vskip -1.9ex 
\membername{getStatuses}
{\tt public Status[] {\bf getStatuses}(  )
}%end signature
\begin{itemize}
\sld
\item{{\bf Returns} - 
Возвращает магические статусы, влияющие на живого юнита. 
}%end item
\end{itemize}
}%end item
\end{itemize}
}}
\startsubsubsection{Methods inherited from class {\tt CircularUnit}}{
\par{\small 
\refdefined{l5}\vskip -2em
\begin{itemize}
\item{\vskip -1.9ex 
\membername{getRadius}
{\tt public double {\bf getRadius}(  )
}%end signature
\begin{itemize}
\sld
\item{{\bf Returns} - 
Возвращает радиус объекта. 
}%end item
\end{itemize}
}%end item
\end{itemize}
}}
\startsubsubsection{Methods inherited from class {\tt Unit}}{
\par{\small 
\refdefined{l21}\vskip -2em
\begin{itemize}
\item{\vskip -1.9ex 
\membername{getAngle}
{\tt public final double {\bf getAngle}(  )
}%end signature
\begin{itemize}
\sld
\item{{\bf Returns} - 
Возвращает угол поворота объекта в радианах. Нулевой угол соответствует направлению оси абсцисс.
 Положительные значения соответствуют повороту по часовой стрелке. 
}%end item
\end{itemize}
}%end item
\divideents{getAngleTo}
\item{\vskip -1.9ex 
\membername{getAngleTo}
{\tt public double {\bf getAngleTo}( {\tt double } {\bf x},
{\tt double } {\bf y} )
}%end signature
\begin{itemize}
\sld
\item{
\sld
{\bf Parameters}
\sld\isep
  \begin{itemize}
\sld\isep
   \item{
\sld
{\tt x} - X-координата точки.}
   \item{
\sld
{\tt y} - Y-координата точки.}
  \end{itemize}
}%end item
\item{{\bf Returns} - 
Возвращает ориентированный угол [{\tt -PI}, {\tt PI}] между направлением
 данного объекта и вектором из центра данного объекта к указанной точке. 
}%end item
\end{itemize}
}%end item
\divideents{getAngleTo}
\item{\vskip -1.9ex 
\membername{getAngleTo}
{\tt public double {\bf getAngleTo}( {\tt Unit } {\bf unit} )
}%end signature
\begin{itemize}
\sld
\item{
\sld
{\bf Parameters}
\sld\isep
  \begin{itemize}
\sld\isep
   \item{
\sld
{\tt unit} - Объект, к центру которого необходимо определить угол.}
  \end{itemize}
}%end item
\item{{\bf Returns} - 
Возвращает ориентированный угол [{\tt -PI}, {\tt PI}] между направлением
 данного объекта и вектором из центра данного объекта к центру указанного объекта. 
}%end item
\end{itemize}
}%end item
\divideents{getDistanceTo}
\item{\vskip -1.9ex 
\membername{getDistanceTo}
{\tt public double {\bf getDistanceTo}( {\tt double } {\bf x},
{\tt double } {\bf y} )
}%end signature
\begin{itemize}
\sld
\item{
\sld
{\bf Parameters}
\sld\isep
  \begin{itemize}
\sld\isep
   \item{
\sld
{\tt x} - X-координата точки.}
   \item{
\sld
{\tt y} - Y-координата точки.}
  \end{itemize}
}%end item
\item{{\bf Returns} - 
Возвращает расстояние до точки от центра данного объекта. 
}%end item
\end{itemize}
}%end item
\divideents{getDistanceTo}
\item{\vskip -1.9ex 
\membername{getDistanceTo}
{\tt public double {\bf getDistanceTo}( {\tt Unit } {\bf unit} )
}%end signature
\begin{itemize}
\sld
\item{
\sld
{\bf Parameters}
\sld\isep
  \begin{itemize}
\sld\isep
   \item{
\sld
{\tt unit} - Объект, до центра которого необходимо определить расстояние.}
  \end{itemize}
}%end item
\item{{\bf Returns} - 
Возвращает расстояние от центра данного объекта до центра указанного объекта. 
}%end item
\end{itemize}
}%end item
\divideents{getFaction}
\item{\vskip -1.9ex 
\membername{getFaction}
{\tt public Faction {\bf getFaction}(  )
}%end signature
\begin{itemize}
\sld
\item{{\bf Returns} - 
Возвращает фракцию, к которой принадлежит данный юнит. 
}%end item
\end{itemize}
}%end item
\divideents{getId}
\item{\vskip -1.9ex 
\membername{getId}
{\tt public long {\bf getId}(  )
}%end signature
\begin{itemize}
\sld
\item{{\bf Returns} - 
Возвращает уникальный идентификатор объекта. 
}%end item
\end{itemize}
}%end item
\divideents{getSpeedX}
\item{\vskip -1.9ex 
\membername{getSpeedX}
{\tt public final double {\bf getSpeedX}(  )
}%end signature
\begin{itemize}
\sld
\item{{\bf Returns} - 
Возвращает X-составляющую скорости объекта. Ось абсцисс направлена слева направо.
 \bl 
 Для юнитов, способных мгновенно менять свою скорость, возвращается значение перемещения за последний тик. 
}%end item
\end{itemize}
}%end item
\divideents{getSpeedY}
\item{\vskip -1.9ex 
\membername{getSpeedY}
{\tt public final double {\bf getSpeedY}(  )
}%end signature
\begin{itemize}
\sld
\item{{\bf Returns} - 
Возвращает Y-составляющую скорости объекта. Ось ординат направлена сверху вниз.
 \bl 
 Для юнитов, способных мгновенно менять свою скорость, возвращается значение перемещения за последний тик. 
}%end item
\end{itemize}
}%end item
\divideents{getX}
\item{\vskip -1.9ex 
\membername{getX}
{\tt public final double {\bf getX}(  )
}%end signature
\begin{itemize}
\sld
\item{{\bf Returns} - 
Возвращает X-координату центра объекта. Ось абсцисс направлена слева направо. 
}%end item
\end{itemize}
}%end item
\divideents{getY}
\item{\vskip -1.9ex 
\membername{getY}
{\tt public final double {\bf getY}(  )
}%end signature
\begin{itemize}
\sld
\item{{\bf Returns} - 
Возвращает Y-координату центра объекта. Ось ординат направлена сверху вниз. 
}%end item
\end{itemize}
}%end item
\end{itemize}
}}
}
}
\startsection{Class}{Unit}{l21}{%
{\small Базовый класс для определения объектов (<<юнитов>>) на игровом поле.}
\vskip .1in 
\startsubsubsection{Declaration}{
\fbox{\vbox{
\hbox{\vbox{\small public abstract 
class 
Unit}}
\noindent\hbox{\vbox{{\bf extends} Object}}
}}}

% Removed by DocsPostProcessor:
% \startsubsubsection{Constructors}{
% \vskip -2em
% \begin{itemize}
% \item{\vskip -1.9ex 
% \membername{Unit}
% {\tt protected {\bf Unit}( {\tt long } {\bf id},
% {\tt double } {\bf x},
% {\tt double } {\bf y},
% {\tt double } {\bf speedX},
% {\tt double } {\bf speedY},
% {\tt double } {\bf angle},
% {\tt Faction } {\bf faction} )
% \label{l378}\label{l379}}%end signature
% }%end item
% \end{itemize}
% }
% \\ Removed by DocsPostProcessor.

\startsubsubsection{Methods}{
\vskip -2em
\begin{itemize}
\item{\vskip -1.9ex 
\membername{getAngle}
{\tt public final double {\bf getAngle}(  )
\label{l380}\label{l381}}%end signature
\begin{itemize}
\sld
\item{{\bf Returns} - 
Возвращает угол поворота объекта в радианах. Нулевой угол соответствует направлению оси абсцисс.
 Положительные значения соответствуют повороту по часовой стрелке. 
}%end item
\end{itemize}
}%end item
\divideents{getAngleTo}
\item{\vskip -1.9ex 
\membername{getAngleTo}
{\tt public double {\bf getAngleTo}( {\tt double } {\bf x},
{\tt double } {\bf y} )
\label{l382}\label{l383}}%end signature
\begin{itemize}
\sld
\item{
\sld
{\bf Parameters}
\sld\isep
  \begin{itemize}
\sld\isep
   \item{
\sld
{\tt x} - X-координата точки.}
   \item{
\sld
{\tt y} - Y-координата точки.}
  \end{itemize}
}%end item
\item{{\bf Returns} - 
Возвращает ориентированный угол [{\tt -PI}, {\tt PI}] между направлением
 данного объекта и вектором из центра данного объекта к указанной точке. 
}%end item
\end{itemize}
}%end item
\divideents{getAngleTo}
\item{\vskip -1.9ex 
\membername{getAngleTo}
{\tt public double {\bf getAngleTo}( {\tt Unit } {\bf unit} )
\label{l384}\label{l385}}%end signature
\begin{itemize}
\sld
\item{
\sld
{\bf Parameters}
\sld\isep
  \begin{itemize}
\sld\isep
   \item{
\sld
{\tt unit} - Объект, к центру которого необходимо определить угол.}
  \end{itemize}
}%end item
\item{{\bf Returns} - 
Возвращает ориентированный угол [{\tt -PI}, {\tt PI}] между направлением
 данного объекта и вектором из центра данного объекта к центру указанного объекта. 
}%end item
\end{itemize}
}%end item
\divideents{getDistanceTo}
\item{\vskip -1.9ex 
\membername{getDistanceTo}
{\tt public double {\bf getDistanceTo}( {\tt double } {\bf x},
{\tt double } {\bf y} )
\label{l386}\label{l387}}%end signature
\begin{itemize}
\sld
\item{
\sld
{\bf Parameters}
\sld\isep
  \begin{itemize}
\sld\isep
   \item{
\sld
{\tt x} - X-координата точки.}
   \item{
\sld
{\tt y} - Y-координата точки.}
  \end{itemize}
}%end item
\item{{\bf Returns} - 
Возвращает расстояние до точки от центра данного объекта. 
}%end item
\end{itemize}
}%end item
\divideents{getDistanceTo}
\item{\vskip -1.9ex 
\membername{getDistanceTo}
{\tt public double {\bf getDistanceTo}( {\tt Unit } {\bf unit} )
\label{l388}\label{l389}}%end signature
\begin{itemize}
\sld
\item{
\sld
{\bf Parameters}
\sld\isep
  \begin{itemize}
\sld\isep
   \item{
\sld
{\tt unit} - Объект, до центра которого необходимо определить расстояние.}
  \end{itemize}
}%end item
\item{{\bf Returns} - 
Возвращает расстояние от центра данного объекта до центра указанного объекта. 
}%end item
\end{itemize}
}%end item
\divideents{getFaction}
\item{\vskip -1.9ex 
\membername{getFaction}
{\tt public Faction {\bf getFaction}(  )
\label{l390}\label{l391}}%end signature
\begin{itemize}
\sld
\item{{\bf Returns} - 
Возвращает фракцию, к которой принадлежит данный юнит. 
}%end item
\end{itemize}
}%end item
\divideents{getId}
\item{\vskip -1.9ex 
\membername{getId}
{\tt public long {\bf getId}(  )
\label{l392}\label{l393}}%end signature
\begin{itemize}
\sld
\item{{\bf Returns} - 
Возвращает уникальный идентификатор объекта. 
}%end item
\end{itemize}
}%end item
\divideents{getSpeedX}
\item{\vskip -1.9ex 
\membername{getSpeedX}
{\tt public final double {\bf getSpeedX}(  )
\label{l394}\label{l395}}%end signature
\begin{itemize}
\sld
\item{{\bf Returns} - 
Возвращает X-составляющую скорости объекта. Ось абсцисс направлена слева направо.
 \bl 
 Для юнитов, способных мгновенно менять свою скорость, возвращается значение перемещения за последний тик. 
}%end item
\end{itemize}
}%end item
\divideents{getSpeedY}
\item{\vskip -1.9ex 
\membername{getSpeedY}
{\tt public final double {\bf getSpeedY}(  )
\label{l396}\label{l397}}%end signature
\begin{itemize}
\sld
\item{{\bf Returns} - 
Возвращает Y-составляющую скорости объекта. Ось ординат направлена сверху вниз.
 \bl 
 Для юнитов, способных мгновенно менять свою скорость, возвращается значение перемещения за последний тик. 
}%end item
\end{itemize}
}%end item
\divideents{getX}
\item{\vskip -1.9ex 
\membername{getX}
{\tt public final double {\bf getX}(  )
\label{l398}\label{l399}}%end signature
\begin{itemize}
\sld
\item{{\bf Returns} - 
Возвращает X-координату центра объекта. Ось абсцисс направлена слева направо. 
}%end item
\end{itemize}
}%end item
\divideents{getY}
\item{\vskip -1.9ex 
\membername{getY}
{\tt public final double {\bf getY}(  )
\label{l400}\label{l401}}%end signature
\begin{itemize}
\sld
\item{{\bf Returns} - 
Возвращает Y-координату центра объекта. Ось ординат направлена сверху вниз. 
}%end item
\end{itemize}
}%end item
\end{itemize}
}
\hide{inherited}{
}
}
\startsection{Class}{Wizard}{l22}{%
{\small Класс, определяющий волшебника. Содержит также все свойства живого юнита.}
\vskip .1in 
\startsubsubsection{Declaration}{
\fbox{\vbox{
\hbox{\vbox{\small public 
class 
Wizard}}
\noindent\hbox{\vbox{{\bf extends} LivingUnit}}
}}}

% Removed by DocsPostProcessor:
% \startsubsubsection{Constructors}{
% \vskip -2em
% \begin{itemize}
% \item{\vskip -1.9ex 
% \membername{Wizard}
% {\tt public {\bf Wizard}( {\tt long } {\bf id},
% {\tt double } {\bf x},
% {\tt double } {\bf y},
% {\tt double } {\bf speedX},
% {\tt double } {\bf speedY},
% {\tt double } {\bf angle},
% {\tt Faction } {\bf faction},
% {\tt double } {\bf radius},
% {\tt int } {\bf life},
% {\tt int } {\bf maxLife},
% {\tt Status[]} {\bf statuses},
% {\tt long } {\bf ownerPlayerId},
% {\tt boolean } {\bf me},
% {\tt int } {\bf mana},
% {\tt int } {\bf maxMana},
% {\tt double } {\bf visionRange},
% {\tt double } {\bf castRange},
% {\tt int } {\bf xp},
% {\tt int } {\bf level},
% {\tt SkillType[]} {\bf skills},
% {\tt int } {\bf remainingActionCooldownTicks},
% {\tt int[]} {\bf remainingCooldownTicksByAction},
% {\tt boolean } {\bf master},
% {\tt Message[]} {\bf messages} )
% \label{l402}\label{l403}}%end signature
% }%end item
% \end{itemize}
% }
% \\ Removed by DocsPostProcessor.

\startsubsubsection{Methods}{
\vskip -2em
\begin{itemize}
\item{\vskip -1.9ex 
\membername{getCastRange}
{\tt public double {\bf getCastRange}(  )
\label{l404}\label{l405}}%end signature
\begin{itemize}
\sld
\item{{\bf Returns} - 
Возвращает максимальное расстояние (от центра волшебника),
 которое может преодолеть выпущенный им магический снаряд.
 \bl 
 Также является максимально возможной дальностью применения заклинаний, накладывающих на цель магический статус
 ({\tt HASTE} и {\tt SHIELD}). 
}%end item
\end{itemize}
}%end item
\divideents{getLevel}
\item{\vskip -1.9ex 
\membername{getLevel}
{\tt public int {\bf getLevel}(  )
\label{l406}\label{l407}}%end signature
\begin{itemize}
\sld
\item{{\bf Returns} - 
Возвращает текущий уровень волшебника.
 \bl 
 Начальный уровень каждого волшебника равен {\tt 0}, а максимальный --- {\tt game.levelUpXpValues.length}.
 \bl 
 В некоторых режимах игры рост уровня волшебника может быть заблокирован. 
}%end item
\end{itemize}
}%end item
\divideents{getMana}
\item{\vskip -1.9ex 
\membername{getMana}
{\tt public int {\bf getMana}(  )
\label{l408}\label{l409}}%end signature
\begin{itemize}
\sld
\item{{\bf Returns} - 
Возвращает текущее количество магической энергии волшебника. 
}%end item
\end{itemize}
}%end item
\divideents{getMaxMana}
\item{\vskip -1.9ex 
\membername{getMaxMana}
{\tt public int {\bf getMaxMana}(  )
\label{l410}\label{l411}}%end signature
\begin{itemize}
\sld
\item{{\bf Returns} - 
Возвращает максимальное количество магической энергии волшебника. 
}%end item
\end{itemize}
}%end item
\divideents{getMessages}
\item{\vskip -1.9ex 
\membername{getMessages}
{\tt public Message[] {\bf getMessages}(  )
\label{l412}\label{l413}}%end signature
\begin{itemize}
\sld
\item{{\bf Returns} - 
Возвращает сообщения, предназначенные данному волшебнику, если есть право на их просмотр.
 \bl 
 Стратегия может просматривать только сообщения, адресатом которых является управляемый ею волшебник. 
}%end item
\end{itemize}
}%end item
\divideents{getOwnerPlayerId}
\item{\vskip -1.9ex 
\membername{getOwnerPlayerId}
{\tt public long {\bf getOwnerPlayerId}(  )
\label{l414}\label{l415}}%end signature
\begin{itemize}
\sld
\item{{\bf Returns} - 
Возвращает идентификатор игрока, которому принадлежит волшебник. 
}%end item
\end{itemize}
}%end item
\divideents{getRemainingActionCooldownTicks}
\item{\vskip -1.9ex 
\membername{getRemainingActionCooldownTicks}
{\tt public int {\bf getRemainingActionCooldownTicks}(  )
\label{l416}\label{l417}}%end signature
\begin{itemize}
\sld
\item{{\bf Returns} - 
Возвращает количество тиков, оставшееся до любого следующего действия.
 \bl 
 Для совершения произвольного действия {\tt actionType} необходимо, чтобы оба значения
 {\tt remainingActionCooldownTicks} и {\tt remainingCooldownTicksByAction[actionType.ordinal()]} были равны
 нулю. 
}%end item
\end{itemize}
}%end item
\divideents{getRemainingCooldownTicksByAction}
\item{\vskip -1.9ex 
\membername{getRemainingCooldownTicksByAction}
{\tt public int[] {\bf getRemainingCooldownTicksByAction}(  )
\label{l418}\label{l419}}%end signature
\begin{itemize}
\sld
\item{{\bf Returns} - 
Возвращает массив целых неотрицательных чисел. Каждая ячейка массива содержит значение количества тиков,
 оставшегося до совершения следующего действия с соответствующим индексом.
 \bl 
 Например, {\tt remainingCooldownTicksByAction[0]} соответствует действию {\tt NONE} и всегда равно нулю.
 {\tt remainingCooldownTicksByAction[1]} соответствует действию {\tt STAFF} и равно количеству тиков,
 оставшемуся до совершения данного действия. {\tt remainingCooldownTicksByAction[2]} соответствует действию
 {\tt MAGIC\_MISSILE} и так далее.
 \bl 
 Для совершения произвольного действия {\tt actionType} необходимо, чтобы оба значения
 {\tt remainingActionCooldownTicks} и {\tt remainingCooldownTicksByAction[actionType.ordinal()]} были равны
 нулю. 
}%end item
\end{itemize}
}%end item
\divideents{getSkills}
\item{\vskip -1.9ex 
\membername{getSkills}
{\tt public SkillType[] {\bf getSkills}(  )
\label{l420}\label{l421}}%end signature
\begin{itemize}
\sld
\item{{\bf Returns} - 
Возвращает умения, изученные волшебником. 
}%end item
\end{itemize}
}%end item
\divideents{getVisionRange}
\item{\vskip -1.9ex 
\membername{getVisionRange}
{\tt public double {\bf getVisionRange}(  )
\label{l422}\label{l423}}%end signature
\begin{itemize}
\sld
\item{{\bf Returns} - 
Возвращает максимальное расстояние (от центра до центра),
 на котором данный волшебник обнаруживает другие объекты. 
}%end item
\end{itemize}
}%end item
\divideents{getXp}
\item{\vskip -1.9ex 
\membername{getXp}
{\tt public int {\bf getXp}(  )
\label{l424}\label{l425}}%end signature
\begin{itemize}
\sld
\item{{\bf Returns} - 
Возвращает количество очков опыта, полученное волшебником в процессе игры. 
}%end item
\end{itemize}
}%end item
\divideents{isMaster}
\item{\vskip -1.9ex 
\membername{isMaster}
{\tt public boolean {\bf isMaster}(  )
\label{l426}\label{l427}}%end signature
\begin{itemize}
\sld
\item{{\bf Returns} - 
Возвращает {\tt true} в том и только том случае, если этот волшебник является верховным.
 \bl 
 Количество верховных волшебников в каждой фракции строго равно одному. 
}%end item
\end{itemize}
}%end item
\divideents{isMe}
\item{\vskip -1.9ex 
\membername{isMe}
{\tt public boolean {\bf isMe}(  )
\label{l428}\label{l429}}%end signature
\begin{itemize}
\sld
\item{{\bf Returns} - 
Возвращает {\tt true} в том и только том случае, если этот волшебник ваш. 
}%end item
\end{itemize}
}%end item
\end{itemize}
}
\hide{inherited}{
\startsubsubsection{Methods inherited from class {\tt LivingUnit}}{
\par{\small 
\refdefined{l9}\vskip -2em
\begin{itemize}
\item{\vskip -1.9ex 
\membername{getLife}
{\tt public int {\bf getLife}(  )
}%end signature
\begin{itemize}
\sld
\item{{\bf Returns} - 
Возвращает текущее количество жизненной энергии. 
}%end item
\end{itemize}
}%end item
\divideents{getMaxLife}
\item{\vskip -1.9ex 
\membername{getMaxLife}
{\tt public int {\bf getMaxLife}(  )
}%end signature
\begin{itemize}
\sld
\item{{\bf Returns} - 
Возвращает максимальное количество жизненной энергии. 
}%end item
\end{itemize}
}%end item
\divideents{getStatuses}
\item{\vskip -1.9ex 
\membername{getStatuses}
{\tt public Status[] {\bf getStatuses}(  )
}%end signature
\begin{itemize}
\sld
\item{{\bf Returns} - 
Возвращает магические статусы, влияющие на живого юнита. 
}%end item
\end{itemize}
}%end item
\end{itemize}
}}
\startsubsubsection{Methods inherited from class {\tt CircularUnit}}{
\par{\small 
\refdefined{l5}\vskip -2em
\begin{itemize}
\item{\vskip -1.9ex 
\membername{getRadius}
{\tt public double {\bf getRadius}(  )
}%end signature
\begin{itemize}
\sld
\item{{\bf Returns} - 
Возвращает радиус объекта. 
}%end item
\end{itemize}
}%end item
\end{itemize}
}}
\startsubsubsection{Methods inherited from class {\tt Unit}}{
\par{\small 
\refdefined{l21}\vskip -2em
\begin{itemize}
\item{\vskip -1.9ex 
\membername{getAngle}
{\tt public final double {\bf getAngle}(  )
}%end signature
\begin{itemize}
\sld
\item{{\bf Returns} - 
Возвращает угол поворота объекта в радианах. Нулевой угол соответствует направлению оси абсцисс.
 Положительные значения соответствуют повороту по часовой стрелке. 
}%end item
\end{itemize}
}%end item
\divideents{getAngleTo}
\item{\vskip -1.9ex 
\membername{getAngleTo}
{\tt public double {\bf getAngleTo}( {\tt double } {\bf x},
{\tt double } {\bf y} )
}%end signature
\begin{itemize}
\sld
\item{
\sld
{\bf Parameters}
\sld\isep
  \begin{itemize}
\sld\isep
   \item{
\sld
{\tt x} - X-координата точки.}
   \item{
\sld
{\tt y} - Y-координата точки.}
  \end{itemize}
}%end item
\item{{\bf Returns} - 
Возвращает ориентированный угол [{\tt -PI}, {\tt PI}] между направлением
 данного объекта и вектором из центра данного объекта к указанной точке. 
}%end item
\end{itemize}
}%end item
\divideents{getAngleTo}
\item{\vskip -1.9ex 
\membername{getAngleTo}
{\tt public double {\bf getAngleTo}( {\tt Unit } {\bf unit} )
}%end signature
\begin{itemize}
\sld
\item{
\sld
{\bf Parameters}
\sld\isep
  \begin{itemize}
\sld\isep
   \item{
\sld
{\tt unit} - Объект, к центру которого необходимо определить угол.}
  \end{itemize}
}%end item
\item{{\bf Returns} - 
Возвращает ориентированный угол [{\tt -PI}, {\tt PI}] между направлением
 данного объекта и вектором из центра данного объекта к центру указанного объекта. 
}%end item
\end{itemize}
}%end item
\divideents{getDistanceTo}
\item{\vskip -1.9ex 
\membername{getDistanceTo}
{\tt public double {\bf getDistanceTo}( {\tt double } {\bf x},
{\tt double } {\bf y} )
}%end signature
\begin{itemize}
\sld
\item{
\sld
{\bf Parameters}
\sld\isep
  \begin{itemize}
\sld\isep
   \item{
\sld
{\tt x} - X-координата точки.}
   \item{
\sld
{\tt y} - Y-координата точки.}
  \end{itemize}
}%end item
\item{{\bf Returns} - 
Возвращает расстояние до точки от центра данного объекта. 
}%end item
\end{itemize}
}%end item
\divideents{getDistanceTo}
\item{\vskip -1.9ex 
\membername{getDistanceTo}
{\tt public double {\bf getDistanceTo}( {\tt Unit } {\bf unit} )
}%end signature
\begin{itemize}
\sld
\item{
\sld
{\bf Parameters}
\sld\isep
  \begin{itemize}
\sld\isep
   \item{
\sld
{\tt unit} - Объект, до центра которого необходимо определить расстояние.}
  \end{itemize}
}%end item
\item{{\bf Returns} - 
Возвращает расстояние от центра данного объекта до центра указанного объекта. 
}%end item
\end{itemize}
}%end item
\divideents{getFaction}
\item{\vskip -1.9ex 
\membername{getFaction}
{\tt public Faction {\bf getFaction}(  )
}%end signature
\begin{itemize}
\sld
\item{{\bf Returns} - 
Возвращает фракцию, к которой принадлежит данный юнит. 
}%end item
\end{itemize}
}%end item
\divideents{getId}
\item{\vskip -1.9ex 
\membername{getId}
{\tt public long {\bf getId}(  )
}%end signature
\begin{itemize}
\sld
\item{{\bf Returns} - 
Возвращает уникальный идентификатор объекта. 
}%end item
\end{itemize}
}%end item
\divideents{getSpeedX}
\item{\vskip -1.9ex 
\membername{getSpeedX}
{\tt public final double {\bf getSpeedX}(  )
}%end signature
\begin{itemize}
\sld
\item{{\bf Returns} - 
Возвращает X-составляющую скорости объекта. Ось абсцисс направлена слева направо.
 \bl 
 Для юнитов, способных мгновенно менять свою скорость, возвращается значение перемещения за последний тик. 
}%end item
\end{itemize}
}%end item
\divideents{getSpeedY}
\item{\vskip -1.9ex 
\membername{getSpeedY}
{\tt public final double {\bf getSpeedY}(  )
}%end signature
\begin{itemize}
\sld
\item{{\bf Returns} - 
Возвращает Y-составляющую скорости объекта. Ось ординат направлена сверху вниз.
 \bl 
 Для юнитов, способных мгновенно менять свою скорость, возвращается значение перемещения за последний тик. 
}%end item
\end{itemize}
}%end item
\divideents{getX}
\item{\vskip -1.9ex 
\membername{getX}
{\tt public final double {\bf getX}(  )
}%end signature
\begin{itemize}
\sld
\item{{\bf Returns} - 
Возвращает X-координату центра объекта. Ось абсцисс направлена слева направо. 
}%end item
\end{itemize}
}%end item
\divideents{getY}
\item{\vskip -1.9ex 
\membername{getY}
{\tt public final double {\bf getY}(  )
}%end signature
\begin{itemize}
\sld
\item{{\bf Returns} - 
Возвращает Y-координату центра объекта. Ось ординат направлена сверху вниз. 
}%end item
\end{itemize}
}%end item
\end{itemize}
}}
}
}
\startsection{Class}{World}{l23}{%
{\small Этот класс описывает игровой мир. Содержит также описания всех игроков и игровых объектов (<<юнитов>>).}
\vskip .1in 
\startsubsubsection{Declaration}{
\fbox{\vbox{
\hbox{\vbox{\small public 
class 
World}}
\noindent\hbox{\vbox{{\bf extends} Object}}
}}}

% Removed by DocsPostProcessor:
% \startsubsubsection{Constructors}{
% \vskip -2em
% \begin{itemize}
% \item{\vskip -1.9ex 
% \membername{World}
% {\tt public {\bf World}( {\tt int } {\bf tickIndex},
% {\tt int } {\bf tickCount},
% {\tt double } {\bf width},
% {\tt double } {\bf height},
% {\tt Player[]} {\bf players},
% {\tt Wizard[]} {\bf wizards},
% {\tt Minion[]} {\bf minions},
% {\tt Projectile[]} {\bf projectiles},
% {\tt Bonus[]} {\bf bonuses},
% {\tt Building[]} {\bf buildings},
% {\tt Tree[]} {\bf trees} )
% \label{l430}\label{l431}}%end signature
% }%end item
% \end{itemize}
% }
% \\ Removed by DocsPostProcessor.

\startsubsubsection{Methods}{
\vskip -2em
\begin{itemize}
\item{\vskip -1.9ex 
\membername{getBonuses}
{\tt public Bonus[] {\bf getBonuses}(  )
\label{l432}\label{l433}}%end signature
\begin{itemize}
\sld
\item{{\bf Returns} - 
Возвращает список видимых бонусов (в случайном порядке).
 После каждого тика объекты, задающие бонусы, пересоздаются. 
}%end item
\end{itemize}
}%end item
\divideents{getBuildings}
\item{\vskip -1.9ex 
\membername{getBuildings}
{\tt public Building[] {\bf getBuildings}(  )
\label{l434}\label{l435}}%end signature
\begin{itemize}
\sld
\item{{\bf Returns} - 
Возвращает список видимых строений (в случайном порядке).
 После каждого тика объекты, задающие строения, пересоздаются. 
}%end item
\end{itemize}
}%end item
\divideents{getHeight}
\item{\vskip -1.9ex 
\membername{getHeight}
{\tt public double {\bf getHeight}(  )
\label{l436}\label{l437}}%end signature
\begin{itemize}
\sld
\item{{\bf Returns} - 
Возвращает высоту мира. 
}%end item
\end{itemize}
}%end item
\divideents{getMinions}
\item{\vskip -1.9ex 
\membername{getMinions}
{\tt public Minion[] {\bf getMinions}(  )
\label{l438}\label{l439}}%end signature
\begin{itemize}
\sld
\item{{\bf Returns} - 
Возвращает список видимых последователей (в случайном порядке).
 После каждого тика объекты, задающие последователей, пересоздаются. 
}%end item
\end{itemize}
}%end item
\divideents{getMyPlayer}
\item{\vskip -1.9ex 
\membername{getMyPlayer}
{\tt public Player {\bf getMyPlayer}(  )
\label{l440}\label{l441}}%end signature
\begin{itemize}
\sld
\item{{\bf Returns} - 
Возвращает вашего игрока. 
}%end item
\end{itemize}
}%end item
\divideents{getPlayers}
\item{\vskip -1.9ex 
\membername{getPlayers}
{\tt public Player[] {\bf getPlayers}(  )
\label{l442}\label{l443}}%end signature
\begin{itemize}
\sld
\item{{\bf Returns} - 
Возвращает список игроков (в случайном порядке).
 После каждого тика объекты, задающие игроков, пересоздаются. 
}%end item
\end{itemize}
}%end item
\divideents{getProjectiles}
\item{\vskip -1.9ex 
\membername{getProjectiles}
{\tt public Projectile[] {\bf getProjectiles}(  )
\label{l444}\label{l445}}%end signature
\begin{itemize}
\sld
\item{{\bf Returns} - 
Возвращает список видимых магических снарядов (в случайном порядке).
 После каждого тика объекты, задающие снаряды, пересоздаются. 
}%end item
\end{itemize}
}%end item
\divideents{getTickCount}
\item{\vskip -1.9ex 
\membername{getTickCount}
{\tt public int {\bf getTickCount}(  )
\label{l446}\label{l447}}%end signature
\begin{itemize}
\sld
\item{{\bf Returns} - 
Возвращает базовую длительность игры в тиках. Реальная длительность может отличаться от этого значения в
 меньшую сторону. Эквивалентно {\tt game.tickCount}. 
}%end item
\end{itemize}
}%end item
\divideents{getTickIndex}
\item{\vskip -1.9ex 
\membername{getTickIndex}
{\tt public int {\bf getTickIndex}(  )
\label{l448}\label{l449}}%end signature
\begin{itemize}
\sld
\item{{\bf Returns} - 
Возвращает номер текущего тика. 
}%end item
\end{itemize}
}%end item
\divideents{getTrees}
\item{\vskip -1.9ex 
\membername{getTrees}
{\tt public Tree[] {\bf getTrees}(  )
\label{l450}\label{l451}}%end signature
\begin{itemize}
\sld
\item{{\bf Returns} - 
Возвращает список видимых деревьев (в случайном порядке).
 После каждого тика объекты, задающие деревья, пересоздаются. 
}%end item
\end{itemize}
}%end item
\divideents{getWidth}
\item{\vskip -1.9ex 
\membername{getWidth}
{\tt public double {\bf getWidth}(  )
\label{l452}\label{l453}}%end signature
\begin{itemize}
\sld
\item{{\bf Returns} - 
Возвращает ширину мира. 
}%end item
\end{itemize}
}%end item
\divideents{getWizards}
\item{\vskip -1.9ex 
\membername{getWizards}
{\tt public Wizard[] {\bf getWizards}(  )
\label{l454}\label{l455}}%end signature
\begin{itemize}
\sld
\item{{\bf Returns} - 
Возвращает список видимых волшебников (в случайном порядке).
 После каждого тика объекты, задающие волшебников, пересоздаются. 
}%end item
\end{itemize}
}%end item
\end{itemize}
}
\hide{inherited}{
}
}
}
}
\newpage
\def\packagename{\textless none\textgreater }
\chapter{\bf Package \textless none\textgreater }{
\vskip -.25in
\hbox to \hsize{\it Package Contents\hfil Page}
\rule{\hsize}{.7mm}
\vskip .13in
\hbox{\bf Interfaces}
\entityintro{Strategy}{l456}{Стратегия --- интерфейс, содержащий описание методов искусственного интеллекта волшебника.}
\vskip .1in
\rule{\hsize}{.7mm}
\vskip .1in
\newpage
\section{Interfaces}{
\startsection{Interface}{Strategy}{l456}{%
{\small Стратегия --- интерфейс, содержащий описание методов искусственного интеллекта волшебника.
 Каждая пользовательская стратегия должна реализовывать этот интерфейс.
 Может отсутствовать в некоторых языковых пакетах, если язык не поддерживает интерфейсы.}
\vskip .1in 
\startsubsubsection{Declaration}{
\fbox{\vbox{
\hbox{\vbox{\small public interface 
Strategy}}
}}}
\startsubsubsection{Methods}{
\vskip -2em
\begin{itemize}
\item{\vskip -1.9ex 
\membername{move}
{\tt public void {\bf move}( {\tt Wizard } {\bf self},
{\tt World } {\bf world},
{\tt Game } {\bf game},
{\tt Move } {\bf move} )
\label{l457}\label{l458}}%end signature
\begin{itemize}
\sld
\item{
\sld
{\bf Usage}
  \begin{itemize}\isep
   \item{
Основной метод стратегии, осуществляющий управление волшебником.
 Вызывается каждый тик для каждого волшебника.
}%end item
  \end{itemize}
}
\item{
\sld
{\bf Parameters}
\sld\isep
  \begin{itemize}
\sld\isep
   \item{
\sld
{\tt self} - Волшебник, которым данный метод будет осуществлять управление.}
   \item{
\sld
{\tt world} - Текущее состояние мира.}
   \item{
\sld
{\tt game} - Различные игровые константы.}
   \item{
\sld
{\tt move} - Результатом работы метода является изменение полей данного объекта.}
  \end{itemize}
}%end item
\end{itemize}
}%end item
\end{itemize}
}
\hide{inherited}{
}
}
}
}
\end{document}
